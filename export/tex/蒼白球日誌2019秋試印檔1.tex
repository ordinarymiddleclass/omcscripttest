% Options for packages loaded elsewhere
\PassOptionsToPackage{unicode}{hyperref}
\PassOptionsToPackage{hyphens}{url}
%
\documentclass[a5paper, 12pt
]{book}
\renewcommand{\baselinestretch}{1.3} 
\usepackage{sectsty}
\sectionfont{\clearpage}
\usepackage[utf8]{inputenc}
\usepackage[inner=1in, outer=1.5cm, bottom=1in, top=1in, headheight=13.6pt]{geometry}
\usepackage{lmodern}
\usepackage{amssymb,amsmath}
\usepackage{ifxetex,ifluatex}
\usepackage {xeCJK}
\setCJKmainfont {Noto Serif CJK TC}
\usepackage{fancyhdr}
\pagestyle{fancy}
\fancyhf{}
\fancyfoot[CE,CO]{\rightmark}
\fancyhead[LE,RO]{\thepage}
\fancyhead[RE,LO]{蒼白球日誌2019秋}
\renewcommand{\headrulewidth}{2pt}
\renewcommand{\footrulewidth}{1pt}
\ifnum 0\ifxetex 1\fi\ifluatex 1\fi=0 % if pdftex
  \usepackage[T1]{fontenc}
  \usepackage[utf8]{inputenc}
  \usepackage{textcomp} % provide euro and other symbols
\else % if luatex or xetex
  \usepackage{unicode-math}
  \defaultfontfeatures{Scale=MatchLowercase}
  \defaultfontfeatures[\rmfamily]{Ligatures=TeX,Scale=1}
\fi
% Use upquote if available, for straight quotes in verbatim environments
\IfFileExists{upquote.sty}{\usepackage{upquote}}{}
\IfFileExists{microtype.sty}{% use microtype if available
  \usepackage[]{microtype}
  \UseMicrotypeSet[protrusion]{basicmath} % disable protrusion for tt fonts
}{}
\makeatletter
\@ifundefined{KOMAClassName}{% if non-KOMA class
  \IfFileExists{parskip.sty}{%
    \usepackage{parskip}
  }{% else
    \setlength{\parindent}{0pt}
    \setlength{\parskip}{6pt plus 2pt minus 1pt}}
}{% if KOMA class
  \KOMAoptions{parskip=half}}
\makeatother
\usepackage{xcolor}
\IfFileExists{xurl.sty}{\usepackage{xurl}}{} % add URL line breaks if available
\IfFileExists{bookmark.sty}{\usepackage{bookmark}}{\usepackage{hyperref}}
\hypersetup{
  hidelinks,
  pdfcreator={LaTeX via pandoc}}
\urlstyle{same} % disable monospaced font for URLs
\setlength{\emergencystretch}{3em} % prevent overfull lines
\providecommand{\tightlist}{%
  \setlength{\itemsep}{0pt}\setlength{\parskip}{0pt}}
\setcounter{secnumdepth}{-\maxdimen} % remove section numbering
\renewcommand\contentsname{目錄}
\setcounter{tocdepth}{1}
\date{}

\begin{document}
\tableofcontents
\clearpage

\hypertarget{ux84bcux767dux7403ux65e5ux8a8c000120190929}{%
\section{蒼白球日誌0001(20190929)}\label{ux84bcux767dux7403ux65e5ux8a8c000120190929}}

\hypertarget{ux65e5ux671f-date}{%
\subsubsection{日期 Date}\label{ux65e5ux671f-date}}

\begin{itemize}
\tightlist
\item
  世界協調時間 2019年09月29日 / Unix紀元 18168 日 / 星期日
\item
  September 29, 2019 (UTC) / 18168 days since Unix Epoch / Sunday
\end{itemize}

\hypertarget{ux5e74ux9f61-age}{%
\subsubsection{年齡 Age}\label{ux5e74ux9f61-age}}

\begin{itemize}
\tightlist
\item
  33歲5個月7天
\item
  33 years 5 months 7 days old
\end{itemize}

\hypertarget{ux672cux6587-content}{%
\subsubsection{本文 Content}\label{ux672cux6587-content}}

這周是這個被操爆的一年中,被操得最爆的一周,所以才會到禮拜天才有空把所有放日記的帳號設定好。一本日記同時放到Blogspot、Github、Dropbox、Google
Drive跟Hackmd上面,應該暫時算是保險的吧?Google、微軟跟Dropbox同時倒掉應該短期內不會發生吧?反正我暫時不想要存紙本,就先這樣吧。

回歸正題,到底被操得有多爆呢?這麼爆。禮拜三坐著車子去B院開會,趁在B院的空檔完成數位病理專欄後,搭車回本部,做顯微鏡診斷做到半夜兩點,然後睡一下,禮拜四早上七點去開會,接著繼續做診斷到晚上十點。再加上禮拜六跟母后一起寫計畫,身心根本就是到了一個極限。

然後我身上還有很多研究的事情延宕中,還沒做,超級度日如年,度日如年到什麼程度呢?我以為空氣清淨機已經買八個月,該換濾心了,結果查信用卡交易紀錄,居然才三個月。

所以在這個情形下我真的無法理解我的同儕朋友們,他們竟然有餘力關注跟參加今天的挺香港反送中遊行,是日常生活不夠累嗎?那就讓他們繼續去發洩多餘的體力好了,我已經累到,在我可以繼續這樣過活的前提下,蔡英文政府的各種施政議題我實在是沒有餘力思考了。中華民國體制啊,拜託你繼續維持好好的狀態,我沒有力氣關心你了。

\hypertarget{ux84bcux767dux7403ux65e5ux8a8c000220190930}{%
\section{蒼白球日誌0002(20190930)}\label{ux84bcux767dux7403ux65e5ux8a8c000220190930}}

\hypertarget{ux65e5ux671f-date-1}{%
\subsubsection{日期 Date}\label{ux65e5ux671f-date-1}}

\begin{itemize}
\tightlist
\item
  世界協調時間 2019年09月30日 / Unix紀元 18166 日 / 星期一
\item
  September 30, 2019 (UTC) / 18169 days since Unix Epoch / Sunday
\end{itemize}

\hypertarget{ux5e74ux9f61-age-1}{%
\subsubsection{年齡 Age}\label{ux5e74ux9f61-age-1}}

\begin{itemize}
\tightlist
\item
  33歲5個月8天
\item
  33 years 5 months 8 days old
\end{itemize}

\hypertarget{ux672cux6587-content-1}{%
\subsubsection{本文 Content}\label{ux672cux6587-content-1}}

結束三個月的混戰期後,終於可以開始處理明明十二月就得結案,卻從六月拖延到現在的研究計劃事宜。在這個計畫裡面最不妙的事情倒不是研究本身,而是經費概算表:我需要的研究材料,很明顯的跟當時所編的預算不符,當時登錄的研究人員也有一些無法請領研究費。聽到我的這個問題,醫研部居然給了一個充滿溫度的回應:『院內計畫是一個讓大家練習如何執行計畫的平台,所以我不會要求你一定要上簽呈才能修正預算表,請在預算總額不變的前提下,提交一個更新的經費概算表即可。』雖然這間醫院名聲不好,但是某些事情的執行上,還是有點人性的\ldots\ldots 吧?

以為時間很有餘裕的我,不但利用時間處理了經費問題,還進行了一連串的生活重整:整理了辦公室,整理了本月的行事曆跟工作排程,藉由這個重新整理時間軸與空間軸的過程中,為原本亂七八糟的班表除錯,把排班喬到避開B院評鑑跟PBL教學的時間。正當想說可以慢慢把一切都回歸正軌時,總醫師突然提醒我了一件很可怕的事實:

\textbf{我明天要報告!而且我沒有準備!}

我身為一個自認很會,超會報告期刊,能力凌駕整個病理科的人,絕對不可以因為沒有預先準備報告就哀求總醫師取消,因此我在要求明天照常報告以後,明天的報告跟原本今天要做的,母后工作上的文件,兩個又疊在一起了。我的生活又暫時亂掉了。

算了,這批的報告時效好像還好,就挪一點白天的時間給母后吧。還能怎麼樣呢?

\hypertarget{ux84bcux767dux7403ux65e5ux8a8c000320191001}{%
\section{蒼白球日誌0003(20191001)}\label{ux84bcux767dux7403ux65e5ux8a8c000320191001}}

\hypertarget{ux65e5ux671f-date-2}{%
\subsubsection{日期 Date}\label{ux65e5ux671f-date-2}}

\begin{itemize}
\tightlist
\item
  世界協調時間 2019年10月1日 / Unix紀元 18170 日 / 星期二 /
  日記第一頁以來第3日
\item
  October 1, 2019 (UTC) / 18170 days since Unix Epoch / Tuesday / Diary
  day 3
\end{itemize}

\hypertarget{ux5e74ux9f61-age-2}{%
\subsubsection{年齡 Age}\label{ux5e74ux9f61-age-2}}

\begin{itemize}
\tightlist
\item
  33 歲 5 個月 9 天
\item
  33 years 5 months 9 days old
\end{itemize}

\hypertarget{ux672cux6587-content-2}{%
\subsubsection{本文 Content}\label{ux672cux6587-content-2}}

因為我學姊與學妹同時考上專科醫師(註一),因此學姊、學妹跟學弟(註二)一起請大家吃茹絲葵(註三)。這可是大手筆的請客,原本想要放輕鬆地吃一頓飯,結果因為部主任有參加,大師兄就跟主任在餐桌上談嚴肅的公事,實在有夠敗興。明明是好吃的昂貴料理,在這種氛圍下也實在不是很開心。

幸好最後的甜點真的很精彩,香蕉奶油派跟蘋果派都美味又富有層次,把所有人的焦點都轉回到食物。可惜的是因為大家都吃很飽了,最後蘋果派吃不完,於是我一個人獨吞了半個六吋蘋果派。太爽了。

雖然以高級蘋果派結束應該要是完美的一天了,但是還是有美中不足的地方。今天下午我又被興大的工程研究生問了一堆「病理醫師如何判讀」的知識,超魯洨超煩的,明明用類神經網路訓練機器診斷的流行,就是「我給你粗資料,你給我求出暴力解」啊,你幹嘛跟我討一堆病理醫師的看法,暴力解才夯嘛,加了專家系統的料感覺就落伍了。但是或許我是錯的,或許還是要給一些專家意識會比較好?算了,先做再說。

註一:學姊考了三年,所以他會跟學妹一起考上\ldots\ldots 真是辛苦他了

註二:學弟去年考上沒請客,今年補請

註三:Ruth Chris
Steakhouse,位於中華民國境內台灣島上台中市的著名老牌西餐廳,嗯,這樣應該解釋得很清楚了,如果看到這篇的人已經不知道什麼是中華民國也不知道什麼是台中市了,那東亞應該也毀滅得差不多了\ldots\ldots{}

\hypertarget{ux84bcux767dux7403ux65e5ux8a8c000420191002}{%
\section{蒼白球日誌0004(20191002)}\label{ux84bcux767dux7403ux65e5ux8a8c000420191002}}

\hypertarget{ux65e5ux671f-date-3}{%
\subsubsection{日期 Date}\label{ux65e5ux671f-date-3}}

\begin{itemize}
\tightlist
\item
  世界協調時間 2019年10月2日 / Unix紀元 18171 日 / 星期三 /
  日記第一頁以來第4日
\item
  October 2, 2019 (UTC) / 18171 days since Unix Epoch / Wednesday /
  Diary day 4
\end{itemize}

\hypertarget{ux5e74ux9f61-age-3}{%
\subsubsection{年齡 Age}\label{ux5e74ux9f61-age-3}}

\begin{itemize}
\tightlist
\item
  33 歲 5 個月 10 天
\item
  33 years 5 months 10 days old
\end{itemize}

\hypertarget{ux672cux6587-content-3}{%
\subsubsection{本文 Content}\label{ux672cux6587-content-3}}

\begin{enumerate}
\def\labelenumi{\arabic{enumi}.}
\item
  網路購物

  前天半夜在momo網路購物(註一)下訂了知名日本廠牌tunemaker的小小幾罐保養品,因為覺得網購就是同時把想要的東西全部買齊比較不用擔心運費,所以就順便下訂了沐浴乳、幾條牙膏跟一盒牙線。結果今天宅配到辦公室時,出現的居然是四大箱紙箱,一箱裝沐浴乳跟保養品,兩箱裝牙膏,一箱裝牙線,然後所有箱子的大小都是內容物的三倍高,感覺好像訂了什麼大型物體一樣。科秘書看到這個場景不禁恥笑了我兩下,一笑不環保,二笑我懶到連牙膏都要網購。

  同時,我學姊看到我買保養品的行為,給了我一個良心建議:應該先去診所用雷射把臉上的缺陷打掉,再來用保養品。但我暫時不想花那麼多錢在臉上,所以雖然這個建議真的很合理,我還是決定不要照做。
\item
  B分院院長的請託

  因為B分院要評鑑,所以很機車的B分院院長前幾天又重提了一個無理的要求,就是想要知道B院有那些外科術式與診斷不符(註二)的個案,可以在評鑑的時候拿出來檢討,為了這個甚至還透過我們主任對我施壓。平常在B院的醫師J醫師(註三)想必是每年都跟他說沒有這種資料,他應該忍很多年了,因此他一定是想藉由暫時換人的機會施壓看看,可不可以拿到以前沒有的東西。

  其實J醫師拿不出這資料真的不是刁難他,要拿到這個東西難度非常高,用一般的關鍵字搜尋幾乎不可能做到。但是我身為科內最自以為資料人的病理醫師,怎麼可以因為用關鍵字搜尋不到就被打倒?絕對不能說拿不出來。但是,要怎麼做?

  在一般機器運算豪無用武之地的狀況下,我採用了一個以前曾經拿來作弊的方式:就是先思考一下病理醫師的常識。我回想了在過去幾年的病理經驗中,什麼樣的檢體是最容易被開壞的,用這些檢體種類去搜尋,結果果然找到了好幾例開刀開得非常不恰當的個案,寄給B院院長。好,我給了B院長他引頸企盼的的內鬥武器跟評鑑談資,他要拿來對付誰,就看他了,我看好戲就好了。
\end{enumerate}

註一:網路購物係指以網際網路為媒介傳送數位訂單,讓公司依據此數位訂單寄出貨品
。momo購物為中華民國所轄台灣島境內著名的網路購物平台,屬於東森公司。

註二:指從事後得到的病理診斷而言,其手術方式不恰當者。例如說診斷非惡性腫瘤,且無其他器官切除適應症,卻做了器官切除的手術。

註三:J醫師出國進修一年半,這段期間我代替他守B院,他回來再交還。

\hypertarget{ux84bcux767dux7403ux65e5ux8a8c000520191003}{%
\section{蒼白球日誌0005(20191003)}\label{ux84bcux767dux7403ux65e5ux8a8c000520191003}}

\hypertarget{ux65e5ux671f-date-4}{%
\subsubsection{日期 Date}\label{ux65e5ux671f-date-4}}

\begin{itemize}
\tightlist
\item
  世界協調時間 2019年10月3日 / Unix紀元 18172 日 / 星期四 /
  蒼白球日記紀元第5日
\item
  October 3, 2019 (UTC) / 18172 days since Unix Epoch / Thursday /
  PaleBall Diary day 5
\item
  特殊註記:B院評鑑倒數15天
\end{itemize}

\hypertarget{ux5e74ux9f61-age-4}{%
\subsubsection{年齡 Age}\label{ux5e74ux9f61-age-4}}

\begin{itemize}
\tightlist
\item
  33 歲 5 個月 11 天
\item
  33 years 5 months 11 days old
\end{itemize}

\hypertarget{ux672cux6587-content-4}{%
\subsubsection{本文 Content}\label{ux672cux6587-content-4}}

\begin{enumerate}
\def\labelenumi{\arabic{enumi}.}
\item
  非洲鼓教案

  此刻我正在幫母后大人寫非洲鼓(Djembe,註一)教案的其中一部份,原本以為今天晚上能夠完成這個草稿,然而沒想到資料越查越複雜,六點趕回家到現在才完成八個大項裡面的六個,真的是效率有夠差,明天可能又要被母后碎碎念了。

  可是不能怪我,在用google(註二)查資料前,沒有人會想到非洲鼓的風行,除了他方便攜帶以外,其實還有另外一個原因,就是一個新獨立國家的成功公關操作。公元1958年幾內亞脫離法國獨立,急需一個凝聚民族認同的方式,因此1960年代就汲取Djole、Kuku等傳統音樂元素,創造了一個舞團以及一整套的舞曲藝術。這不但凝聚了幾內亞的國族認同,因為這套完整的音樂發明,也讓幾內亞音樂傳到全世界,進而造就了非洲鼓的全球流行。目前市面上的所有非洲鼓教材都有Djole跟Kuku兩個單元,然而卻沒有提到這個歷史,真的是殊為可惜,這可是劉仲敬(註三)所謂民族發明學的經典案例呢。
\item
  保養品

  前幾天在跟科秘書還有學妹聊天的時候,聊到我是在聽從皮膚科醫師建議抹防曬以後,才開始長一整年都好不了的痘痘,他們聽到以後異口同聲地說「你一定是沒有卸妝!」

  聽到這句話以後我整個愣了一下,原來白天擦防曬晚上要卸妝喔?沒有人告訴我啊,難怪我痘痘會好不了。這幾天我開始使用卸妝水,皮膚感覺好多了,保養這件事情真的是博大精深啊\ldots\ldots{}
\item
  B院評鑑

  原本以為B院評鑑我可以其中一天去B院,另外一天溜回C院辦事情,結果哪那麼簡單,W院長終究是個機車龜毛人,他強烈要求我要兩天都在那邊待命,於是我只好到處求科內同事,把班表畫得面目全非,才終於調成一個可以應付評鑑的樣貌,但是後半月的工作應該還是會很慘。

  傍晚開始寫母后文件前,想到這幾天可能會是有時間餘裕的最後幾天,就決定要做一個看起來很悠閒很尊貴的事情:晚餐去第二市場(註四)吃東西。第二市場的這些飲食老店即使在觀光客湧入的狀況下,居然還是維持著一定的品質,可能因為過去這些店家曾是仕紳階級才吃得起的東西,多少還是有點貴族習氣的吧。以今天吃的店來說,茂川肉丸店的餛飩麵依然小巧精緻,沒有因為階級下降就變俗又大碗,老賴紅茶更不用說,一品青草紅茶,甫入口感覺到的是經典成功路青草街(註五)那種芳香,入喉後開始漫出老賴紅茶特有的香料味,真是經典啊,台中人從公元1975年喝到現在的味道。我覺得這些店說不定還能再戰20年。
\end{enumerate}

註一:一種圓形的手持鼓,又譯金貝鼓

註二:公元二十一世紀初最流行的網際網路搜尋引擎

註三:公元1974年生,中華人民共和國(又稱共產中國)出身的歷史學者與政治倡議者,其認為共產中國終將崩潰的論點,在漢文化地區有許多信徒,然其歷史分析亦常被學界所質疑。

註四:第二市場為1917年11月開設,位於台中市(目前為中華民國轄下,當時為大日本帝國轄下)新富町三丁目(今中區中正路、臺灣大道、中山路及興中街所圍成街廓),故稱為新富町市場,其呈三翼放射狀的本館與第一市場為同個樣式,當時耗資39,000圓。除了本館,在外側也設有簡易店鋪,以及魚、蔬菜批發市場。由於是臺中市所建設的第二個消費市場,又俗稱為第二市場。第二市場當時以精美的貨品品質、較高的售價服務於當時新富町的日本人,因此又有「日本人的市場」之稱。

註五:此街位於第二市場附近,長久以來以許多青草店著稱。

\hypertarget{ux84bcux767dux7403ux65e5ux8a8c000620191004}{%
\section{蒼白球日誌0006(20191004)}\label{ux84bcux767dux7403ux65e5ux8a8c000620191004}}

\hypertarget{ux65e5ux671f-date-5}{%
\subsubsection{日期 Date}\label{ux65e5ux671f-date-5}}

\begin{itemize}
\tightlist
\item
  世界協調時間 2019年(中華民國108年)10月4日 / Unix紀元 18173 日 / 星期五
  / 蒼白球紀元第6日
\item
  October 4, 2019 (UTC) / 18173 days since Unix Epoch / Friday / Globus
  Pallidum day 6
\item
  特殊註記:B院評鑑倒數14天
\end{itemize}

\hypertarget{ux5e74ux9f61-age-5}{%
\subsubsection{年齡 Age}\label{ux5e74ux9f61-age-5}}

\begin{itemize}
\tightlist
\item
  33 歲 5 個月 12 天
\item
  33 years 5 months 12 days old
\end{itemize}

\hypertarget{ux672cux6587-content-5}{%
\subsubsection{本文 Content}\label{ux672cux6587-content-5}}

\begin{enumerate}
\def\labelenumi{\arabic{enumi}.}
\item
  與瑣事之間的掙扎

  這幾天為了研究如何寫出最好的日誌並且盡量保存,耗了我不少精神,結果今天H大學早上九點半的課就忘記去上了,而且差點忘了健身教練課要帶的體育服,幸好向來非常豪爽的課程承辦員還是讓我簽到了,真是感謝他。為了一些自己的執著而影響了工作的排程,或許就一般觀點來講算得不償失吧。

  但是對我來講這點損失或許是必要的,畢竟我不甘心自己被埋在這些瑣事裡面,從此像大部分的同事一樣消失,變成社會齒輪的一部分,我還想要在世界上留下一點痕跡,就算不能寫出傳世的音樂,至少也要讓這本日誌變成這個時代的註腳,讓後人能夠從這個日誌知悉在二十一世紀初(公元2000-2030年),西太平洋邊緣的台灣島(東經120度至122度,北緯22度至25度
  120°E-122°E 22°N-25°N)上,中產階級的細部生活樣貌。

  對,如果我寫不出好音樂,因此不能為後人留下音樂,至少也要留下關於我自己,以及我眼中的這個島的紀錄。我不想要後世提到二十一世紀初的台灣,就只有一些聲量很大的撈仔(註一)寫下的歷史敘事、還有幾種刻板印象的小吃。那個太悲哀了。
\item
  辦公室的神經病

  那個很明顯有妄想症狀的同事,最近狀況又惡化了,一整天我都看到他什麼事都不做,只一直在莫名憤怒地重複講著那些他的政治論述。內容不外是高雄、閩南人、知識分子、共產黨、美國的排列組合。讓我來示範一次:

  「你有看過高雄醫學大學的某醫師嗎,他老是到處審查學生是不是擁有正統閩南意識的,正港的台灣人,甚至用這一套來批鬥別人,其實那都是共產黨那一套,很明顯他已經是匪諜了。這些高喊台獨的閩南人因為個性卑劣,所以去過日本都被共產黨吸收變成匪諜,尤其是那些閩南知識份子更是無惡不作,浪費我們的稅金,但是我們的政府都不作為,其實美國人早就看在眼裡了,我們的政府應該趕快把這些人都公開槍殺。」

  好,這樣就夠了,連我自己都覺得這個排列組合的遊戲很厭煩了,他居然可以講一整天。要有妄想至少也要像以西結書(註二)一樣有創意,拜託。
\item
  討厭的教練課

  健身本身已經很酸痛了,被教練訓練又是另外一層的痛苦,而且我的教練是一個腦子很空洞又聒噪的中年男子,談吐就跟館長(註三)差不多水準,我們兩個真的非常不對盤。

  但是他有一個優點就是耐心很足,而且不挑客戶,只要願意付學費他就收,並且真的認真教,廣結善緣到了同事會稱呼他「南無XX菩薩救苦救難」的程度。男友前陣子覺得我的身材有在進步,一部分是這位囉嗦的阿伯教出來的。所以我還是繼續忍耐吧。
\item
  有關底線

  為什麼標題的三個部分要用底線連接?因為我年紀大,所以我曾經在1997年用過使用MS-DOS的電腦,當時的檔名是不可以有空白的,連字號(hyphen)則會在某些指令上面出問題,所以檔名底線就變成了一個習慣。
\end{enumerate}

註一:一般指「政治撈仔」,來自通行於福建與台灣地區的閩南語,指為在政治上操作各種議題,以取得自己利益的人士,多帶有諷刺的意味。

註二:猶太經典塔納赫(Tanakh)分為三個部分,指引書(又譯妥拉、Torah)、先知書(Nevi'im)以及其他著作(又譯詩歌智慧書,Ketuvim)。其中以西結(Ezekiel)書為先知書中的主要文本之一,書中假托許多幻象以隱喻猶大國(Judah)在公元前593年覆亡的命運。如果去脈絡解釋書中幻象的話,相當接近精神分裂症的症狀,故亦為後世精神醫學研究的文本之一。目前認為此書確實成書於前571年,非後世偽作。

註三、陳之漢(公元1979年3月12日-),原名陳思翰,臺灣宜蘭縣人,生於臺北縣蘆洲鄉(今新北市蘆洲區),知名部落客、直播主、YouTuber、網路名人、企業家、慈善家、運動員、健身教練、散打武術家,由於身為連鎖健身館創辦人,人稱「館長」,其談吐被視為具有草莽基層風格。

\hypertarget{ux84bcux767dux7403ux65e5ux8a8c000720191005}{%
\section{蒼白球日誌0007(20191005)}\label{ux84bcux767dux7403ux65e5ux8a8c000720191005}}

\hypertarget{ux65e5ux671f-date-6}{%
\subsubsection{日期 Date}\label{ux65e5ux671f-date-6}}

\begin{itemize}
\tightlist
\item
  世界協調時間 2019年(中華民國108年)10月5日 / Unix紀元 18174 日 / 星期六
  / 蒼白球紀元第7日
\item
  October 5, 2019 (UTC) / 18174 days since Unix Epoch / Saturday /
  Globus Pallidum day 7
\item
  特殊註記:B院評鑑倒數13天
\end{itemize}

\hypertarget{ux5e74ux9f61-age-6}{%
\subsubsection{年齡 Age}\label{ux5e74ux9f61-age-6}}

\begin{itemize}
\tightlist
\item
  33 歲 5 個月 13 天
\item
  33 years 5 months 13 days old
\end{itemize}

\hypertarget{ux672cux6587-content-6}{%
\subsubsection{本文 Content}\label{ux672cux6587-content-6}}

\begin{enumerate}
\def\labelenumi{\arabic{enumi}.}
\item
  公元2019年10月5日,香港《禁蒙面法》施行

  既然號稱是台灣島生活紀錄,照理來講不應該過度紀錄政治情勢,尤其不應該記錄非台灣島境內的政治新聞,然而《禁蒙面法》施行事件重要到,可以成為一個歷史的節點,未來勢必會被用來錨定時間軸,因此還是記錄如下:

  公元2019年10月4日,香港特首(\textbf{註一})林鄭月娥引用《緊急情況規例條例》的授權,頒布《禁止蒙面規例》(簡稱禁蒙面法),於公元2019年10月5日施行。其實許多承平時期的國家都對陳抗的蒙面有限制,並非只有進入緊急狀態的國家有,因此最大的爭議點並非在禁蒙面法本身,而在於引用《緊急情況規例條例》這個措施,相當於直接宣布香港進入形同戒嚴,可以限制人身自由的例外狀況,並使1997年中華人民共和國承諾香港的一國兩制形同廢棄。

  在此一規例宣布後,10月5日香港各地均出現陳抗升級的現象,甚至有社運者起草《香港臨時政府宣言》(\textbf{備份於附錄}),升高獨立意識。在未來的幾個月內,香港,以及整個東亞的命運將會怎麼改變呢?
\item
  空汙迷幻仙境

  照理來講不應該在這種空氣污染嚴重,整個台中市(位於中華民國轄下台灣島內的第三大都市)霧濛濛的天氣下去望高寮(\textbf{註二})看風景,但是難得男友跟我都有空,機會難得,於是還是帶他去了。即使空氣導致視野不是很好,然而在望高寮觀景平台上,台中市的繁華美景依然一覽無遺,且因蒙塵而反而多了一種朦朧感,就像某網友在景點評論裡曾提到的「空汙迷幻仙境」,別有一番風味。

  然而在這美景前,男友注意到一個很突兀的物體,就是台中市政府在高台前擺的兩個雙心型粉紅色的造景。那兩個東西應該是因為望高寮是情侶約會熱點,所以市府公務員就想像了一個「心心相印」(\textbf{註三})的意象擺非常庸俗的物品來讓人拍照,卻破壞了整個風景。悲哀的是,搞不好三百年以後挖掘台中市遺址,最先出土的不是什麼名雕塑,而反而是在山上的這個東西。到時一定會有很機車的考古學家把它解釋成祭壇,設計成兩個心臟形狀是因為這種祭典一次會挖兩個人(一大一小)的心臟,旁邊的塑膠水管則是引流血液使用。太可怕了,文字千萬不可以遺失啊,否則台中市以後會被寫成到處都是邪教祭壇的地方。
\item
  烤肉這檔事

  太陽下山以後,我跟男友就跑去望高寮觀景平台旁邊的烤肉店吃烤肉看夜景,原本覺得那個店設備簡陋,就只是在看起來像垃圾桶的水泥柱上面用烤肉架燒炭烤東西,收一個人458元新台幣(\textbf{註四})實在太貴,不過後來看到店家送來滿滿一桌的肉,就覺得也還算有誠意了。

  不過有誠意歸有誠意,這個店基本上作風還是非常的粗糙,沒有多餘的服務,一口氣送來大量肉品,裡面甚至有一整根香腸這種很容易烤不熟的東西,只好跟店家要剪刀自己把他剪開。在我剪的時候男友提出了一個疑問,就是豬肉不熟難道不能吃嗎?我嚴正地跟他說不行,長了豬肉絛蟲(\textbf{註五})可不是好玩的事情,我甚至懷疑豬肉絛蟲就是導致古代中東地區禁吃豬肉的遠因。

  解決了香腸以後其他肉品就沒有什麼難事了,我們在宜人的木炭煙霧裡面把一整桌的肉吃光光,配著旁邊大家族烤肉的嬉鬧聲,開始有種非常非常放鬆豪放的感覺。如果是在一般的燒烤店,中途一定會有店員來干擾吧,只有在這種店員不管事的地方才能這麼放鬆,也好。
\end{enumerate}

\hypertarget{ux6ce8ux91cb-comment}{%
\subsubsection{注釋 Comment}\label{ux6ce8ux91cb-comment}}

註一:香港特別行政區行政長官(又稱「行政長官」、「特首」),是中華人民共和國香港特別行政區的行政首長。此職位設立於1997年7月1日,大致代替英國殖民統治時期的「香港總督」一職,行使香港特別行政區政府首腦的權責。由選舉委員會選出,國務院總理親自簽令任命,每屆任期五年,最多可連任一次。根據《香港基本法》,行政長官對中央人民政府(國務院)負責,也只有中央人民政府(國務院)有權罷免行政長官。

註二:望高寮又稱「王田坎」,是台中市南方的大肚台地最高的一處山野,為一座大斷崖。站在眺望平台可以遠看台中港區、彰化大肚溪以南平地、以及臺中市區,夜晚更可以看到三地萬家燈火的夜景。由於地形的優勢,使得望高寮成為一觀日及賞星光的最佳去處。

註三:許多地區的文化認為心臟主管人類的神智,因此「心心相印」用來比喻彼此心靈互通,情意相合,並非兩人心臟被挖出放在一起的意思。

註四:新臺幣,是中華民國現行的法定貨幣,於1949年6月15日起發行流通,當時定位為限定臺灣使用的區域貨幣,目前則在整個中華民國有效統治區域(臺澎金馬)均可流通使用。原始發行機構為臺灣銀行,2000年起由中華民國中央銀行收回負責。基本單位為圓(簡作元)。在這個時間的台中市物價,一個正餐餐盒的價格約為60-80元新台幣,960毫升(milliliter)容量的市售麥茶價格為20元新台幣。

註五:豬肉絛蟲,即豬帶絛蟲(學名:Taenia solium;pork
tapeworm),也稱有鉤絛蟲或鏈狀帶絛蟲,體長2-3米,寬7-8毫米,共有800-900個節片,後端成熟節片長約10毫米。人是唯一的最終宿主,寄生於小腸內,引起豬肉絛蟲病,中間宿主主要是豬,人同時也可以成為異常中間宿主(abnormal
intermediate
host)。一般情況下,豬在吃了有絛蟲卵的食物後,絛蟲卵在豬體內胃酸和酶的作用下開始孵化,生出六鉤蚴(hexacanth)。六鉤蚴接著進入豬的循環系統,最終在肌肉組織、腦組織等地方停留,發育成囊尾蚴,並被豬組織包圍形成一個囊(cyst)。當人吃了未煮熟的有絛蟲囊尾蚴的豬肉(通稱「米心肉」)後感染,囊尾蚴在人體小腸發育成豬肉絛蟲成蟲,頭節掛在小腸壁以吸取營養和並分生出節片以繁殖後代。特殊情況下,人偶然進食了豬肉絛蟲卵,蟲卵發育出的六鉤蚴可能通過人體循環系統迴圈到腦組織或者眼球裡發育成包囊,危及人體健康甚至生命。(牛肉絛蟲不會有這條異常生活史,因此較不危險,故牛肉可以生食)

\hypertarget{ux9644ux9304-appendix}{%
\subsubsection{附錄 Appendix}\label{ux9644ux9304-appendix}}

《香港臨時政府宣言》(原文載於https://lihkg.com/thread/1619523/page/1)

在人類文明的進程中,推翻破舊之物以建更美好之物乃必然之事,此之為人類進化之本。如舊有之政府不為人民所立、不為人民所治、不為人民所享,則人民建立屬民之政府,亦為必然之理。今香港特別行政區政府已然不為香港人民所立、所治、所享,故今日我等宣告成立香港臨時政府。

「人人生而平等,上天賦予全人類某此不可剝奪之權,包含生命權,自由權、尊嚴權及追求幸福之權。」此乃我等一直所認同及不可踐踏之真理及原則。人民為了不剝奪及不被剝奪,始建立法律及政府以保障自身及他人之權。政府一切之權力,乃源於人所賦予之權。政府如破壞以上原則,則人民有絕對之權力推翻及建立。

如今香港特別行政區政府受中華人民共和國及中國共產黨所控制,對香港人民的訴求視若無睹,不斷剝削人民之權利,不立利民之法,反而進一步剝奪人民之自由。香港特別行政區政府今日繞過立法會制定《反蒙面法》,企圖繼續壓制人民集會之權,不理會香港絕大多數的人民之意願。我等認為香港特別行政區政府已然失去人民之認可及授權,故今日我等宣布香港特別行政區政府失去其合法性,行政長官,各司、局長立即失去其職位所賦予之權。

香港臨時政府宣布: 1. 原有香港特別行政區政府各部門歸入香港臨時政府管理。
2.
原有香港特別行政區政府行政長官、各司長、各局長及副局長,各部門首長及副首長馬上離任及懸空職位,直至香港臨時政府委任。
3.
各部門馬上停止自二O一八年起所有由香港特別行政區政府所頒布之新政策,各級人員保留其職位,並維持各部門必要之日常運作,直至另行通知。
4.
香港臨時政府任期為五年或直至以全民提名及普選政府之體制及政府首長(以較短日期為准)。香港臨時政府必需在成立起計一年內籌備選舉,並於三年內完成選舉。
5.
香港臨時政府首長及各部門之臨時委任官員,在其任期結束後,終身不可擔任香港政府及公營機構之任何官職及受薪之決策人員。
6. 《香港法律》各條文暫予以保留,直至香港臨時政府頒布新法律。 7.
解散香港立法會,並於三個月內進行選舉臨時立法會,一年內重新選舉立法會。臨時立法會議員席位為七十席。其中香港島、九龍西選區十二席,九龍東十席,新界西、新界東選區十八席。

\hypertarget{ux84bcux767dux7403ux65e5ux8a8c000820191006}{%
\section{蒼白球日誌0008(20191006)}\label{ux84bcux767dux7403ux65e5ux8a8c000820191006}}

\hypertarget{ux65e5ux671f-date-7}{%
\subsubsection{日期 Date}\label{ux65e5ux671f-date-7}}

\begin{itemize}
\tightlist
\item
  世界協調時間 2019年(中華民國108年)10月6日 / Unix紀元 18175 日 / 星期日
  / 蒼白球紀元第8日
\item
  October 6, 2019 (UTC) / 18175 days since Unix Epoch / Sunday / Globus
  Pallidum day 8
\item
  特殊註記:B院評鑑倒數12天
\end{itemize}

\hypertarget{ux5e74ux9f61-age-7}{%
\subsubsection{年齡 Age}\label{ux5e74ux9f61-age-7}}

\begin{itemize}
\tightlist
\item
  33 歲 5 個月 14 天
\item
  33 years 5 months 14 days old
\end{itemize}

\hypertarget{ux672cux6587-content-7}{%
\subsubsection{本文 Content}\label{ux672cux6587-content-7}}

今年母后大人的非洲鼓{[}1{]}教學班繼續維持三個班,所以照慣例有好多教學計畫,還有評鑑文件要寫。趁著最近兩周我的日常工作比較少,母子兩人已經完成了某些文件準備工作,而今天就是要整體完成所有計畫的日子。我們從早上八點半一路工作到晚上十點半,結結實實的工作了十四個小時。而且因為母后大人思考速度非常快,非常密集,又不會發呆,所以我幾乎沒有喘息的空間,這真的比上班累多了。

讀到這裡大概會有人有疑問,就是既然母后思考速度那麼快,那為何需要我幫忙寫?首要原因當然是因為他並非跟著電腦{[}2{]}一起長大的世代,所以在文件的繕打、文書處理軟體{[}3{]}跟資料的爬梳上面遠不如我,在這個情況下我的輔助可以大大提升整體寫作速度。次要原因則是有關唬爛{[}4{]}這件事。雖然我的文字表達能力不佳,但在母后需要一些唬爛出來的,半真半假的知識以矇騙文件閱讀者的時候,我可以快速地提供之,讓他藉由那個點子完成整篇騙人的文章。

例如說,有人建議說這個申請文件要跟記憶力有關,那麼我就設法去硬是引用幾篇神經科學{[}5{]}的論文,串在一起穿鑿附會說因為這些科學證據,非洲鼓就可以改善記憶力了,並且途中有經由這個大腦路徑又有經過大腦路徑,有沒有很有證據力呢?程度不夠的人就會這樣被唬爛過去了,而比較有經過科學訓練的人就會知道這根本等於沒有證據,畢竟把沒有直接關係的幾個證據串在一起,導出來的結論,就等於沒有用嘛。

但是我們有很多程度不行的學者在審文件,所以我跟母后可以開開心心的在這個禮拜天用唬爛把文件全部完成,讓我明天好好去上班,並且還可以帶著微笑預見:那些學者會在很滿意地看完他之後覺得自己好棒棒,看了好有學問的文章而且還看懂了!一百分!

晚安,這個唬爛的世界。

\hypertarget{ux6ce8ux91cb-comment-1}{%
\subsubsection{注釋 Comment}\label{ux6ce8ux91cb-comment-1}}

{[}1{]}
這邊指的是Djembe,又稱金貝鼓,由皮革鼓皮覆蓋高腳杯狀的木箱所組成,起源於公元1200年左右的西非,但是在他前七百年的歷史僅見於非洲,公元1950年左右才從非洲傳到歐洲及全世界。因方便攜帶而風行於21世紀的台灣。

{[}2{]}
電腦(亦稱電子計算機)是利用數位電子技術,根據一系列指令指示並且自動執行任意算術或邏輯操作序列的裝置。通用電腦因有能遵循被稱為「程式」的一般操作集的能力而使得它們能夠執行極其廣泛的任務。電腦可以用作各種工業和消費裝置的控制系統,例如說你現在看到的這個文字,就是使用電腦輸入,而非手寫的。筆者的字很醜,有電腦很棒。

{[}3{]} 文書處理器(英語:word
processors),文書軟體,用作桌面出版(例如文書格式處理)的電腦軟體,它與文字編輯器不同之處是在於它並非用作編寫程式,而是用來給普通文書工作者使用,以編寫書信材料、公私檔案與出版作品等等以普通文字所組成的檔案。

{[}4{]}
即畫虎𡳞(白話字:Ōe-hó͘-lān),異用字話虎膦,為閩南俗語,流行於福建南部及台灣一帶,有說話誇大不實、吹牛之意。在台灣有時會在前方加上「畫山畫水」四字而成為「畫山畫水畫虎𡳞」,意義相同。「𡳞」原指雄性生殖器。源於佛典中的「吹大法螺」(大吹法螺),「法螺」日語讀成「ホラ(hora)」,而台語轉音變成「虎爛」(ho.-lan7)。

{[}5{]}
神經科學(英語:neuroscience),又稱神經生物學,是專門研究神經系統(包含大腦、脊隨等神經)的的一門科學。對行為及學習的研究都是神經科學的分支。目前神經科學最關心的議題是人類的腦部,尤其重視的是研究思維及知覺。畢竟,人類可以用人腦想到發明電腦的方式,然後我這個人類可以藉由人腦使用人類發明的電腦對別的人類唬爛,是一件神經系統的奇蹟,不是嗎?

\hypertarget{ux9644ux9304-appendix-1}{%
\subsubsection{附錄 Appendix}\label{ux9644ux9304-appendix-1}}

無。

\hypertarget{ux84bcux767dux7403ux65e5ux8a8c000920191007}{%
\section{蒼白球日誌0009(20191007)}\label{ux84bcux767dux7403ux65e5ux8a8c000920191007}}

\hypertarget{ux65e5ux671f-date-8}{%
\subsubsection{日期 Date}\label{ux65e5ux671f-date-8}}

\begin{itemize}
\tightlist
\item
  世界協調時間 2019年(中華民國108年)10月7日 / Unix紀元 18176 日 / 星期一
  / 蒼白球紀元第9日
\item
  October 7, 2019 (UTC) / 18176 days since Unix Epoch / Monday / Globus
  Pallidum day 9
\item
  特殊註記:B院評鑑倒數11天
\end{itemize}

\hypertarget{ux5e74ux9f61-age-8}{%
\subsubsection{年齡 Age}\label{ux5e74ux9f61-age-8}}

\begin{itemize}
\tightlist
\item
  33 歲 5 個月 15 天
\item
  33 years 5 months 15 days old
\end{itemize}

\hypertarget{ux672cux6587-content-8}{%
\subsubsection{本文 Content}\label{ux672cux6587-content-8}}

\begin{enumerate}
\def\labelenumi{\arabic{enumi}.}
\item
  疲倦

  在昨天寫了一整天文件,耗盡所有精神之後,今天一整個無法集中精神,中午十二點以前都在電腦{[}1{]}前面發呆,無法發出任何一行診斷。在下午三點上完醫學系的PBL課程{[}2{]}回來以後,耗竭的狀況沒有改善,反而更嚴重了,因此下午五點就決定提早收工,去健身房隨便拉個幾下就回家。

  回家並不代表休息,我還有研究的事情要處理,尤其是那三個研究生每天都在殷殷期盼我圈片子{[}3{]}給他們訓練演算法{[}4{]},我也確實需要那個演算法去應付院內的研究壓力,但是因為身心真的很累很累很累,怎麼辦?隨便做個一點點上傳吧,這樣就可以說我很有誠意有做了喔,這樣混過去。就可以早點睡了。懶人就是這樣\ldots\ldots{}
\item
  我的PBL

  學校出給我們的PBL題目通常都跟真實的病患有關,並牽涉具體的標準醫療知識,因此我很不喜歡傳統PBL那種放任學員天馬行空地去思考、辯證的方式,畢竟當內容指向一群已知的知識的時候,為什麼要在上面發揮創意?並且,大部分的醫學生恐怕都跟我小時候一樣,並無法很熱心地去參與這種奇怪也沒有意義的討論。那麼硬是叫他們在空白空間發揮,不是他們也痛苦我也痛苦嗎?

  所以我的方式是直接根據題目對學員一一進行詰問,讓他們在一連串的問題跟忙亂的資料查詢中,以自己的力量跟腦袋攝取到整個診斷程序的全貌。這並非一個合格的小組討論,更不是一個合格的PBL,但是我注重的是讓學員領略到醫學的脈絡,而不是演出一個形式上看起來很美麗的討論。或許我是錯的吧?反正我就這樣做了。
\end{enumerate}

\hypertarget{ux6ce8ux91cb-comment-2}{%
\subsubsection{注釋 Comment}\label{ux6ce8ux91cb-comment-2}}

{[}1{]}
電腦(亦稱電子計算機)是利用數位電子技術,根據一系列指令指示並且自動執行任意算術或邏輯操作序列的裝置。通用電腦因有能遵循被稱為「程式」的一般操作集的能力而使得它們能夠執行極其廣泛的任務。電腦可以用作各種工業和消費裝置的控制系統,例如說你現在看到的這個文字,就是使用電腦輸入,而非手寫的。筆者的字很醜,有電腦很棒。

{[}2{]} PBL(Problem-Based
Learning)問題導向學習法指的是透過問題或情境誘發學生思考,並建立學習目標,學生進行自我導向式研讀,增進新知或修正舊有的知識內容。PBL不只能夠解決問題,在處理問題的同時,也是我們精進知識的最佳時機。\textbf{其實就是一種小組討論課}。

{[}3{]}
在電腦圖像上面圈選區域,並且標示此處細胞的種類,為訓練機器學習演算法{[}4{]}時相當重要的一步

{[}4{]}
這邊指的是機器學習的演算法。最常見的機器學習指的是一種可以藉由對分析外界資料修正自身演算法,以使演算法更加適合資料的程式,他的行為很類似人類的嘗試-錯誤學習。目前許多研究者正在研究如何利用機器學習的方式,使程式能模仿病理醫師看片診斷的方式。我就是其中之一,但是我們團隊的程度相當落後。我們會設法趕上的。

\hypertarget{ux9644ux9304-appendix-2}{%
\subsubsection{附錄 Appendix}\label{ux9644ux9304-appendix-2}}

無。

\hypertarget{ux84bcux767dux7403ux65e5ux8a8c001020191008}{%
\section{蒼白球日誌0010(20191008)}\label{ux84bcux767dux7403ux65e5ux8a8c001020191008}}

\hypertarget{ux65e5ux671f-date-9}{%
\subsubsection{日期 Date}\label{ux65e5ux671f-date-9}}

\begin{itemize}
\tightlist
\item
  世界協調時間 2019年(中華民國108年)10月8日 / Unix紀元 18177 日 / 星期一
  / 蒼白球紀元第10日
\item
  October 8, 2019 (UTC) / 18177 days since Unix Epoch / Monday / Globus
  Pallidum day 10
\item
  特殊註記:B院評鑑倒數10天
\end{itemize}

\hypertarget{ux5e74ux9f61-age-9}{%
\subsubsection{年齡 Age}\label{ux5e74ux9f61-age-9}}

\begin{itemize}
\tightlist
\item
  33 歲 5 個月 16 天
\item
  33 years 5 months 16 days old
\end{itemize}

\hypertarget{ux672cux6587-content-9}{%
\subsubsection{本文 Content}\label{ux672cux6587-content-9}}

\begin{enumerate}
\def\labelenumi{\arabic{enumi}.}
\item
  有關註釋

  與男友討論了一下日誌重複註釋的問題。

  在前面九篇日誌中,為了文件的保存性,我完全不引用外部文獻,而是把讀者可能需要的資訊都包含在文件的註釋內部,希望即使大部分頁面亡佚,人類依然能夠從殘卷底下的註釋中得到能夠解釋本文的資訊。

  這個方式有一個令人苦惱的缺點,那就是會出現連續兩篇日誌有完全相同註釋的情形,看起來很突兀。今天早上跟男友討論這件事情的時候,我原本是固執己見的,認為即使會出現這個問題,還是什麼都要附才行,然而經過一天的反芻以後我態度開始軟化了,畢竟即使窮盡所有話語,也無法完整描述我們這個時代的文明,不如放輕鬆一點。

  因此這篇開始,註釋起來太麻煩或太重複的名詞會引用別篇日誌,或是其他參考資料。
\item
  有關文字表達能力

  因為這幾天工作比較沒那麼重,稍微思考了一下該怎麼改進日誌寫作的流暢度,於是想到了有一種東西叫做寫作書。

  原本想要直接買一些寫作書來看,但是想到未來收入可能會減少,非必要的花費還是不要亂花,所以就上了C醫院圖書館{[}1{]}網站{[}2{]}查詢,找到了張大春{[}3{]}的「文章自在」,正想要借來看的時候,就發現網路上有盜版。

  於是我就開始看這本盜版書了。如果真的有幫助的話,或許有一天我會買一本來報答張大春吧。
\end{enumerate}

\hypertarget{ux6ce8ux91cb-comment-3}{%
\subsubsection{注釋 Comment}\label{ux6ce8ux91cb-comment-3}}

{[}1{]}
圖書館是個收藏資訊、原始資料、資料庫並提供相關服務的地方,可以由公共團體、政府機構或者私人組織開辦。圖書館在人類文明發展及歷史存留具顯著作用,是人類智慧的寶庫。

{[}2{]}
網站是指在網際網路上,根據一定的規則,使用HTML等工具製作的用於展示特定內容的相關網頁的集合。簡單地說,網站是一種通訊工具,就像布告欄一樣,人們可以通過網站來發布自己想要公開的資訊,或網站來提供相關的網路服務。人們可以通過網頁瀏覽器來存取網站,獲取自己需要的資訊或者享受網路服務。網際網路(英語:Internet)是指21世紀之初網路與網路之間所串連成的龐大網路。這些網路以一些標準的網路協定相連,連接全世界幾十億個裝置,形成邏輯上的單一巨大國際網絡。

{[}3{]}
張大春(1957年6月14日-),筆名大頭春,當代華文作家。21世紀後張大春以時事批評者的角度活躍,口吻嘻笑怒罵、用字尖酸刻薄、看法相當具有爭議與批判性。其相關爭議有:「狗屁文創產業」事件、新聞局秘書邀約過程不嚴謹「我沒空」事件、「說我作品不深的人絕對是笨蛋」事件、「本土意識窮極無聊」事件、「笨蛋的作品自稱魔幻寫得很爛」事件、「張大春肖想當蘋果日報總編輯」事件、「人渣治國」事件、《致中閔書》事件(針對2018年國立臺灣大學校長遴選事件)等。

\hypertarget{ux9644ux9304-appendix-3}{%
\subsubsection{附錄 Appendix}\label{ux9644ux9304-appendix-3}}

無。

\hypertarget{ux84bcux767dux7403ux65e5ux8a8c001120191009}{%
\section{蒼白球日誌0011(20191009)}\label{ux84bcux767dux7403ux65e5ux8a8c001120191009}}

\hypertarget{ux65e5ux671f-date-10}{%
\subsubsection{日期 Date}\label{ux65e5ux671f-date-10}}

\begin{itemize}
\tightlist
\item
  世界協調時間 2019年(中華民國108年)10月9日 / Unix紀元 18178 日 / 星期三
  / 蒼白球紀元第11日
\item
  October 9, 2019 (UTC) / 18177 days since Unix Epoch / Wednesday /
  Globus Pallidum day 11
\item
  特殊註記:B院評鑑倒數9天
\end{itemize}

\hypertarget{ux5e74ux9f61-age-10}{%
\subsubsection{年齡 Age}\label{ux5e74ux9f61-age-10}}

\begin{itemize}
\tightlist
\item
  33 歲 5 個月 17 天
\item
  33 years 5 months 17 days old
\end{itemize}

\hypertarget{ux672cux6587-content-10}{%
\subsubsection{本文 Content}\label{ux672cux6587-content-10}}

\begin{enumerate}
\def\labelenumi{\arabic{enumi}.}
\item
  養成隨時筆記的習慣

  寫前十天的日誌時,都是回到家以後才開始整理,寫的時候必須要在腦裡面反芻當日的事件,相當沒有效率。今天早上看著辦公室的設備,突然意識到我手上的文書處理功能,24小時都非常充沛,不僅電腦隨手可得,即使在少數沒有電腦的狀況,也有其他可以連上網路的攜帶式設備{[}1{]}{[}2{]}。因此,應該不要浪費這個文書能量,隨時把握零碎時間把值得紀錄的事件寫下,這樣每天晚上整理的時候才能更快而且更全面。

  於是我今天就這麼做了,似乎成效還不錯,記憶保存得比前幾天好。未來應該會越來越好吧。
\item
  機器學習研究這回事

  手上正在進行的機器學習研究案,與神經膠細胞瘤{[}3{]}的機器判讀有關,而我自己跟參與研究的工程師都發現,要能夠訓練出好的演算法,必須收集更多正常腦細胞的圖像{[}4{]}。因此,這一兩天我花了一點時間處理跟切片數位化{[}5{]}廠商聯絡,向醫院動支研究經費,並且跟玻片室管理員聯絡玻片調閱{[}6{]}事宜。幸好我跟廠商以及玻片室管理員有私交,因此一些不太明白的細節他們都很有耐心地帶著我處理了。

  初學醫學研究的路程,除了醫學上的考驗以外,更多的是行政磨練。而行政磨練裡的一大部分是建立人際網路的磨練。這真的比醫療作業更需要溝通。
\item
  紅磚牆

  早上上班經過一個正在被拆除的三層房屋,發現所有牆面都是紅磚,當時的想法是:「這應該是超過四十年歷史的房屋吧,才會是紅磚結構,而非RC鋼骨{[}7{]}。」然而在查詢網路之後,才知道即使是二十一世紀的新成屋,一樣會有一部分或是全部的牆面是磚牆,只有特別高的建物會採全部RC鋼骨。

  這個世界的知識真是複雜啊。
\end{enumerate}

\hypertarget{ux6ce8ux91cb-comment-4}{%
\subsubsection{注釋 Comment}\label{ux6ce8ux91cb-comment-4}}

{[}1{]}
主要指行動電話。行動電話,又稱「手提式電話機」或「手提電話」,簡稱「手機」,是可以在較大範圍內使用的可攜式電話,與固定電話(座機)相對。1990年代中期以前價格昂貴,只有極少部分經濟實力較佳的人才買得起,而且體積龐大,因此又有大哥大的俗稱。1990年代後期大幅降價,如今已成為現代人日常不可或缺的電子用品之一。目前的行動電話可以執行相當多電腦的功能。

{[}2{]} 有關電腦與文書處理,可參考蒼白球日誌0008。

{[}3{]}
Glioma,為神經系統的支持細胞增生導致的腫瘤。大部分的原發惡性腦瘤都屬於此類。

{[}4{]} 有關機器學習與圖像分析,可參考蒼白球日誌0009。

{[}5{]} 將源自實體人體組織之玻片做全切片掃描,藉以得到圖像的過程。

{[}6{]}
病理科絕大部分的業務來自外科病理,也就是針對活人身上取下的人體組織做診斷的工作。在這個工作中,最重要的資源是將人體組織蠟化以長期保存的「蠟塊」,以及由蠟塊所切下、染色,並承載於玻璃片上,藉以在顯微鏡下判讀的「玻片」。因此,醫學中心的蠟塊室與玻片室,不僅是人體知識的大圖書館,更是有醫病爭議時的法律證據來源,為醫學中心重地中的重地,需嚴密管理。玻片室與蠟塊室的管理員通常必須有非常非常嚴密的人格特質。

{[}7{]} 指鋼筋混凝土(英語:Reinforced
Concrete,Ferroconcrete,RC),是指通過在混凝土中加入鋼筋、鋼筋網、鋼板或纖維而構成的一種組合材料,兩者共同工作從而改善混凝土抗拉強度不足的力學性質,為混凝土加固的一種最常見形式。

\hypertarget{ux9644ux9304-appendix-4}{%
\subsubsection{附錄 Appendix}\label{ux9644ux9304-appendix-4}}

無。

\hypertarget{ux84bcux767dux7403ux65e5ux8a8c001220191010}{%
\section{蒼白球日誌0012(20191010)}\label{ux84bcux767dux7403ux65e5ux8a8c001220191010}}

\hypertarget{ux65e5ux671f-date-11}{%
\subsubsection{日期 Date}\label{ux65e5ux671f-date-11}}

\begin{itemize}
\tightlist
\item
  世界協調時間 2019年(中華民國108年)10月10日 / Unix紀元 18179 日 /
  星期四 / 蒼白球紀元第12日
\item
  October 10, 2019 (UTC) / 18178 days since Unix Epoch / Thursday /
  Globus Pallidum day 12
\item
  特殊註記:B院評鑑倒數8天
\end{itemize}

\hypertarget{ux5e74ux9f61-age-11}{%
\subsubsection{年齡 Age}\label{ux5e74ux9f61-age-11}}

\begin{itemize}
\tightlist
\item
  33 歲 5 個月 18 天
\item
  33 years 5 months 18 days old
\end{itemize}

\hypertarget{ux672cux6587-content-11}{%
\subsubsection{本文 Content}\label{ux672cux6587-content-11}}

\begin{enumerate}
\def\labelenumi{\arabic{enumi}.}
\item
  病人與心意這件事

  大姑姑因為泌尿道感染合併敗血性休克{[}1{]}住進了南投{[}2{]}某院的加護病房。今早去工作前,父皇大人強烈的希望我去看他,然而當我想到還有新的玻片{[}3{]}要看,而且健身不能中斷,就婉拒今天去看他,表示工作太多了無法抽身,而且我去看對於敗血性休克並沒有幫助。

  父皇不禁埋怨了一下,說沒有幫助歸沒有幫助,至少表達一個心意嘛。此時心裡出現了一句話:「我的字典沒有心意兩個字!」我無法在實際作為上施力的事情,基本上就不表示關心了,所以無法理解為何他們那麼重視心意。

  但是這句話終究還是說不出口,畢竟我也隱約感覺到那個很傷人,所以我就沉默了。
\item
  尋找作文法書籍意外發現馬橋詞典

  張大春的「文章自在」寫得有點太過藝術,我認為那不是我一個初學文字表達的人需要看的第一本書,因此就重新搜尋了一遍網路,發現劉承慧{[}4{]}的「寫作文法入門九講」寫得非常清晰易懂,所以就決定先從這個講義開始讀。

  在提到中文的分句時,劉承慧摘錄了一個韓少功{[}5{]}小說「馬橋詞典」的片段,說明逗號在長句中的作用。由於那個片段寫得太鮮活,我忍不住好奇那是什麼樣一本書,於是又中途暫停了「寫作文法入門」的閱讀,開始閱讀「馬橋詞典」。不讀還好,一讀就陷了下去。

  因為這本書太好看了。

  這本書他媽的太好看了。

  實在很難形容這本書多麼的神,所以就在這邊停筆了。總之,如果中華人民共和國倒下,文獻開始毀壞的時候,至少要先救這本書。
\item
  今天是中華民國的生日

  但是所有檯面上的人物都不在慶祝的心情上,而且居然連批鬥誰沒有尊重國旗國歌的心情都沒有。真是罕見,可能最近東亞真的太多事吧。
\item
  雜記:台中市物價與其他

  \begin{itemize}
  \tightlist
  \item
    健身房附近的全家便利商店常常會把接近保存時間的熟食打七折販售,稱為「友善食光七折」,因此85元{[}7{]}的雞肉丸栗子南瓜燉飯我用60元買到,加上29元的鹼性離子水,以及10元的茶葉蛋,共99元。
  \item
    雞肉丸栗子南瓜燉飯裡面的雞肉丸,味道跟麥克雞塊{[}8{]}一樣。
  \item
    健身後喝一杯連鎖店「台灣第一味」的甘蔗青茶,50元。
  \item
    母后託我買的非洲鼓教材前天就寄到辦公室了,但是我連續三天忘了拿回家,所以還擺在辦公室。記憶力真的開始退化了嗎?
  \end{itemize}
\end{enumerate}

\hypertarget{ux6ce8ux91cb-comment-5}{%
\subsubsection{注釋 Comment}\label{ux6ce8ux91cb-comment-5}}

{[}1{]} 敗血性休克(septic
shock),又稱感染性休克,是指罹患敗血症(sepsis)的病人出現休克症狀,出現低血壓,從而產生器官灌注(perfusion)異常,導致如乳酸中毒(lactic
acidosis)、少尿(oliguria)、或急性精神異常的情況。

{[}2{]} 南投縣(臺灣話:Lâm-tâu Kuān;國姓鄉四縣客語:Namˇ Teuˇ
Ien),是中華民國臺灣省的縣和省會,位處臺灣中部,坐落在本島正中央,是臺灣唯一的內陸縣、唯三的內陸行政區,縣治及最大城市為南投市。

{[}3{]} 有關病理切片,可見蒼白球日記0011

{[}4{]} 時任台灣清華大學中國文學系教授。

{[}5{]}
韓少功(1953年1月1日-),湖南長沙人,中國作家。作品主要以文革{[}6{]}時的有關經歷為素材。

{[}6{]}
文化大革命簡稱文革,是一場於1966年---1976年在中華人民共和國境內所發生的政治運動。文革是由時任中國共產黨中央委員會主席的毛澤東主導,在中國大陸發動的階級鬥爭。普遍認為官方鼓勵的批鬥、抄家及告密等行為,毀壞清代以來的文化傳承,整體經濟受嚴重影響,受害人數以千萬計。

{[}7{]} 新台幣計價。有關新台幣可見蒼白球日誌0007。

{[}8{]} 麥克鷄塊(英語:Chicken
McNuggets)是國際速食連鎖餐飲企業麥當勞於1983年在各家國際加盟餐廳所推出的菜色,是將雞肉經過加工並攪拌成肉漿後再加入麵包粉、之後加以油炸而成的雞塊產品。

\hypertarget{ux84bcux767dux7403ux65e5ux8a8c001320191011}{%
\section{蒼白球日誌0013(20191011)}\label{ux84bcux767dux7403ux65e5ux8a8c001320191011}}

\hypertarget{ux65e5ux671f-date-12}{%
\subsubsection{日期 Date}\label{ux65e5ux671f-date-12}}

\begin{itemize}
\tightlist
\item
  世界協調時間 2019年(中華民國108年)10月11日 / Unix紀元 18180 日 /
  星期五 / 蒼白球紀元第13日
\item
  October 11, 2019 (UTC) / 18180 days since Unix Epoch / Friday / Globus
  Pallidum day 13
\item
  特殊註記:B院評鑑倒數7天
\end{itemize}

\hypertarget{ux5e74ux9f61-age-12}{%
\subsubsection{年齡 Age}\label{ux5e74ux9f61-age-12}}

\begin{itemize}
\tightlist
\item
  33 歲 5 個月 19 天
\item
  33 years 5 months 19 days old
\end{itemize}

\hypertarget{ux672cux6587-content-12}{%
\subsubsection{本文 Content}\label{ux672cux6587-content-12}}

\begin{enumerate}
\def\labelenumi{\arabic{enumi}.}
\item
  研究生來訪

  這一年斷斷續續地與H大生物資訊研究生對話,希望能夠藉由這些溝通一起做出一點機器判讀的演算法,然而因為從來沒有在顯微鏡前面一起看片,所以總覺得有種隔閡感,無法讓他們理解病理人的判讀思維,自然也做不出什麼來。{[}1{]}

  今天因為某種偶然的機緣,終於讓其中一個研究生來到病理科一起看片,我像教住院醫師一樣在顯微鏡前跟他解釋:病理切片是看結構診斷的,因此圖像判讀一定要根據一個區域的綜合特徵,不可以單純以單一細胞的特徵判斷。雖然大意只有這三句,然而因為細節非常多,所以我們從下午三點整整談到了四點五十分才結束。希望在這幾乎兩小時的課程後,他能夠理解病理在做什麼,並且指導機器做出正確的圖像辨識。讓我們創造H大與C院的明天吧。
\item
  機車解封印一天

  父皇與母后去高雄{[}2{]}準備一場演出,會兩天不在家,因此停在家裡好幾個月的機車{[}3{]}終於有機會騎出去透透氣了。得到這個機會後,第一件事情不是發動機車,而是上班時間先用自然人憑證{[}4{]}註冊監理服務網{[}5{]}帳號後,新增一個通訊地址。這樣做的用意是萬一這兩天產生罰單,就不會寄到家裡,而是寄到公司,避免可能的爭吵。

  解決了通訊地址的問題,我回到家裡馬上把機車騎到附近的機車行換了機油與齒輪油{[}7{]},並且幫輪胎打點氣。明天如果有空的話,應該就可以出去兜風了。
\item
  有關一中街{[}6{]}

  一中街。以難看制服的高中為中心,往外輻射出兩個、三個世代的青春,15、25、35、45歲的人在這裡凍結成一個巨大無比的中學時期。

  一個充滿隱喻的地方。本身就是詩。或許真的該寫一首詩來描述一中街與我,但不是今天。今天累了,不要去挑戰這麼可怕的任務。
\item
  保養品

  幾個月前本來以為既然C院的合作廠商是理膚寶水,我就應該可以使用理膚寶水的全套保養品而不會出問題。

  結果我大錯特錯,付出了爆痘的代價,後來經過母后、玻片倉庫管理員與科秘書的種種指導,終於找到了我皮膚可以承受的組合,居然包含了好幾種廠牌,保養真是大學問。(這個地方不用備註,文明滅亡的時候,保養品廠商是不需要的記憶。)

  \begin{itemize}
  \tightlist
  \item
    露得清 - 深層淨化高效即淨卸妝水 (因為便宜)
  \item
    理膚寶水 - 青春潔膚凝膠 (目前覺得理膚唯一沒有問題的東西,所以留著)
  \item
    蘭蔻 - 超水妍舒緩保濕凝露 (母后推薦)
  \item
    雅頓 - 8小時防曬膏 (母后推薦)
  \item
    Tunemaker - 毛孔收縮水 (化妝水,倉庫番推薦)
  \item
    Tunemaker - 維他命E與金縷梅
  \end{itemize}
\item
  雜記 - 物價與其他{[}8{]}

  \begin{itemize}
  \tightlist
  \item
    換機油與齒輪油 400元
  \item
    多喝水檸檬味氣泡水:26元,C院餐廳午餐(紫米飯、豬柳、海帶、高麗菜、花菜):72元(原價80員工九折),一中街「二口鐵板麵」羊肉炒麵:80元,一中街「花茶大師」桑菊薄荷茶:50元,礦泉水:18元
  \end{itemize}
\end{enumerate}

\hypertarget{ux6ce8ux91cb-comment-6}{%
\subsubsection{注釋 Comment}\label{ux6ce8ux91cb-comment-6}}

{[}1{]} 有關病理診斷、機器學習與我的研究,請見蒼白球日誌0009以及0011。

{[}2{]}
高雄市,通稱高雄(臺灣話:Ko-hiông;客家話:Kô-hiùng),時為中華民國的第三大城市,位於台灣島南部。

{[}3{]}
摩托車(來自英語的「Motorcycle」或「Motorbike」),在臺灣另稱為機車、歐兜邁(台灣話,源自日語オートバイ
Auto-Bi),為代步工具之一,是指兩輪或三輪的機動車輛,由摩托化自行車衍化而來,以兩輪為大宗。

{[}4{]}
中華民國政府所核發的一種身分識別用晶片卡,可以用於在網路上辨識身分。

{[}5{]} 監理所為中華民國政府管理車輛的單位,服務網為其網頁。

{[}6{]}
一中商圈(亦稱一中街商圈、台中一中商圈、台中一中街商圈)是位於臺中市北區的大型商圈,範圍以台中第一名校,也就是台中第一高級中學為核心,原本是一般學生商圈,近年已發展成大型的綜合商圈。

{[}7{]} 機車使用的潤滑油。

{[}8{]} 新台幣計價。有關新台幣可見蒼白球日誌0007。

\hypertarget{ux9644ux9304-appendix-5}{%
\subsubsection{附錄 Appendix}\label{ux9644ux9304-appendix-5}}

無。

\hypertarget{ux84bcux767dux7403ux65e5ux8a8c001420191012}{%
\section{蒼白球日誌0014(20191012)}\label{ux84bcux767dux7403ux65e5ux8a8c001420191012}}

\hypertarget{ux65e5ux671f-date-13}{%
\subsubsection{日期 Date}\label{ux65e5ux671f-date-13}}

\begin{itemize}
\tightlist
\item
  世界協調時間 2019年(中華民國108年)10月12日 / Unix紀元 18181 日 /
  星期六 / 蒼白球紀元第14日
\item
  October 12, 2019 (UTC) / 18181 days since Unix Epoch / Saturday /
  Globus Pallidum day 14
\item
  特殊註記:B院評鑑倒數6天
\end{itemize}

\hypertarget{ux5e74ux9f61-age-13}{%
\subsubsection{年齡 Age}\label{ux5e74ux9f61-age-13}}

\begin{itemize}
\tightlist
\item
  33 歲 5 個月 19 天
\item
  33 years 5 months 19 days old
\end{itemize}

\hypertarget{ux672cux6587-content-13}{%
\subsubsection{本文 Content}\label{ux672cux6587-content-13}}

\begin{enumerate}
\def\labelenumi{\arabic{enumi}.}
\item
  香港

  香港今年(2019)大規模抗爭以來,不明原因死亡者已達118人,據說死者多為年輕抗爭者,且男女狀況不同:男性多為墜樓,女性多為落水。這樣的死亡特徵使得社群網路上開始瀰漫著男性抗爭者遭虐打,女性抗爭者遭性侵的傳聞。

  我知道傳聞未必屬實,而且即使傳聞屬實,在準戰爭狀況底下發生一些噁心的事情,在人類來講也很正常,並不是什麼特別需要大驚小怪的狀況。然而,還是忍不住覺得有點淡淡的悲傷,只能提醒自己,在這個大局之下有太多情緒也沒有用,不需要像某些同儕那樣陷入激憤裡面。

  畢竟,大時代的劇本或許已經寫好了。而我們這種大時代的齒輪能做什麼,或許就像下面這句話說的吧:

  『做得了什麼的人此刻不能說話,我是做不了什麼的人,做不了什麼的人能做的就是把這些事見證下來。』-張紹中
\item
  日誌寫作軟體上面的小嘗試

  前面13天的日誌都是在hackmd{[}1{]}上面寫的,然而因為Hackmd沒有辦法切換到舒服的Solarized{[}3{]}配色,大黑大白的很刺眼,因此本來想要全面改用VSCode{[}4{]}寫,然後用Git版本控制當作備份媒介,徹底拋棄Hackmd。

  昨天試了一下後發現完全不行。脫離Hackmd會馬上遇到兩個無法解決的問題,其一是多空間備份,也就是只用Git跟Github同步的話,無法同時備份到Dropbox與Google
  Drive,等於同時少了Hackmd,Dropbox,Google
  Drive三個空間的備份;其二是行動裝置問題。

  什麼叫做行動裝置問題?就是Git事實上只能在桌上型電腦上運作良好,對於手機或平板來說並不是那麼友善,只用Git等於把寫作的空間侷限於辦公室與家中,無法在手機上同步編輯日誌檔案,實在不太方便。

  所以我暫時還是保留hackmd,不過還是會練習用VSCode/Git寫作,順便練版本控制。
\item
  寫程式產生日誌

  每天寫日誌前都去數今天會是Unix紀元第幾天,其實滿煩的,所以一直有醞釀要用程式自動產生每天的日記範本{[}5{]}。

  然後我今天就真的做了。在離開辦公室回家以後。

  而且真的完成了。只是付出的代價是,原本要運動跟讀論文的時間,就這樣被寫程式吃掉了。可以說得不償失嗎?至少每天用程式產生日誌表頭的感覺應該滿爽的啦。至於運動跟研究,再說吧。
\item
  雜記 - 物價與其他 {[}6{]}

  \begin{itemize}
  \tightlist
  \item
    『四海游龍』菜肉大餛飩麵 65
    元,『多喝水』檸檬口味氣泡水兩瓶46元(第二瓶六折),痘痘藥膏兩條400元,剪頭髮250元
  \item
    2009年燒錄,放在塑膠收納箱裡面的光碟目前外觀還安好,但是我所有的光碟機都壞掉了。要買光碟機來測試光碟的保存性嗎?想想看好了。
  \item
    機車解封印本來要出去玩,但想到工作進度,跟今天要值冰凍切片{[}7{]}這件事,就還是先去辦公室了。
  \end{itemize}
\end{enumerate}

\hypertarget{ux6ce8ux91cb-comment-7}{%
\subsubsection{注釋 Comment}\label{ux6ce8ux91cb-comment-7}}

{[}1{]} HackMD 是個跨平台的
Markdown{[}2{]}即時協作筆記,由於是在網頁瀏覽器上操作,並且內容存放於其伺服器,因此可以在電腦、平板或是手機上編輯。有關電腦,手機及網頁,請見蒼白球日誌0008,0010以及0011。

{[}2{]} Markdown是一種輕量級標記式語言,創始人為John
Gruber。它允許人們「使用易讀易寫的純文字格式編寫文件,然後轉換成有效的XHTML文件」,為非常方便寫作與保存的一種檔案格式。

{[}3{]} Ethan
Schoonover所設計的低對比度電腦螢幕顯示配色,試圖模擬『白天坐在一片樹蔭下閱讀』的舒適感。

{[}4{]} Visual Studio Code(簡稱VS
Code)是一個由微軟開發,同時支援Windows 、
Linux和macOS等操作系統且開放原始碼的程式碼編輯器,它支援測試,並內建了Git
版本控制功能。

{[}5{]} 使用python語言撰寫,程式碼可見
https://gist.github.com/ordinarymiddleclass/112afb60d9ae1a0d0d9615e2e30d3eaf
,因太冗長,不宜記載於附錄。

{[}6{]} 新台幣計價。有關新台幣可見蒼白球日誌0007。

{[}7{]}
在低溫下進行人類組織切片,並即時染色判讀的一種技術,可以在手術中作為快速初步診斷的方法,因此病理科每日,包含假日,都會安排值班醫師,外科醫師在假日臨時需要做冰凍切片時,可電聯值班病理醫師進行切片。

\hypertarget{ux9644ux9304-appendix-6}{%
\subsubsection{附錄 Appendix}\label{ux9644ux9304-appendix-6}}

無。

\hypertarget{ux84bcux767dux7403ux65e5ux8a8c001520191013}{%
\section{蒼白球日誌0015(20191013)}\label{ux84bcux767dux7403ux65e5ux8a8c001520191013}}

\hypertarget{ux65e5ux671f-date-14}{%
\subsubsection{日期 Date}\label{ux65e5ux671f-date-14}}

\begin{itemize}
\tightlist
\item
  世界協調時間2019年(中華民國108年)10月13日 / Unix 紀元 18182 日 /
  星期日 / 蒼白球紀元第15日
\item
  October 13, 2019 (UTC) / 18182 days since Unix Epoch / Sunday / Globus
  Pallidum day 15
\item
  特殊註記:
\end{itemize}

\hypertarget{ux5e74ux9f61-age-14}{%
\subsubsection{年齡 Age}\label{ux5e74ux9f61-age-14}}

\begin{itemize}
\tightlist
\item
  33 years 5 months 20 days
\item
  33 歲 5 個月 20 天
\end{itemize}

\hypertarget{ux672cux6587-content-14}{%
\subsubsection{本文 Content}\label{ux672cux6587-content-14}}

\begin{enumerate}
\def\labelenumi{\arabic{enumi}.}
\item
  在手機上面設定Git {[}1{]}

  因為行動裝置的安全管理比一般電腦上面的作業系統要嚴格,所以原本就預期在手機上面使用Git會是一件非常困難的事情。只是我沒想到會困難到花了三個小時在上面。

  這三小時到底都在做什麼?其實有兩個半小時都在設定SSH{[}2{]}。在windows電腦上面的Git客戶軟體很貼心,如果你的Github登入需要兩步驟驗證{[}3{]}時,會開一個小程式進行兩步驟驗證,然而沒有一種手機的Git客戶端有這種功能,只好乖乖地設定SSH通道登入。總共試了三種Git客戶端跟3種SSH產生器,最後發現只有MGit可以成功地同時進行Git的功能以及SSH驗證。

  在今天的努力後,蒼白球日誌在手機上也有一份了。以後要在手機上打日誌,不用再開黑黑的Hackmd了。可喜可賀?
\item
  母后的投影片{[}4{]}

  11月6號母后要去報告,所以今天除了設定Git以外的時間,全部都貢獻給母后的投影片了。之所以會在投影片上面花這麼多時間,原因是我非常不會設計畫面,因此在我初步做好一個段落後,母后要花很多時間一邊生氣,一邊指導我怎麼設計畫面才會好看。沒辦法,我就沒有視覺美感啊\ldots\ldots{}
\item
  像是應酬的聚餐

  被拉著去跟母后還有父皇的兩個朋友兼工作夥伴聚餐,然而身為一個賴在家裡的媽寶,我聽著他們在談的房地產話題其實有點尷尬,再加上對方等同於長輩,那個世代差異讓我跟他們更是話不投機。

  於是我就找機會早早離席了,途中當然有被父皇嫌沒禮貌,但是我不管了。這種像是應酬的聚餐實在太難熬了。
\end{enumerate}

\hypertarget{ux6ce8ux91cb-comment-8}{%
\subsubsection{注釋 Comment}\label{ux6ce8ux91cb-comment-8}}

{[}1{]}
Git為分散式版本控制系統,一開始的用途是用來管理Linux作業系統核心,不過因為非常好用,所以後來就擴散到整個軟體(即電腦程式)界。Git可以把檔案的狀態作為更新歷史記錄保存起來。因此可以把編輯過的檔案復原到以前的狀態,也可以顯示編輯過內容的差異。有關電腦軟體,有關電腦,手機及網頁,請見蒼白球日誌0008,0010以及0011。

{[}2{]} Secure
Shell(安全外殼協定,簡稱SSH)是一種加密的網路傳輸協定,可在不安全的網路中為網路服務提供安全的傳輸環境。

{[}3{]}
一種管制使用者登入的方式。在兩步驟驗證(簡稱2FA)的管制下,單憑使用者帳密無法登入,還需要第二項「驗證因素」:僅限個人知道的資訊、個人持有的事物
(如透過簡訊發送的認證碼、應用程式或軟體保護鎖)、或是駭客無法取得的個人特徵
(如指紋)。

{[}4{]}
即簡報。簡報,就一個題目向聽眾陳述內容、傳達訊息或觀點的過程。這一過程中所展示的圖片、文字說明也可叫簡報。因常常使用投影機放大畫面,因此又稱為投影片。

\hypertarget{ux9644ux9304-appendix-7}{%
\subsubsection{附錄 Appendix}\label{ux9644ux9304-appendix-7}}

圖片0015-1 http://imgbox.com/NYrhCLlK 台中市柳川,2019年10月12日

\hypertarget{ux84bcux767dux7403ux65e5ux8a8c001620191014}{%
\section{蒼白球日誌0016(20191014)}\label{ux84bcux767dux7403ux65e5ux8a8c001620191014}}

\hypertarget{ux65e5ux671f-date-15}{%
\subsubsection{日期 Date}\label{ux65e5ux671f-date-15}}

\begin{itemize}
\tightlist
\item
  世界協調時間2019年(中華民國108年)10月14日 / Unix 紀元 18183 日 /
  星期一 / 蒼白球紀元第16日
\item
  October 14, 2019 (UTC) / 18183 days since Unix Epoch / Monday / Globus
  Pallidum day 16
\item
  特殊註記:B院評鑑倒數3天
\end{itemize}

\hypertarget{ux5e74ux9f61-age-15}{%
\subsubsection{年齡 Age}\label{ux5e74ux9f61-age-15}}

\begin{itemize}
\tightlist
\item
  33 years 5 months 21 days
\item
  33 歲 5 個月 21 天
\end{itemize}

\hypertarget{ux672cux6587-content-15}{%
\subsubsection{本文 Content}\label{ux672cux6587-content-15}}

\begin{enumerate}
\def\labelenumi{\arabic{enumi}.}
\item
  資淺病理醫師{[}1{]}的白天

  如果外行人來想像資淺病理醫師的工作的話,一定會摹繪出以下情境吧:充滿氣勢且悠閒地走進辦公室,喝杯咖啡,然後開始用充滿專業的架勢坐在顯微鏡前,開始高深莫測的診斷工作,並且一臉慎重地打電話與臨床醫師討論病人的病情。

  實際的情形並沒有那麼美好,實情是行醫真的太難了,難到我已經在病理科六年多,工作上卻還每天都滿頭問號。因此,每天到了辦公室第一件事情,就是把自己沒有自信確診的玻片整理起來,然後四處去敲別的主治醫師的門,問過一輪以後才回到自己的辦公室開始診斷。一方面是真的解決自己的疑惑,一方面也是抓人分擔法律責任。

  只有我的診斷實力這麼糟糕,這麼喜歡拉人當墊背嗎?其實並不然,許多資淺的醫師也都跟我一樣,白天是在四處與人討論中渡過的,就連資深醫師也不時會有個案要跟大家討論才敢診斷。行醫真的太難了,是一輩子的修練。
\item
  寫得到十一月底的話真的要印書嗎?

  我前幾天原本有一個想法,就是能夠持續寫到十一月底的話,就把這個日誌印成蒼白球日誌I(2019秋),之後每季都印一本。為了實現這個想法,今天花了很多時間搜尋資料,結果是不禁有點懷疑自己做不做得到。

  首先,雖然確實有某些印刷廠願意接小量印刷的案件,然而價格與品質均參差不齊,令人擔憂。另外,如果不願意花大錢讓出版社接手排版的話,自己排版也將會是個大工程(今天光是查詢如何將markdown{[}4{]}檔案排版成漂亮的版面,就花了一小時),內頁排版起來痛苦,封面因為有書背的問題,則更痛苦。

  因此,如果要印成看起來還算合格的書的話,這種一邊妥善排版一邊跟印刷廠周旋的工作會非常吃力。或許到時我會選擇去影印店隨便做醜醜的裝訂吧,畢竟紙張就是紙張,就算是影印店印的保存性也是一樣超過三十年啊。
\item
  雜記:物價與其他{[}5{]}

  \begin{itemize}
  \tightlist
  \item
    「河南水餃」鮮肉餛飩麵65元、燙青菜30元(淋滿滿肉燥的地瓜葉),「吳家紅茶冰」麥茶20元
  \item
    明天再來考慮要不要補寫今天去教PBL小組討論課的經過。今天懶得寫了。
  \end{itemize}
\end{enumerate}

\hypertarget{ux6ce8ux91cb-comment-9}{%
\subsubsection{注釋 Comment}\label{ux6ce8ux91cb-comment-9}}

{[}1{]}
藉由以眼睛觀察實體人體組織進行診斷的醫師,一般不進行看診與理學檢查工作,僅觀察檢體。由於微觀證據在病理工作上的重要性,因此顯微鏡{[}2{]}與承載薄片組織的玻璃片(簡稱玻片)在病理醫師的工作中佔有非常重要的地位。

{[}2{]} 一種光學儀器,可以放大影像以觀察人眼不可見的微小物體。

{[}3{]} 有關網路請參見蒼白球日誌0010。

{[}4{]} 有關Markdown請參見蒼白球日誌0014。

{[}5{]} 新台幣計價。有關新台幣請參見蒼白球日誌0007。

\hypertarget{ux9644ux9304-appendix-8}{%
\subsubsection{附錄 Appendix}\label{ux9644ux9304-appendix-8}}

無。

\hypertarget{ux84bcux767dux7403ux65e5ux8a8c001720191015}{%
\section{蒼白球日誌0017(20191015)}\label{ux84bcux767dux7403ux65e5ux8a8c001720191015}}

\hypertarget{ux65e5ux671f-date-16}{%
\subsubsection{日期 Date}\label{ux65e5ux671f-date-16}}

\begin{itemize}
\tightlist
\item
  世界協調時間2019年(中華民國108年)10月15日 / Unix 紀元 18184 日 /
  星期二 / 蒼白球紀元第17日
\item
  October 15, 2019 (UTC) / 18184 days since Unix Epoch / Tuesday /
  Globus Pallidum day 17
\item
  特殊註記:B院評鑑倒數2天
\end{itemize}

\hypertarget{ux5e74ux9f61-age-16}{%
\subsubsection{年齡 Age}\label{ux5e74ux9f61-age-16}}

\begin{itemize}
\tightlist
\item
  33 years 5 months 22 days
\item
  33 歲 5 個月 22 天
\end{itemize}

\hypertarget{ux672cux6587-content-16}{%
\subsubsection{本文 Content}\label{ux672cux6587-content-16}}

\begin{enumerate}
\def\labelenumi{\arabic{enumi}.}
\item
  業餘樂師的宿命

  十一月對C院來說是個特別的月份,院慶跟醫師節兩個大節日都在這個月,使得行政單位在十月必須緊鑼密鼓地籌備各種活動。我很不幸地是在院內小有名氣的業餘編曲者,因此照慣例地,某大頭目又親自殺來病理科詢問表演事宜了。理論上,與醫療本業無關的多餘工作根本不該接,但是終究是吃老闆的飯,老闆要看的活動不支持的話,以後大概不好混,所以院慶跟醫師節的活動都硬著頭皮接了,真的是人在屋簷下不得不低頭。

  兩個活動中,院慶的表演活動是多人演出,屬於相對低負擔的活動,因為我可以把重心放在編曲的幕後工作,把表演的部分賴給別的表演者,但醫師節就沒那麼容易了。醫師節是我的單人表演。單人表演非常可怕,畢竟過去曾經能駕馭的一兩種管弦樂器,在多年的荒廢後都已經只剩下發出噪音的能力,這樣上台等於直接出糗,怎麼辦?

  那就拿非洲鼓吧!在跟母后研發非洲鼓教材的過程中,我多少也學了一點,而這種樂器怎麼打都不會走音,練習時間也短,剛好很適合混過這種臨時需要上場的表演。想辦法用最簡單的方式把這種業餘樂師的宿命應付過去,難道未來會變成我的人生日常嗎?
\item
  PBL小組討論課{[}1{]}

  上兩次上課被我用不斷詰問的方式調教以後,學生看起來都非常緊張,手上拿著筆電隨時處於戰備狀態。然而即使是這樣,他們的資料查詢速度還是慢得讓我有點錯愕,而且氣氛弄得很僵。

  我只好做兩個攤牌的措施:第一個是實際在電腦上查資料給他們看,讓他們體會查資料是真的可以這麼快的,我絕對沒有故意刁難人;第二個是直接坦白說我不會按照課堂上的表現打成績。

  「因為我討厭去評量人,所以基本上所有人的分數都會是差不多的中高水準,絕對不會因為你們的表現而有差異,課堂上的這些詰問都只是要讓你們體驗一下臨床醫師的思考而已。」差不多講了這樣的話。

  覺得自己真的很鄉愿。
\item
  Git心得{[}2{]}

  使用版本控制系統幾天後,基本的同步功能開始上手了,於是開始嘗試各種進階玩法。越玩真的是越有趣,有趣到不知道該用什麼文字描述。如果有興趣的人應該要自己抓軟體來玩玩看,不要聽我講,聽沒有用。
\item
  雜記:物價與其他{[}3{]}

  \begin{itemize}
  \tightlist
  \item
    C院餐廳午餐72元(排骨、豆芽、南瓜、高麗菜),「清心福全」普洱25元,「韓石館」石鍋牛雜鍋145元(超好吃!),「台茶一號」甘蔗青茶50元
  \item
    17天以來,註釋的交互參照已經混亂到我自己也跟不上了,接下來真的要開始整理一個索引了。
  \end{itemize}
\end{enumerate}

\hypertarget{ux6ce8ux91cb-comment-10}{%
\subsubsection{注釋 Comment}\label{ux6ce8ux91cb-comment-10}}

{[}1{]} 見蒼白球日誌0007。

{[}2{]} 有關git見蒼白球日誌0015。

{[}2{]} 新台幣計價。有關新台幣可見蒼白球日誌0007。

\hypertarget{ux9644ux9304-appendix-9}{%
\subsubsection{附錄 Appendix}\label{ux9644ux9304-appendix-9}}

無。

\hypertarget{ux84bcux767dux7403ux65e5ux8a8c001820191016}{%
\section{蒼白球日誌0018(20191016)}\label{ux84bcux767dux7403ux65e5ux8a8c001820191016}}

\hypertarget{ux65e5ux671f-date-17}{%
\subsection{日期 Date}\label{ux65e5ux671f-date-17}}

\begin{itemize}
\tightlist
\item
  世界協調時間2019年(中華民國108年)10月16日 / Unix 紀元 18185 日 /
  星期三 / 蒼白球紀元第18日
\item
  October 16, 2019 (UTC) / 18185 days since Unix Epoch / Wednesday /
  Globus Pallidum day 18
\item
  特殊註記:
\end{itemize}

\hypertarget{ux5e74ux9f61-age-17}{%
\subsection{年齡 Age}\label{ux5e74ux9f61-age-17}}

\begin{itemize}
\tightlist
\item
  33 years 5 months 23 days
\item
  33 歲 5 個月 23 天
\end{itemize}

\hypertarget{ux672cux6587-content-17}{%
\subsection{本文 Content}\label{ux672cux6587-content-17}}

\begin{enumerate}
\def\labelenumi{\arabic{enumi}.}
\item
  寫程式產生未來日誌

  今天偷偷用上班時間寫程式,終於把產生每日日誌的程式碼完成八成,產生了未來一百天的日誌檔案,從此不用每天數日子了,叫出檔案寫就是了。同時,藉由這個過程我對於Git版本控制、Python以及Markdown的了解都多了一點。
\item
  醫師節表演

  跟母后借鼓要去表演,他知道我一個人要去打非洲鼓,連忙說這樣的表演以醫師節來說太單薄了,應該要加一個鋼琴,而且覺得我的曲目「EXEC\_WITH\_METHOD\_METAFALICA/.」沒氣質。我跟他盧了一個多小時,才說服他不要干涉我的曲目,畢竟這是我五年來第一次用自己喜歡的曲目上台,說什麼也要堅持自己的夢想嘛。

  不過我沒有辦法阻止他搬鋼琴去醫師節彈。或許這樣也好,比較熱鬧一點啦。
\item
  雜記:物價與其他{[}1{]}

  \begin{itemize}
  \tightlist
  \item
    早早下班去準備明天評鑑要穿的西裝。
  \item
    醫院午餐72元,「美蓁小吃店」乾意麵45元,空心菜30元,粉腸30元,「蜜兔」青草茶35元
  \end{itemize}
\end{enumerate}

\hypertarget{ux6ce8ux91cb-comment-11}{%
\subsection{注釋 Comment}\label{ux6ce8ux91cb-comment-11}}

{[}1{]} 新台幣計價。有關新台幣請參見蒼白球日誌0007。

\hypertarget{ux9644ux9304-appendix-10}{%
\subsection{附錄 Appendix}\label{ux9644ux9304-appendix-10}}

\hypertarget{ux84bcux767dux7403ux65e5ux8a8c001920191017}{%
\section{蒼白球日誌0019(20191017)}\label{ux84bcux767dux7403ux65e5ux8a8c001920191017}}

\hypertarget{ux65e5ux671f-date-18}{%
\subsection{日期 Date}\label{ux65e5ux671f-date-18}}

\begin{itemize}
\tightlist
\item
  世界協調時間2019年(中華民國108年)10月17日 / Unix 紀元 18186 日 /
  星期四 / 蒼白球紀元第19日
\item
  October 17, 2019 (UTC) / 18186 days since Unix Epoch / Thursday /
  Globus Pallidum day 19
\end{itemize}

\hypertarget{ux5e74ux9f61-age-18}{%
\subsection{年齡 Age}\label{ux5e74ux9f61-age-18}}

\begin{itemize}
\tightlist
\item
  33 years 5 months 24 days
\item
  33 歲 5 個月 24 天
\end{itemize}

\hypertarget{ux672cux6587-content-18}{%
\subsection{本文 Content}\label{ux672cux6587-content-18}}

\begin{enumerate}
\def\labelenumi{\arabic{enumi}.}
\item
  B院評鑑之如何當一個花瓶

  由於B院本身不處理病理檢體{[}1{]},而是統一外送C院作業,因此B院的病理檢體主要由檢驗科{[}2{]}監控,評鑑所需的資料自然也都由檢驗科負責,理論上我是不用隨時待命的。

  但是,W院長可能是為了顯示『我們還是有掛名的病理科醫師喔』這種派頭,依然叫我去參加所有的評鑑會議,就算很明顯地我的功用只有偶爾幫檢驗科的腔。於是就造成了一個很幽默的場景,就是我穿上一身父皇大人演出用的正式西裝,戴上點綴滿了C院旗幟的領帶,以及剛燙到筆挺還繡了姓名的新醫師袍,功用卻只是一個花瓶。

  或許也很正常吧,世界上有很多事情莫非只是一個形式而已。
\item
  Git{[}3{]}新手犯的錯:推送{[}4{]}錯地方

  昨天在辦公室寫的一些程式碼,因為已經按了推送,就以為自己已經把所有血汗結晶都送到伺服器上了,結果今天在B院打開git,發現根本沒有上去!

  仔細看了一下說明書,才突然發現是因為自己某些概念不清楚,忘了檢查遠端主機名字是否正確,導致按下推送的時候沒有推到遠端,反而推送到本機的另外一個地方上了。原來「Git不是一個很好入門的工具」這句話是真的,有很多概念跟操作方法,都需要多加磨練才能學得會。
\item
  柏拉圖,亞里斯多德,與聖經

  今天在幫母后的朋友整理文獻的時候,發現在APA{[}5{]}等論文撰寫格式中,有一個非常明確的規定,就是「古典文件不必列入參考文獻中,文中僅說明引用章節」。何謂古典文件?以西方文明來說,就是柏拉圖的「對話錄」、亞里斯多德的「物理學」等希臘經典,以及「聖經」。

  我認為這個規定代表歐美學界對於知識涵養的基本需求,也就是說,預設讀者在讀任何論文前,必須要對這幾本書有基本認識。如果讀者看到論文中引用「對話錄」或是「聖經」時,居然不知道這是什麼書,代表他連歐洲文明的基石都不清楚,應該要先回去讀讀書再來做學問,而不是要求作者重新說明歐洲文明的基石。

  不敢說這種知識門檻一定是對的,但是我相信這個門檻可以排除很多北七。
\item
  雜記:物價與其他{[}6{]}

  \begin{itemize}
  \tightlist
  \item
    「茶裏王」日式無糖綠茶20元(B院販賣機),B院餐廳自助餐50元(「你是員工喔?算50就好啦」),「Tea's
    原味」麥茶20元,坐計程車從B院到B鎮130元,從B鎮坐回T市客運230元
  \item
    雖然我對稻殼過敏,但是從遠處看稻田還是不錯看的。尤其是從高速公路上看。
  \end{itemize}
\end{enumerate}

\hypertarget{ux6ce8ux91cb-comment-12}{%
\subsection{注釋 Comment}\label{ux6ce8ux91cb-comment-12}}

{[}1{]} 有關病理切片,可見蒼白球日記0011

{[}2{]}
又稱臨床病理,對體液等人體成分進行分析,以獲得診斷的學門。此學門雖然聽起來與我們在做的解剖病理關係密切,但其實並沒有重疊,因為在二十世紀西方醫學體系分工確立後,什麼樣的檢體屬於解剖病理,什麼樣的檢體屬於臨床病理,已經有了非常明確的人為定義。惟限於篇幅,不便在此詳述這兩個學門的分野。

{[}3{]} 有關git版本控制系統見蒼白球日誌0015。

{[}4{]} git版本控制系統中,將本機上的變更套用到遠端主機上的功能。

{[}5{]} APA格式指的就是美國心理學會(American Psychological
Association)出版的《美國心理協會刊物準則》,目前已出版至第六版(ISBN
9781433805615),總頁數272頁,而此協會是目前在美國具有權威性的心理學學者組織。

{[}6{]} 新台幣計價。有關新台幣請參見蒼白球日誌0007。

\hypertarget{ux9644ux9304-appendix-11}{%
\subsection{附錄 Appendix}\label{ux9644ux9304-appendix-11}}

無。

\hypertarget{ux84bcux767dux7403ux65e5ux8a8c002020191018}{%
\section{蒼白球日誌0020(20191018)}\label{ux84bcux767dux7403ux65e5ux8a8c002020191018}}

\hypertarget{ux65e5ux671f-date-19}{%
\subsection{日期 Date}\label{ux65e5ux671f-date-19}}

\begin{itemize}
\tightlist
\item
  世界協調時間2019年(中華民國108年)10月18日 / Unix 紀元 18187 日 /
  星期五 / 蒼白球紀元第20日
\item
  October 18, 2019 (UTC) / 18187 days since Unix Epoch / Friday / Globus
  Pallidum day 20
\item
  特殊註記:
\end{itemize}

\hypertarget{ux5e74ux9f61-age-19}{%
\subsection{年齡 Age}\label{ux5e74ux9f61-age-19}}

\begin{itemize}
\tightlist
\item
  33 years 5 months 25 days
\item
  33 歲 5 個月 25 天
\end{itemize}

\hypertarget{ux672cux6587-content-19}{%
\subsection{本文 Content}\label{ux672cux6587-content-19}}

\begin{enumerate}
\def\labelenumi{\arabic{enumi}.}
\item
  馬橋詞典讀書筆記

  「一個成熟的政權,一個強大的集團,總是擁有自己強大的語言體系,總是伴隨著一系列文牘、會議、禮儀\ldots\ldots 不能取得話份的強權,不過是一些徒有財力或武力的烏合之眾\ldots\ldots 正是體會到了這一點,執政者總是重視文件和會議的。\ldots\ldots 即使是沒有絲毫實際效用的會議,也往往會得到他們本能的歡喜。」
  \textasciitilde{} 韓少功《馬橋詞典 ‧ 話份》\textasciitilde{}

  礙於篇幅與版權,我無法在這邊節錄這頁書的全部內容,但是光是這幾句話就已經切中了政治的要點,讀過整頁整章以後,更是覺得韓少功在幾頁書裡已把這個世界的道理道盡了。是的,在人類的世界裡面,話份,也就是話語權是如此重要,導致人們願意用各式各樣的虛字甚至謊言去填充他。

  中華民國的選舉也好,醫院評鑑也好,某種層面上都是一種話語權的爭奪,那些耗費大量時空的無用文件、耗費大量精神的無用爭吵,看似對做事沒有任何幫助,卻是在話份戰爭中一種必要的東西。

  畢竟,要取得話份戰爭的勝利,取決的並不是誰能最貼近真理與實益,而是取決於誰能取得最多抄本。社群媒體{[}1{]}的點閱率、分享數正是話份戰爭的具象化了,每一個被觸及的對象都是一個抄本,而抄本擴散得越快越大,產生者獲得的話份就越多,實權也越大。

  然後,或許是人類的天性吧,內容糟糕的東西反而抄本多。於是這個世界就變成這樣了。我只能夠期待我能夠留下不糟糕的抄本,並且存活下來一小部分,則此生足矣。
\item
  可怕的編曲工作

  醫事室{[}2{]}11月16日C院院慶表演的樂團名單了,居然高達20人,也就是我得在最短時間內生出三首、中型樂團用的編曲,即使是以手速著稱的我,也忍不住備感吃力。

  於是今天在B院的空餘時間,我就嘗試著開始用那邊不太好的爛電腦做少量的策劃,同時完成聯繫團員,了解他們需求的行政工作。從B院回來以後,更是直奔家門,打開軟體開始奮力打譜,想要在今天晚上完成至少一首,維持我手速超快的名聲。

  不過,才從六點工作到七點多,就覺得好想睡,眼睛好酸,看來今天我是無法完成這工作了。先寫完日誌就去睡吧。我的手速王名聲要生鏽了。
\item
  雜記:物價與其他{[}3{]}

  \begin{itemize}
  \tightlist
  \item
    「Tea's」麥茶18元(穿醫師袍九折),「蜜兔」青草茶35元,「美蓁小吃店」乾餛飩麵60元,空心菜30元(嚴厲地叫老闆不可以加肉燥跟醬油),鴨胗30元(超好吃,加香油真對味)
  \item
    從B院回C院的路上會經過大塊大塊的田,這塊是花生,下一塊是芝麻,下一塊是稻,然後又一塊花生,之類的,其實還滿壯麗的。看著看著,就想到B鎮那家很有名的廟廟口,賣的那些芝麻油啦花生油啦,還有各種芝麻花生相關的零食,感覺快要可以聞到田裡散發出零食的香味了。雖然聽說那些東西多半是採用中華人民共和國產的原料,但是我寧可假裝覺得他們都是取自B鎮旁邊的這個綠色大地,這樣比較浪漫一點。
  \end{itemize}
\end{enumerate}

\hypertarget{ux6ce8ux91cb-comment-13}{%
\subsection{注釋 Comment}\label{ux6ce8ux91cb-comment-13}}

{[}1{]} 社群媒體(social
media)是人們用來創作、分享、交流意見、觀點及經驗的虛擬社區和網絡平台。社群媒體和一般的社會大眾媒體最顯著的不同是,讓用戶享有更多的選擇權利和編輯能力,自行集結成某種閱聽社群。

{[}2{]} 醫院中管理紙本檔案以及某些雜務的單位

{[}3{]} 新台幣計價。有關新台幣請參見蒼白球日誌0007。

\hypertarget{ux9644ux9304-appendix-12}{%
\subsection{附錄 Appendix}\label{ux9644ux9304-appendix-12}}

\hypertarget{ux84bcux767dux7403ux65e5ux8a8c002120191019}{%
\section{蒼白球日誌0021(20191019)}\label{ux84bcux767dux7403ux65e5ux8a8c002120191019}}

\hypertarget{ux65e5ux671f-date-20}{%
\subsection{日期 Date}\label{ux65e5ux671f-date-20}}

\begin{itemize}
\tightlist
\item
  世界協調時間2019年(中華民國108年)10月19日 / Unix 紀元 18188 日 /
  星期六 / 蒼白球紀元第21日
\item
  October 19, 2019 (UTC) / 18188 days since Unix Epoch / Saturday /
  Globus Pallidum day 21
\item
  特殊註記:
\end{itemize}

\hypertarget{ux5e74ux9f61-age-20}{%
\subsection{年齡 Age}\label{ux5e74ux9f61-age-20}}

\begin{itemize}
\tightlist
\item
  33 years 5 months 26 days
\item
  33 歲 5 個月 26 天
\end{itemize}

\hypertarget{ux672cux6587-content-20}{%
\subsection{本文 Content}\label{ux672cux6587-content-20}}

\begin{enumerate}
\def\labelenumi{\arabic{enumi}.}
\item
  香蕉飴

  昨天在對話中跟男友提起這種甜點,用一種「你覺得香蕉飴好不好吃」的語氣跟他說,結果他居然滿頭問號,原來他並不知道有香蕉飴。這令我十分訝異,看起來這種甜點只有在台灣的西南部流行,在其他地方並不普遍,而且像糖蔥{[}1{]}一樣正在消逝中。

  於是我只好從頭開始解說什麼叫做香蕉飴,花了一點力氣。這讓我覺得,香蕉飴的記憶很容易失傳,一定要在日誌裡面寫下來。這樣至少未來的人讀到這段的時候可以去想像這種曾經的甜點。

  香蕉飴,簡而言之就是加了工業溶劑乙酸異戊酯(化學式:C7H14O2)的涼糕,可以由糯米或是地瓜粉製成。至於為什麼不含香蕉卻要叫做「香蕉」飴,不外是因為這種溶劑有水果香味,尤其近似於香蕉,可以用作香蕉味香精的緣故。值得注意的是,如果使用真的水果製作香蕉飴的話,反而就沒有那種道地的味道了,是少數必須用工業原料才能做得好的甜點之一。
\item
  做惡夢

  B院評鑑讓我離開了病理切片{[}2{]}兩天,可能因為對於未完成工作的焦慮,或者是兩天沒看片覺得有點手癢,晚上居然做起了關於病理診斷的惡夢,驚醒以後覺得完全沒有休息到,只好補眠到中午再去上班。
\item
  我要成為編曲王!

  昨天跟醫事室主任討論院慶大團表演事宜{[}3{]}的時候,提到我禮拜天應該可以編出譜來,他大感詫異,同時似乎有點懷疑是否真的能做得到。哼,我抄譜俠、手速人絕對不能變成浪得虛名,明天一定要用實力證明,我做得到。不只做得到,還做得好。到時樂團裡面的大五的弟弟妹妹、老醫師、還有老行政人員一定得跪著讚嘆我,用編曲王的名號榮耀我。等著看吧。

  2013年我在志航基地當醫官,曾發下豪語:「我要成為病理王」,而今我已經在偉大的航道上。從明天起,我要立下另外的誓約:

  我要成為抄譜俠,我要成為手速人,我要成為編曲王!
\item
  雜記:物價與其他{[}4{]}

  \begin{itemize}
  \tightlist
  \item
    「美而堅」紳士鞋2412元,「讚不絕口」藥膳鍋140元(這種食材不新鮮的店一定要點香料味重的鍋底,才能掩蓋那個冰箱味),「迷客夏」麥茶25元,「統一」AB優酪乳44元
  \end{itemize}
\end{enumerate}

\hypertarget{ux6ce8ux91cb-comment-14}{%
\subsection{注釋 Comment}\label{ux6ce8ux91cb-comment-14}}

{[}1{]}
糖蔥薄餅又名白糖蔥,為廣東潮汕的傳統小吃,由廣東傳到台灣、香港、越南及其他東南亞華人聚居地的零食甜點。糖蔥是利用糖加熱融化時的可塑性,進而改變糖的外觀而形成蔥的形狀。

{[}2{]} 有關病理切片,可見蒼白球日記0011。

{[}3{]} 此事請見蒼白球日記0020。

{[}4{]} 新台幣計價。有關新台幣請參見蒼白球日誌0007。

\hypertarget{ux9644ux9304-appendix-13}{%
\subsection{附錄 Appendix}\label{ux9644ux9304-appendix-13}}

\hypertarget{ux84bcux767dux7403ux65e5ux8a8c002220191020}{%
\section{蒼白球日誌0022(20191020)}\label{ux84bcux767dux7403ux65e5ux8a8c002220191020}}

\hypertarget{ux65e5ux671f-date-21}{%
\subsection{日期 Date}\label{ux65e5ux671f-date-21}}

\begin{itemize}
\tightlist
\item
  世界協調時間2019年(中華民國108年)10月20日 / Unix 紀元 18189 日 /
  星期日 / 蒼白球紀元第22日
\item
  October 20, 2019 (UTC) / 18189 days since Unix Epoch / Sunday / Globus
  Pallidum day 22
\item
  特殊註記:
\end{itemize}

\hypertarget{ux5e74ux9f61-age-21}{%
\subsection{年齡 Age}\label{ux5e74ux9f61-age-21}}

\begin{itemize}
\tightlist
\item
  33 years 5 months 27 days
\item
  33 歲 5 個月 27 天
\end{itemize}

\hypertarget{ux672cux6587-content-21}{%
\subsection{本文 Content}\label{ux672cux6587-content-21}}

此刻我正在非常忙碌地為了11月11日與11月16日的演出趕稿中。四首需要的編曲曲目,已經完成了兩首:愛爾加『愛的禮讚』與大約翰史特勞斯『拉黛斯基進行曲』,還剩下鄧雨賢『望春風』與志方晶子『EXEC\_with\_METHOD\_METAFALICA/.』(對,這首歌叫這個名字,因為他本身是一支可以執行的程式,所以用函式的命名)。

看到這邊應該有對音樂有概念的讀者想要吐槽我了。『這兩首都是有大量參考資料的簡易編曲,用這種曲目來跟我談手速?』對我承認這是簡易編曲,但是就算是近似於抄譜的簡易編曲,可以用一個工作天抄完兩份的人物,在中華民國可能也沒幾個了。這個情形下我自栩為手速者,真的覺得我對得起這個名號。

好今天日誌只簡短寫到這邊,我繼續聽望春風的譜了。

\hypertarget{ux6ce8ux91cb-comment-15}{%
\subsection{注釋 Comment}\label{ux6ce8ux91cb-comment-15}}

今天懶得註解作曲家。反正這幾個作曲家如果被人遺忘的話,世界應該也已經滅亡得差不多了\ldots\ldots{}

\hypertarget{ux9644ux9304-appendix-14}{%
\subsection{附錄 Appendix}\label{ux9644ux9304-appendix-14}}

\hypertarget{ux84bcux767dux7403ux65e5ux8a8c002320191021}{%
\section{蒼白球日誌0023(20191021)}\label{ux84bcux767dux7403ux65e5ux8a8c002320191021}}

\hypertarget{ux65e5ux671f-date-22}{%
\subsection{日期 Date}\label{ux65e5ux671f-date-22}}

\begin{itemize}
\tightlist
\item
  世界協調時間2019年(中華民國108年)10月21日 / Unix 紀元 18190 日 /
  星期一 / 蒼白球紀元第23日
\item
  October 21, 2019 (UTC) / 18190 days since Unix Epoch / Monday / Globus
  Pallidum day 23
\end{itemize}

\hypertarget{ux5e74ux9f61-age-22}{%
\subsection{年齡 Age}\label{ux5e74ux9f61-age-22}}

\begin{itemize}
\tightlist
\item
  33 years 5 months 28 days
\item
  33 歲 5 個月 28 天
\end{itemize}

\hypertarget{ux672cux6587-content-22}{%
\subsection{本文 Content}\label{ux672cux6587-content-22}}

\begin{enumerate}
\def\labelenumi{\arabic{enumi}.}
\item
  與母后大人的合奏

  為了能夠完成演奏神曲「EXEC\_with\_METHOD\_METAFALICA/.」{[}1{]}的夢想,我趕稿趕到昨天晚上完全沒睡,今天去B院開會前才勉強補眠一下下,感到超級疲倦。

  本來心想,付出這種心力真的值得嗎?但是在今天晚上拿著提琴與母后的鋼琴合奏此曲後,覺得完全值得了,值回票價。這是我近年來寫過最好聽的鋼琴了,光是一支提琴加鋼琴就非常美。開始期待周五的鋼琴五重奏,一定很棒吧?

  比較遺憾的是我本來想要試著配著原歌曲演奏,但是母后怎麼對都對不上,只好放棄。畢竟彈鋼琴是很花專注力的事情,真的不太能強迫他卡拉OK。算了。
\item
  看田

  從B院回C院的路上,照例能夠看到大片大片的田。前幾天在這些田裏面認出了花生、芝麻、稻子,但是有一種一大叢一大叢葉子大大的作物,當時認不出它是什麼。今天花了一點時間仔細看,才突然想到,啊,應該是地瓜吧?畢竟B鎮著名的廟口有賣地瓜片,種點地瓜來加工也是很合理的!

  話說,一般人對於地瓜的想像應該是根莖可以吃,葉子也可以吃,完全不浪費,但是其實這個印象在近代已經不太適用了。由於能生出好吃葉子的品種就長不出好吃地瓜,反之亦然,所以一般都是地瓜葉田跟地瓜田分開,採收以後把不好吃的部分通通丟掉,只留該品種好吃的部分。

  想想這還真是奢侈。
\item
  雜記:物價與其他{[}2{]}

  「Tea's」麥茶18元,「上海顧家小館」{[}3{]}雪菜肉絲麵100元,「清心福全」冬瓜青茶35元。
\end{enumerate}

\hypertarget{ux6ce8ux91cb-comment-16}{%
\subsection{注釋 Comment}\label{ux6ce8ux91cb-comment-16}}

{[}1{]} 志方晶子作,為電玩遊戲「魔塔大陸 II Ar tonelico
II」的結尾曲,此曲的獨特性在於它的五個段落分別象徵著五種不同的魔法,在這五種魔法合而為一時才能創造世界。

{[}2{]} 新台幣計價。有關新台幣請參見蒼白球日誌0007。

{[}3{]} 1986年創立的老店

\hypertarget{ux9644ux9304-appendix-15}{%
\subsection{附錄 Appendix}\label{ux9644ux9304-appendix-15}}

\hypertarget{ux84bcux767dux7403ux65e5ux8a8c002420191022}{%
\section{蒼白球日誌0024(20191022)}\label{ux84bcux767dux7403ux65e5ux8a8c002420191022}}

\hypertarget{ux65e5ux671f-date-23}{%
\subsection{日期 Date}\label{ux65e5ux671f-date-23}}

\begin{itemize}
\tightlist
\item
  世界協調時間2019年(中華民國108年)10月22日 / Unix 紀元 18191 日 /
  星期二 / 蒼白球紀元第24日
\item
  October 22, 2019 (UTC) / 18191 days since Unix Epoch / Tuesday /
  Globus Pallidum day 24
\end{itemize}

\hypertarget{ux5e74ux9f61-age-23}{%
\subsection{年齡 Age}\label{ux5e74ux9f61-age-23}}

\begin{itemize}
\tightlist
\item
  33 years 5 months 29 days
\item
  33 歲 5 個月 29 天
\end{itemize}

\hypertarget{ux672cux6587-content-23}{%
\subsection{本文 Content}\label{ux672cux6587-content-23}}

\begin{enumerate}
\def\labelenumi{\arabic{enumi}.}
\item
  回歸正常病理人生活

  經歷了評鑑與趕樂譜的瘋狂五日後,今天終於回歸(真正的)工作崗位。由於生疏與睡眠不足的疲累,工作速度如意料中的非常緩慢,往往看沒幾眼片子就開始逛網頁發呆。在工作幾小時以後才勉強回過神來,大概明天就能夠恢復正常狀態了。病理人果然不能長時間離開片子{[}3{]},否則就得復健。
\item
  取消的教練課

  正在重新熟悉工作,而且還有一些樂譜要修改的我,實在沒有餘力可以去健身,於是教練課就又取消了。健身真的是一種很奢侈的事情。嗯。必須要時間上很富有的人才能好好執行,而命運並沒有給我這樣的大量時間。
\item
  有關放炮

  柳川旁邊的仙鴻宮正在廟會,連續好幾天都在晚上八點左右放鞭炮{[}2{]},音響大到連中友百貨都聽得非常清楚。此時我剛好路過摩門教{[}1{]}會,裡面衝出來一個亞裔但很明顯非台灣人的年輕結實帥哥,驚恐地跑出來說「What's
  up there!?」

  我不知道哪裡來的膽子,居然趁機跟他搭話,說柳川邊剛好有一間廟,他們正在做祭典,大聲放鞭炮非常正常。畢竟不是大家都像摩門教那麼安靜嘛。

  「但是已經吵好幾天了耶」(用中文說)。

  這孩子漢語講得不錯嘛。要不是我急著要回去上班,還真想跟他多聊聊廟跟廟對於這個島的意義,享受一下跟超級帥哥講話的感覺,順便讓他對於別的宗教跟文化有點概念,否則摩門教男孩都被教得笨笨的,實在有點可惜。
\item
  雜記:物價與其他{[}4{]}

  「全家便利商店」沙茶牛肉炒麵65元、洋甘菊茶28元,「花茶大師」桑菊薄荷茶50元,C院餐廳午餐(紫米飯,油菜,南瓜,海帶,洋蔥豬肉)72元,「清心福全」普洱茶25元
\end{enumerate}

\hypertarget{ux6ce8ux91cb-comment-17}{%
\subsection{注釋 Comment}\label{ux6ce8ux91cb-comment-17}}

{[}1{]} 耶穌基督後期聖徒教會(The Church of Jesus Christ of Latter-day
Saints),總部位於美國猶他州鹽湖城,分支遍及世界。教會成員相信該教會復興了耶穌最初建立的教會。據該教會統計,該教會擁有70000所教堂、信徒超過1500萬人。該教會成員被稱為「後期聖徒」或「耶穌基督後期聖徒教會成員」。教徒相信耶穌基督是世界的救主、神的兒子,是每個人的救贖主;信賴耶穌基督,遵從祂的福音會帶來永恆的幸福;也相信約瑟·斯密是神所揀選的先知,先見,啟示者。

{[}2{]}
爆竹又稱鞭炮、紙炮仔(客家話)、爆竹(粵語)、炮仔(閩南話),起源至今有2000多年的歷史。鞭炮最開始主要用於驅魔避邪,而在現代,華人在傳統節日、婚禮喜慶、各類慶典、廟會活動等場合幾乎都會燃放鞭炮,特別是在農曆新年期間,鞭炮的使用量超過全年用量的一半。

{[}3{]} 有關病理切片,可見蒼白球日記0011

{[}4{]} 新台幣計價。有關新台幣請參見蒼白球日誌0007。

\hypertarget{ux9644ux9304-appendix-16}{%
\subsection{附錄 Appendix}\label{ux9644ux9304-appendix-16}}

\hypertarget{ux84bcux767dux7403ux65e5ux8a8c002520191023}{%
\section{蒼白球日誌0025(20191023)}\label{ux84bcux767dux7403ux65e5ux8a8c002520191023}}

\hypertarget{ux65e5ux671f-date-24}{%
\subsection{日期 Date}\label{ux65e5ux671f-date-24}}

\begin{itemize}
\tightlist
\item
  世界協調時間2019年(中華民國108年,令和元年)10月23日 / Unix 紀元 18192
  日 / 星期三 / 蒼白球紀元第25日
\item
  October 23, 2019 (UTC) / 18192 days since Unix Epoch / Wednesday /
  Globus Pallidum day 25
\end{itemize}

\hypertarget{ux5e74ux9f61-age-24}{%
\subsection{年齡 Age}\label{ux5e74ux9f61-age-24}}

\begin{itemize}
\tightlist
\item
  33 years 6 months 0 days
\item
  33 歲 6 個月 0 天
\end{itemize}

\hypertarget{ux672cux6587-content-24}{%
\subsection{本文 Content}\label{ux672cux6587-content-24}}

\begin{enumerate}
\def\labelenumi{\arabic{enumi}.}
\item
  日本德仁天皇今日進行即位大典,包含英國王儲等多位虛位元首均共襄盛舉。我認為這個場面,基本上認可了古老的皇室傳統在二十一世紀世界存在的意義,或許日本、英國、瑞典的皇室會存續到二十二世紀也說不定,畢竟人們需要一個精神象徵嘛。
\item
  把片片擱著,早早衝回家趕院慶的樂譜,結果打了一個多小時就倒在床上,真的太累了。可惡,難道我禮拜五會交不出譜嗎,我無法成為編曲王了嗎\ldots\ldots{}
\item
  雜記:物價與其他{[}1{]}
\end{enumerate}

\hypertarget{ux6ce8ux91cb-comment-18}{%
\subsection{注釋 Comment}\label{ux6ce8ux91cb-comment-18}}

{[}1{]} 新台幣計價。有關新台幣請參見蒼白球日誌0007。

\hypertarget{ux9644ux9304-appendix-17}{%
\subsection{附錄 Appendix}\label{ux9644ux9304-appendix-17}}

\hypertarget{ux84bcux767dux7403ux65e5ux8a8c002620191024}{%
\section{蒼白球日誌0026(20191024)}\label{ux84bcux767dux7403ux65e5ux8a8c002620191024}}

\hypertarget{ux65e5ux671f-date-25}{%
\subsection{日期 Date}\label{ux65e5ux671f-date-25}}

\begin{itemize}
\tightlist
\item
  世界協調時間2019年(中華民國108年)10月24日 / Unix 紀元 18193 日 /
  星期四 / 蒼白球紀元第26日
\item
  October 24, 2019 (UTC) / 18193 days since Unix Epoch / Thursday /
  Globus Pallidum day 26
\end{itemize}

\hypertarget{ux5e74ux9f61-age-25}{%
\subsection{年齡 Age}\label{ux5e74ux9f61-age-25}}

\begin{itemize}
\tightlist
\item
  33 years 6 months 1 days
\item
  33 歲 6 個月 1 天
\end{itemize}

\hypertarget{ux672cux6587-content-25}{%
\subsection{本文 Content}\label{ux672cux6587-content-25}}

\begin{enumerate}
\def\labelenumi{\arabic{enumi}.}
\item
  有關時間

  人們常把日曆與時鐘上的數字視為理所當然,彷彿時間是很自然地出現在我們的生活周遭,這樣的態度是非常無知的,象徵著科學教育的失敗。事實上,如果稍微有讀過維基百科{[}1{]}的話,就會發現,時間單位的確立,是數千年天文觀測的結晶;時間的校準,則有賴於二十世紀物理學的進展,一切都是很不容易的。

  要珍惜「時間」這種文明的產物,不僅僅是因為人類花了很多心血才得到它,更是因為「時間」需要大量的資源去維護,實際上非常脆弱。

  其中最脆弱的莫過於「秒」的精準度。目前人類使用的秒是採用放射性元素衰變時放出的輻射所測定,即所謂「原子鐘」,它牽涉到非常精密且使用電能的儀器,一旦這種儀器全數損壞或失去電力,人類將被迫將一日分割86400份以得到一秒的長度,這相對來講會是一個相當不準確的單位。至於「日」與「年」的概念,牽涉到對地球公轉的觀察,相對來講比較沒有那麼依賴精密儀器,但一樣需要專業人員與道具,在文明墜落時一樣會變得越來越不準。

  所以我必須在人類還有精準的日期時,盡量紀錄一切的一切,珍惜時間。嗯,珍惜時間。
\item
  文字編輯器的選擇: Visual Studio Code與Notepad++

  微軟{[}2{]}的文字編輯器與Visual Studio
  Code功能強大,內建許多整合開發環境{[}3{]}的功能,我在家中的電腦{[}4{]}已經使用這套軟體寫作。原本想要在辦公室電腦也全面改用這套軟體,然而實際使用後,發現這個軟體因為佔用許多記憶體資源,在辦公室的老舊電腦上面執行起來有點吃力,因此辦公室電腦只好繼續以Notepad++這套比較輕量的文字編輯器寫作。

  畢竟公司要賺錢,不可能常常換電腦嘛。
\item
  訂便當

  為了10月25日的院慶表演練習,副院長很貼心地準備了餐費,讓我們自由選擇當天要訂的晚餐。因為人數高達20人,為了不想浪費時間點餐,我起了個頭說要石蜜園(台中市中心的中價位精緻簡餐)以及迷客夏(連鎖飲料店),因為這兩家店的品項少,可以決定得很快。

  所有C院的同事幾乎都在幾小時內跟著選了石蜜園的餐點,但是參與演出的醫學生們似乎覺得,難得有人請客卻吃普通簡餐實在太沒趣了,於是他們八人自己選了肯德基的套餐,變成行政人員必須要聯絡三個店家,雖然多了一點飲食樂趣,但效率慢很多。也對啦,大五或許是他們能夠悠閒享受生活情趣的最後一年了,趁這段時間作一些拖慢進度的事情也好。
\item
  雜記:物價與其他{[}5{]}

  \begin{itemize}
  \tightlist
  \item
    C院午餐(紫米飯,油菜,蓮子,排骨,筊白筍)72元,「7-11」冰美式咖啡兩杯共80元,「泰山」氣泡水26元,「半天水」椰子水39元,「盛香珍」白桃果凍41元,C院牛肉麵90元
  \item
    記得要來研究pandoc文件轉換器跟排版
  \end{itemize}
\end{enumerate}

\hypertarget{ux6ce8ux91cb-comment-19}{%
\subsection{注釋 Comment}\label{ux6ce8ux91cb-comment-19}}

{[}1{]}
維基百科(英語:Wikipedia)是網路百科全書計畫,特點是自由內容、自由編輯。目前是全球網路上最大且最受大眾歡迎的參考工具書,名列全球十大最受歡迎的網站。本日誌相當多的內容剽竊自維基百科。

{[}2{]}
微軟(英語:Microsoft;NASDAQ:MSFT)是美國一家跨國電腦科技公司,以研發、製造、授權和提供廣泛的電腦軟體服務為主。總部位於美國華盛頓州的雷德蒙德,最為著名和暢銷的產品為Microsoft
Windows作業系統和Microsoft
Office辦公室軟體。C院的辦公室全面使用微軟產品。

{[}3{]} 整合開發環境(Integrated Development
Environment,簡稱IDE,也稱為Integration Design Environment、Integration
Debugging
Environment)是一種輔助程式開發人員開發軟體的應用軟體,在開發工具內部就可以輔助編寫原始碼文字、並編譯打包成為可用的程式,有些甚至可以設計圖形介面。

{[}4{]} 有關電腦,軟體,手機及網頁,請見蒼白球日誌0008,0010以及0011。

{[}5{]} 新台幣計價。有關新台幣請參見蒼白球日誌0007。

\hypertarget{ux9644ux9304-appendix-18}{%
\subsection{附錄 Appendix}\label{ux9644ux9304-appendix-18}}

無。

\hypertarget{ux84bcux767dux7403ux65e5ux8a8c002720191025}{%
\section{蒼白球日誌0027(20191025)}\label{ux84bcux767dux7403ux65e5ux8a8c002720191025}}

\hypertarget{ux65e5ux671f-date-26}{%
\subsection{日期 Date}\label{ux65e5ux671f-date-26}}

\begin{itemize}
\tightlist
\item
  世界協調時間2019年(中華民國108年,令和1年)10月25日 / Unix 紀元 18194
  日 / 星期五 / 蒼白球紀元第27日
\item
  October 25, 2019 (UTC) / 18194 days since Unix Epoch / Friday / Globus
  Pallidum day 27
\item
  特殊註記:
\end{itemize}

\hypertarget{ux5e74ux9f61-age-26}{%
\subsection{年齡 Age}\label{ux5e74ux9f61-age-26}}

\begin{itemize}
\tightlist
\item
  33 years 6 months 2 days
\item
  33 歲 6 個月 2 天
\end{itemize}

\hypertarget{ux672cux6587-content-26}{%
\subsection{本文 Content}\label{ux672cux6587-content-26}}

\begin{enumerate}
\def\labelenumi{\arabic{enumi}.}
\item
  85度C與菸與男人

  早上經過H大附近的85度C{[}1{]},看到騎樓的座位上坐著兩個45歲左右,肉壯黝黑有小腹,可能是從事勞力工作的男人。兩人一人含著電子菸{[}2{]},一人含著噴出陣陣白霧的普通香菸,正在進行一場充滿愉快香菸滋味的愜意談話。

  看到這個場景,我突然明白這種價格比普通飲料店高,裝潢又遠不如咖啡館舒適的店,為什麼能夠繼續存在了。在傳統茶坊式微之後,可以盡情抽菸談笑的半開放空間越來越少見了:一般的咖啡館空間過於封閉且價格過高,不適合隨時去點一杯飲料大聲談笑,飲料店則少有座位。在咖啡館與飲料店的中間,傳統茶坊壞死後留下的這個空腔,於是填滿了85度C,簡直就像是一種封閉的纖維化。這既是生理,也是病理,商業就跟醫學一樣奧妙。
\item
  來H大聽屁話

  今天H大的特論課是我自己的指導老師授課,在到達H大之前我就有一種不太妙的預感,感覺他一定不會好好上。果不其然,到了課堂上他隨便講講學佛的好處,跟佛經如何超越科學(注意:這是一堂生物資訊特論)就混過了一節,接下來交給博士生A與博士生B報告。

  本來以為今天的荒謬就這樣結束了,結果研究生A的投影片一打開,哇,這不是碩士生C、碩士生D跟我的合作案嗎?我知道C跟D常常需要跟A討教,結果沒想到其實A才是這案子的大頭目啊。我忍不住插嘴,在台下跟他交手,在談話中他攻破了我很多對於資訊技術不了解的地方,我也糾正了他對於臨床的誤解,互有來往,辯論了整整一節課,博士生B的報告只好草草報完。

  演出了一堂混亂又荒謬的課,我覺得我們實驗室真的太棒了。
\item
  母后與學弟妹合奏

  母后大人今天終於與醫學生所組成的弦樂四重奏合奏了我編的EXEC\_with\_METHOD\_METAFALICA/.,因為四人裡面有三人技術不錯,且相當具有團隊默契,因此很順利地完成了整首曲子。

  技術並無法克服一些天性的盲點,在這場合奏中母后覺得,這幾個人雖然技術到位,但是完全沒有情感,好像機器人拉琴沒什麼音樂感。不過這個缺陷真的算不可抗力因素,在現行的學制下,醫學系本來就很容易招到性格平穩的人,像我這種個性怪怪的有各式各樣難以言喻情緒的人算非常少數,所以想要從這裡面抽到一組情感澎湃的四重奏,真的非常困難啊\ldots\ldots 不要再抱怨了\ldots\ldots{}
\item
  雜記:物價與其他{[}3{]}

  「麥當勞」大麥克餐122元,「泰山」氣泡水26元(買一送一,店員用很親切的語氣跟我說「這是秘密折扣喔,沒有貼在店裡面宣傳,是我們私底下送的」)
\end{enumerate}

\hypertarget{ux6ce8ux91cb-comment-20}{%
\subsection{注釋 Comment}\label{ux6ce8ux91cb-comment-20}}

{[}1{]} 85度咖啡蛋糕烘焙專賣店,又稱85°C、85攝氏度或85度C(英語:85
Cafe, 85C Daily Cafe, 85°C Bakery Café, 85
Degrees),是一家來自台灣、於開曼群島註冊的跨國連鎖餐飲店。除咖啡、茶、蛋糕之外,亦有冰沙、果汁、伴手禮、麵包類等食品。

{[}2{]}
電子霧化器,俗稱「電子煙」;音譯「威卜」,是一種以可充電鋰聚合物電池供電驅動霧化器,透過加熱油艙中的電子液體,電子煙無需燃燒,只會產生「蒸氣」亦無焦油及一氧化碳,使部分醫學界人士認為其是有潛力的尼古丁替代品,理論上也可以減少部分傳統紙菸對身體的影響、火災發生的可能,並讓以電子煙替代紙菸的吸菸人士遠離紙菸中致癌物。

{[}3{]} 新台幣計價。有關新台幣請參見蒼白球日誌0007。

\hypertarget{ux9644ux9304-appendix-19}{%
\subsection{附錄 Appendix}\label{ux9644ux9304-appendix-19}}

\hypertarget{ux84bcux767dux7403ux65e5ux8a8c002820191026}{%
\section{蒼白球日誌0028(20191026)}\label{ux84bcux767dux7403ux65e5ux8a8c002820191026}}

\hypertarget{ux65e5ux671f-date-27}{%
\subsection{日期 Date}\label{ux65e5ux671f-date-27}}

\begin{itemize}
\tightlist
\item
  世界協調時間2019年(中華民國108年)10月26日 / Unix 紀元 18195 日 /
  星期六 / 蒼白球紀元第28日
\item
  October 26, 2019 (UTC) / 18195 days since Unix Epoch / Saturday /
  Globus Pallidum day 28
\end{itemize}

\hypertarget{ux5e74ux9f61-age-27}{%
\subsection{年齡 Age}\label{ux5e74ux9f61-age-27}}

\begin{itemize}
\tightlist
\item
  33 years 6 months 3 days
\item
  33 歲 6 個月 3 天
\end{itemize}

\hypertarget{ux672cux6587-content-27}{%
\subsection{本文 Content}\label{ux672cux6587-content-27}}

\begin{enumerate}
\def\labelenumi{\arabic{enumi}.}
\item
  玩了一整天的LaTeX{[}1{]}

  下周因為要去學校教課且工作多,預期會非常忙碌,所以今天早上去了醫院,準備先把工作清掉一些。不過事與願違,因為周日、周一的睡眠不足造成的影響,完全無法專心工作,反而在辦公室的螢幕玩起了LaTeX排版。

  一開始只是想說稍微放鬆一下,拿已經累積到編號27{[}2{]}的日誌來排排看,結果一排就入迷了。此物莫名其妙的複雜度、隨時會冒出來的Bug{[}3{]},以及經過重重關卡產生美麗排版的成就感,帶來一種電玩般的遊戲性,不知不覺就玩了好幾個小時。最後我要回家時,落寞地發現片子只寫了10\%。好吧,明天再寫好了。
\item
  有關研究

  醫學研究者應該要常常吸收新知,以免知識落後影響研究,然而自稱醫學研究者的我,已經好久沒有讀期刊了,真是慚愧。沒辦法,一方面工作真的多,一方面很難抗拒各種興趣(嗯例如說LaTeX)的誘惑,要把心定下來好好唸書唸期刊,真的是很困難的事。

  到底應該要好好約束自己,隨時注意精進,還是乾脆放輕鬆呢\ldots\ldots{}
\item
  雜記:物價與其他{[}4{]}

  \begin{itemize}
  \item
    「波波食堂Bubble House
    台中店」抹茶鬆餅:最近才開來台中的連鎖店,賣約15公分見方,氣泡布{[}5{]}造型的鬆餅,純鬆餅價格是65元,如需要淋醬或餡料則要另外加價。這個東西以外帶點心來講其實滿貴的,本來覺得自己怎麼會這樣亂花錢,但是在入口的那一刻,卻完全被折服了,覺得65元划算。他酥脆的外皮彷彿洋芋片般的觸感,在咬破外皮後,裡面的鬆餅體濕潤柔軟且充滿孔洞,像長崎蛋糕的口感卻有蜂巢糕的結構。我覺得我會光顧第二次。
  \item
    如果大麥克{[}6{]}可以做為物價的評估標準,那麼診斷一個痔瘡的花費跟大麥克價格的比例,可不可以當成醫療費用的基準呢?我覺得在貨幣因為通膨而混亂的今日,我們需要用這樣的比值來評估事物的價值,過幾天來研究看看好了。
  \item
    「福米記池上便當」烤肉飯75元,「李記紅茶冰」麥香紅茶25元,「喫茶小舖」咖啡紅茶30元,(忘記店名的麵店)餛飩麵60元,青菜30元
  \end{itemize}
\end{enumerate}

\hypertarget{ux6ce8ux91cb-comment-21}{%
\subsection{注釋 Comment}\label{ux6ce8ux91cb-comment-21}}

{[}1{]} TeX是由Donald Knuth (1938-) 創造的電子排版系統,LaTeX則是Leslie
Lamport (1941-)
所開發,對於Tex的延伸與改良。LaTeX的特色是使用者僅用指令決定版面規劃,細部排版設計由程式產生,因此即使美感不好的使用者也可以藉由這套工具,產生優雅且一致的排版。

{[}2{]}
指蒼白球日誌0001-0027,其實叫做Op.27聽起來比較高檔,但是Op似乎是樂譜專用的,所以還是叫編號27好了。

{[}3{]}
程式錯誤(英語:Bug),是程式設計中的術語,是指在軟體執行中因為程式本身有錯誤而造成的功能不正常、當機、資料遺失、非正常中斷等現象。有些程式錯誤會造成電腦安全隱患,此時叫做漏洞。

{[}4{]} 新台幣計價。有關新台幣請參見蒼白球日誌0007。

{[}5{]} 氣泡布(英文:Bubble Wrap)
是一種塑膠包裝材料,一般呈透明狀,上面布滿注入空氣的小氣泡,由於氣泡能有效緩衝,避免包裹碰撞,因此通常用來包裝易碎或不耐衝擊的物品。捏碎氣泡布是一種常見的紓壓管道。

{[}2{]}
指蒼白球日誌0001-0027,其實叫做Op.27聽起來比較高檔,但是Op似乎是樂譜專用的,所以還是叫編號27好了。

{[}3{]}
程式錯誤(英語:Bug),是程式設計中的術語,是指在軟體執行中因為程式本身有錯誤而造成的功能不正常、當機、資料遺失、非正常中斷等現象。有些程式錯誤會造成電腦安全隱患,此時叫做漏洞。

{[}4{]} 新台幣計價。有關新台幣請參見蒼白球日誌0007。

{[}5{]} 氣泡布(英文:Bubble Wrap)
是一種塑膠包裝材料,一般呈透明狀,上面布滿注入空氣的小氣泡,由於氣泡能有效緩衝,避免包裹碰撞,因此通常用來包裝易碎或不耐衝擊的物品。捏碎氣泡布是一種常見的紓壓管道。

{[}6{]} 大麥克指數(英語:Big Mac
index)是一個非正式的經濟指數,在假定購買力平價理論成立的前提下,用以測量兩種貨幣的匯率理論上是否合理。這種測量方法,以各國麥當勞餐廳的大麥克漢堡價格,相對於該國貨幣匯率,作為比較的基準。這個指數在1986年由《經濟學人》雜誌推出,每年出版一次,該指數在英語國家裡衍生了漢堡經濟(Burgernomics)一詞。

{[}6{]} 大麥克指數(英語:Big Mac
index)是一個非正式的經濟指數,在假定購買力平價理論成立的前提下,用以測量兩種貨幣的匯率理論上是否合理。這種測量方法,以各國麥當勞餐廳的大麥克漢堡價格,相對於該國貨幣匯率,作為比較的基準。這個指數在1986年由《經濟學人》雜誌推出,每年出版一次,該指數在英語國家裡衍生了漢堡經濟(Burgernomics)一詞。

\hypertarget{ux9644ux9304-appendix-20}{%
\subsection{附錄 Appendix}\label{ux9644ux9304-appendix-20}}

\hypertarget{ux84bcux767dux7403ux65e5ux8a8c002920191027}{%
\section{蒼白球日誌0029(20191027)}\label{ux84bcux767dux7403ux65e5ux8a8c002920191027}}

\hypertarget{ux65e5ux671f-date-28}{%
\subsection{日期 Date}\label{ux65e5ux671f-date-28}}

\begin{itemize}
\tightlist
\item
  世界協調時間2019年(中華民國108年)10月27日 / Unix 紀元 18196 日 /
  星期日 / 蒼白球紀元第29日
\item
  October 27, 2019 (UTC) / 18196 days since Unix Epoch / Sunday / Globus
  Pallidum day 29
\item
  特殊註記:
\end{itemize}

\hypertarget{ux5e74ux9f61-age-28}{%
\subsection{年齡 Age}\label{ux5e74ux9f61-age-28}}

\begin{itemize}
\tightlist
\item
  33 years 6 months 4 days
\item
  33 歲 6 個月 4 天
\end{itemize}

\hypertarget{ux672cux6587-content-28}{%
\subsection{本文 Content}\label{ux672cux6587-content-28}}

\begin{enumerate}
\def\labelenumi{\arabic{enumi}.}
\item
  伊斯蘭國{[}1{]}的末日

  在2019年3月伊斯蘭國最後的據點被攻克之後的幾個月間,殘餘勢力仍在伊拉克及敘利亞繼續活動,直到今日。

  美東時間{[}2{]}2019年10月26日早上九點,美利堅合眾國總統唐納川普(Donald
  Trump)宣布伊斯蘭國領袖巴格達迪{[}3{]}於敘利亞引爆自殺炸彈身亡。今日之後,伊斯蘭國可以說是實質滅亡了。中東歷史又開始了新的一頁。
\item
  有關光儲存媒體

  今天在幫母后做他要交的資料的時候,燒了一張光碟附在,在這個過程中突然想到,我好像已經超過半年沒有使用任何一張光碟了,任何檔案都是直接使用磁碟或網路傳輸。六七年前我還不相信光儲存媒體會死亡,但今天我發現,光儲存媒體已經名存實亡了。
\item
  完全不行的一天

  起床做一點母后的雜事以後,原本要來修改禮拜五要用的譜,結果一打開打譜軟體,打了幾個音就覺得超級不舒服,這幾天可能已經過勞了,無法再負荷任何工作。

  於是我就跑去醫院附近閒晃了一天。真的是該休息了。
\end{enumerate}

\hypertarget{ux6ce8ux91cb-comment-22}{%
\subsection{注釋 Comment}\label{ux6ce8ux91cb-comment-22}}

{[}1{]} 伊斯蘭國(阿拉伯語:الدولة الإسلامية‎,轉寫:ad-Dawlat
al-Islamiya;英語:The Islamic
State,縮寫:IS),是一個活躍在伊拉克和敘利亞的政治實體,奉行極端保守的伊斯蘭教瓦哈比派,組織領袖巴格達迪自封為哈里發,定國號為「伊斯蘭國」,宣稱自身對於整個穆斯林世界擁有統治地位。

{[}2{]} 北美東部時區(North American Eastern Time
Zone),或稱美國東部時間(The Eastern Time
Zone,ET),主要包括北美東海岸和南美西海岸,其標準時間(EST)為UTC-5,夏令時間(EDT)為UTC-4

{[}3{]}
易卜拉欣·阿瓦德·易卜拉欣·阿里·穆罕默德·巴德里·薩瑪拉·阿布·貝克爾·巴格達迪(阿拉伯語:إبراهيم
عواد إبراهيم علي محمد البدري السامرائي أبو بكر
البغدادي‎;1971年7月28日-2019年10月26日),又被稱作阿布·貝克爾·巴格達迪(أبو
بكر البغدادي‎)、哈里發易卜拉欣(خَلِيفَةُ
إِبْرَاهِيم‎)或阿布·杜阿(أبو
دعاء‎){[}1{]},極端伊斯蘭恐怖主義組織「伊斯蘭國」的首領,並自封為「伊斯蘭國」的「哈里發」

\hypertarget{ux9644ux9304-appendix-21}{%
\subsection{附錄 Appendix}\label{ux9644ux9304-appendix-21}}

\hypertarget{ux84bcux767dux7403ux65e5ux8a8c003020191028}{%
\section{蒼白球日誌0030(20191028)}\label{ux84bcux767dux7403ux65e5ux8a8c003020191028}}

\hypertarget{ux65e5ux671f-date-29}{%
\subsection{日期 Date}\label{ux65e5ux671f-date-29}}

\begin{itemize}
\tightlist
\item
  世界協調時間2019年(中華民國108年,令和元年)10月28日 / Unix 紀元 18197
  日 / 星期一 / 蒼白球紀元第30日
\item
  October 28, 2019 (UTC) / 18197 days since Unix Epoch / Monday / Globus
  Pallidum day 30
\end{itemize}

\hypertarget{ux5e74ux9f61-age-29}{%
\subsection{年齡 Age}\label{ux5e74ux9f61-age-29}}

\begin{itemize}
\tightlist
\item
  33 years 6 months 5 days
\item
  33 歲 6 個月 5 天
\end{itemize}

\hypertarget{ux672cux6587-content-29}{%
\subsection{本文 Content}\label{ux672cux6587-content-29}}

\begin{enumerate}
\def\labelenumi{\arabic{enumi}.}
\item
  C校更新設備

  今天在C校某棟紅磚色的老大樓替學生上課,看到教室裡面的樣子好像跟以前不太一樣,原來是幾十年歷史的老課桌椅被換成了新品。雖然說把老舊設備換掉是好事,不過十幾年前用過的東西突然不見,不免有點感嘆,覺得時代已經棄我而去了。或許有哪一天那個紅磚色的大樓也會被拆掉,那麼我們這代曾經在C校當過學生的印記也就消滅了。

  還好學生上課的時候對我還是滿有敬意的,代表至少是個有資格跟小鮮肉{[}1{]}講話的前輩,而不是青春不再就必須丟棄的課桌椅。過去我總是羨慕擁有良好外貌跟身材,且生活型態充滿活力的同齡或是年輕男性,認為他們擁有真正意義上的青春,但自己開始老邁之後,逐漸看到了一個新的視角:青春總有一天會剝落,當剝落的時候還剩下什麼課桌椅以外的東西,似乎也是很重要的。
\item
  男友的店

  知道男友跟開店合夥人出了一些不好處理的問題後,自以為是喜歡裝懂的我,就硬是講了幾句自己對這件事情的意見。事後其實滿後悔的,畢竟我長期活在父母跟醫院的保護傘下,沒有開過店,接觸的人群也不是很多,硬是對這種經營跟人際關係的事情發表意見,並不是那麼恰當。但說了也就說了,希望不要對他有太不好的影響。

  另外,本來滿想仔細描述那個不是很好處理的問題,因為對我來講很新奇,不過第一這是別人的隱私,第二我還想要把這些日誌印一份放他店裡,所以還是收斂一點好了。
\item
  雜記:物價與其他{[}2{]}

  \begin{itemize}
  \tightlist
  \item
    睡得很飽!!!希望每天都能像這樣!
  \item
    睡得很飽的代價是好多事情擺爛{[}3{]}偷懶。尤其是譜還有研究,都放著沒做。好慚愧。
  \item
    C院餐廳午餐72元(紫米飯,黃豆芽,A菜{[}4{]},莧菜),C院餐廳晚餐72元(白飯,小白菜,地瓜葉,芥菜),「清心福全」普洱25元,「黑松」黑麥茶25元(購自C校販賣機,加了奇怪的異麥芽寡糖,真的不好喝,難怪淪落到販賣機)
  \end{itemize}
\end{enumerate}

\hypertarget{ux6ce8ux91cb-comment-23}{%
\subsection{注釋 Comment}\label{ux6ce8ux91cb-comment-23}}

{[}1{]} 小鮮肉用於形容年輕、帥氣、陽光、有肌肉的男孩及男青年。

{[}2{]} 新台幣計價。有關新台幣請參見蒼白球日誌0007。

{[}3{]} 簡化字形:摆烂,國語注音:ㄅㄞˇ ㄌㄢˋ,漢語拼音:bǎi
làn,釋義:指任由事情往壞的方向繼續發展下去而無所作為,不長進的意思。

{[}4{]} 為菊科植物萵苣(學名:Lactuca
sativa)的一種品種,因閩南語「萵」音近「A」因而被稱為A菜。

\hypertarget{ux9644ux9304-appendix-22}{%
\subsection{附錄 Appendix}\label{ux9644ux9304-appendix-22}}

\hypertarget{ux84bcux767dux7403ux65e5ux8a8c003120191029}{%
\section{蒼白球日誌0031(20191029)}\label{ux84bcux767dux7403ux65e5ux8a8c003120191029}}

\hypertarget{ux65e5ux671f-date-30}{%
\subsection{日期 Date}\label{ux65e5ux671f-date-30}}

\begin{itemize}
\tightlist
\item
  世界協調時間2019年(中華民國108年)10月29日 / Unix 紀元 18198 日 /
  星期二 / 蒼白球紀元第31日
\item
  October 29, 2019 (UTC) / 18198 days since Unix Epoch / Tuesday /
  Globus Pallidum day 31
\item
  特殊註記:
\end{itemize}

\hypertarget{ux5e74ux9f61-age-30}{%
\subsection{年齡 Age}\label{ux5e74ux9f61-age-30}}

\begin{itemize}
\tightlist
\item
  33 years 6 months 6 days
\item
  33 歲 6 個月 6 天
\end{itemize}

\hypertarget{ux672cux6587-content-30}{%
\subsection{本文 Content}\label{ux672cux6587-content-30}}

\begin{enumerate}
\def\labelenumi{\arabic{enumi}.}
\item
  吃獅子頭突然想到

  獅子頭其實是一個非常複雜的食品。它含有豬絞肉,香菇,蔥,還有白菜。豬肉,白菜跟蔥可能來自雲林的大型農場,且必定是兩個以上的農場,因為種菜的人不懂養豬。香菇可能來自南投,或者是自中華人民共和國。這樣就可能牽涉兩國,四個農場了。

  若再追究豬吃的飼料跟種菜的肥料的話,則會牽涉到更多國家更多產業,因為飼料玉米可能來自美國或巴西,肥料則可能來自各國礦產。即使是一道日常菜餚,牽涉的產業數量也已經遠超過一般人所能想像的範圍,更不用提食衣住行的其他方面了。事實上,在開發中或已開發國家,人們不但不是自給自足的,而且是依賴著綿密的產業鏈而活著,一但這個系統有任何一個地方斷裂,所有人的生活都將受到巨大的影響。

  因此,包含我自己在內大多數的島民都應該保持謙卑,對資本主義與現代工業懷抱感謝與敬畏。也許這套系統有很多傾斜腐敗噁心的地方,但我們大部分的生活都已經綁在上面了,脫鉤只會造成很多不便,我們只能設法改善它,不能一味譴責。
\item
  把香蕉飴寄去台北了

  去店裡拿了要給男友嘗鮮的香蕉飴{[}1{]}與咖哩蠶豆,在回醫院的計程車上,看著那些可愛的糯米塊,忍不住偷偷吃了一塊,結果覺得有點失落。這個溶劑味也太淡了吧!沒有進入化學實驗室的感覺,怎麼可以叫做香蕉飴呢?

  可是回頭想想,畢竟我是從大廠「元福麻油」買的,大廠賣的東西必須打保險牌,萬一他出了味道真的很重像在吸膠的道地香蕉飴,結果被不懂吃的顧客客訴而喪失商譽的話,對於元福麻油這塊招牌來講太划不來了,所以他必定逐漸調整成大眾一點的口味。

  或許,道地的化工廠溶劑黃金體驗香蕉飴,只能去小攤子找了。
\item
  雜記:物價與其他{[}2{]}

  \begin{itemize}
  \tightlist
  \item
    計程車兩趟330元,宅急便{[}3{]}運費170元,C院午餐72元(紫米飯,茼蒿,白菜,豆腐,獅子頭),「多喝水」氣泡水26元,C院牛肉麵90元,空氣清淨機濾心790元,「Simple」卸妝水259元,刷樂醫生牙刷四支110元(買貴了,但momo購物網方便所以\ldots\ldots),「NatureMate」大劑量維他命C
    60顆409元
  \item
    今天本來因為睡不好而精神很不好,結果晚上去聽主治醫師大會,太無聊睡著以後精神大振,可以開始加班了。莫非這種睡眠才最提神?
  \end{itemize}
\end{enumerate}

\hypertarget{ux6ce8ux91cb-comment-24}{%
\subsection{注釋 Comment}\label{ux6ce8ux91cb-comment-24}}

{[}1{]} 此事見蒼白球日誌0021。

{[}2{]} 新台幣計價。有關新台幣請參見蒼白球日誌0007。

{[}3{]}
統一速達股份有限公司是一家台灣的宅配{[}4{]}業者,由統一流通次集團旗下的統一超商全資轉投資設立,技術由日本宅配業者大和運輸(ヤマト運輸)提供,獲得大和運輸授權以「黑貓宅急便」(簡稱宅急便)為品牌。

{[}4{]}
宅配來自於日語,日本又稱為「宅配便」。宅配在日本為貨物到府配送服務(含快遞)的通稱,有別於一般郵遞。大和運輸所開創的「宅急便」為有別於傳統貨運與快遞而獨樹一格的服務模式,這種服務模式在2000年正式由台灣宅配通及統一速達相繼引入台灣後亦稱為「宅配」或「宅配通」、「宅急便」。基於此淵源脈絡,「宅配」與「快遞」在台灣有意義上的差別,但隨著傳統快遞業相繼投入日本式的宅配服務市場,兩者的差別又有日趨模糊的傾向。在電子商務的發展中,宅配扮演了「最後一哩」的角色。

\hypertarget{ux9644ux9304-appendix-23}{%
\subsection{附錄 Appendix}\label{ux9644ux9304-appendix-23}}

\hypertarget{ux84bcux767dux7403ux65e5ux8a8c003220191030}{%
\section{蒼白球日誌0032(20191030)}\label{ux84bcux767dux7403ux65e5ux8a8c003220191030}}

\hypertarget{ux65e5ux671f-date-31}{%
\subsection{日期 Date}\label{ux65e5ux671f-date-31}}

\begin{itemize}
\tightlist
\item
  世界協調時間2019年(中華民國108年,令和1年)10月30日 / Unix 紀元 18199
  日 / 星期三 / 蒼白球紀元第32日
\item
  October 30, 2019 (UTC) / 18199 days since Unix Epoch / Wednesday /
  Globus Pallidum day 32
\item
  特殊註記:
\end{itemize}

\hypertarget{ux5e74ux9f61-age-31}{%
\subsection{年齡 Age}\label{ux5e74ux9f61-age-31}}

\begin{itemize}
\tightlist
\item
  33 years 6 months 7 days old / 2 years 0 months 18 days after
  acquiring ROC Surgical Pathology Licence
\item
  33 歲 6 個月 7 天 / 成為病理專科醫師 2 年 0 個月 18 天
\end{itemize}

\hypertarget{ux672cux6587-content-31}{%
\subsection{本文 Content}\label{ux672cux6587-content-31}}

\begin{enumerate}
\def\labelenumi{\arabic{enumi}.}
\item
  久違的健身教練課

  因為各種繁重的腦力業務,跟教練請假了大概半個月吧,雖然說勉強回來上課了但是還是狀況很差,肌力大幅衰退。在狀況不好的時候,又看著健身房裡面那些練拳的,氣色很好的健壯青年正在聊著練拳的話題以及武館的各種八卦,實在是又羨慕跟嫉妒。畢竟我在在身體素質還沒養起來之前就先老去了,已經沒時間且來不及擠進健壯青年的荷爾蒙圈圈了,那個世界終究不是屬於我的。

  或許我的宿命就是在無盡的書面作業上面加速老去,從此變成被青年無視的人吧。有點遺憾,但也就這樣了。身體素質以外的能力是無法彌補身體素質的。
\item
  今天早早把日誌完成的原因

  因為做一些有的沒有的(尤其是演出,我到底接那些事情幹嘛),所以把所有切片都堆到這周了。因此我今天可能要工作到半夜兩點,然後明天早上七點又要去開會。在累倒之前先把日誌撇一撇。
\item
  雜記:物價與其他{[}2{]}

  C院午餐72元(紫米飯,獅子頭,油菜花,杏鮑菇,然後忘記一種青菜),忘記名字的烤餅攤鮪魚烤餅55元,很難喝又貴的仙草牛奶60元
\end{enumerate}

\hypertarget{ux6ce8ux91cb-comment-25}{%
\subsection{注釋 Comment}\label{ux6ce8ux91cb-comment-25}}

{[}1{]} 新台幣計價。有關新台幣請參見蒼白球日誌0007。

\hypertarget{ux9644ux9304-appendix-24}{%
\subsection{附錄 Appendix}\label{ux9644ux9304-appendix-24}}

\hypertarget{ux84bcux767dux7403ux65e5ux8a8c003320191031}{%
\section{蒼白球日誌0033(20191031)}\label{ux84bcux767dux7403ux65e5ux8a8c003320191031}}

\hypertarget{ux65e5ux671f-date-32}{%
\subsection{日期 Date}\label{ux65e5ux671f-date-32}}

\begin{itemize}
\tightlist
\item
  世界協調時間2019年(中華民國108年,令和1年)10月31日 / Unix 紀元 18200
  日 / 星期四 / 蒼白球紀元第33日
\item
  October 31, 2019 (UTC) / 18200 days since Unix Epoch / Thursday /
  Globus Pallidum day 33
\end{itemize}

\hypertarget{ux5e74ux9f61-age-32}{%
\subsection{年齡 Age}\label{ux5e74ux9f61-age-32}}

\begin{itemize}
\tightlist
\item
  33 years 6 months 8 days old / 2 years 0 months 19 days after
  acquiring ROC Surgical Pathology Licence
\item
  33 歲 6 個月 8 天 / 成為病理專科醫師 2 年 0 個月 19 天
\end{itemize}

\hypertarget{ux672cux6587-content-32}{%
\subsection{本文 Content}\label{ux672cux6587-content-32}}

\begin{enumerate}
\def\labelenumi{\arabic{enumi}.}
\item
  有關杜甫

  我思考如何長期保存日誌,已經思考了一整個月,發現使用紙本保存其實在空間與材料上並不便宜,三年的日誌就可以吃掉一個櫥櫃的空間,若需搬動則會耗費兩三個箱子,在緊急時根本帶不走。至於數位格式,則需要憂心未來的讀取性。想到這裡,就忍不住開始質疑起一千多年前的詩人杜甫(712年-770年)作品的真實性。

  看到這裡讀者應該一頭霧水,我怎麼從紙本保存的昂貴性牽連到杜甫作品真實性的?聽起來完全沒有關聯啊?且慢慢聽我說。文獻上所記載的杜甫生平有兩個顯著的特點,其一是他留下了約1500首的大量詩作,其二是他後半生有很長的時間在逃難,且生活窘迫。眾所皆知,八世紀的紙張成本與所佔空間比現代要高很多,如果紙本的保存在現代依然昂貴不便,
  在八世紀想必更加困難。那麼,請問杜甫如何在生活窘迫且到處逃難的狀態下,保存如此多的紙本資料?

  唯一合理的解釋可能是:當時一些生活比較安適的杜甫親友真的獨具慧眼,替狀況不好的杜甫搶救大量的手稿,但如果有能力保存這些手稿,為何沒有能力改善杜甫的生活狀態?莫非真的是冷血的留稿不留人?在唯一合理的解釋依然充滿疑點的狀況下,或許真的有相當量的杜詩是偽託也說不定。
\item
  病理實驗課

  病理實驗課不是你們想像中的那種「實驗」,病理實驗課其實就是教學生怎麼看病理切片{[}1{]}。這個其實意外的,非常困難,因為學習病理切片判讀是一種特徵識別(Pattern
  Recognition)的作業,像機器學習一樣需要在腦子裡面餵大量重複的資料,才得以窺其堂奧,並非用聰明才智,看一兩個切片就能快速學得會的東西,因此在短短的課堂中有很多東西很難讓他們領悟。

  於是,我只好說出很多「其實這個大四不用會,也不會考」,「你們掌握到這一半的要領就好了」等等安慰的話,畢竟如果大四就能夠領悟這些特徵識別的東西的話,就不需要住院醫師訓練了\ldots\ldots{}
\item
  雜記:物價與其他{[}2{]}

  \begin{itemize}
  \tightlist
  \item
    編曲\ldots\ldots 準時編完了!叫我編曲王!
  \item
    C院午餐72元(紫米飯,茄子,花蓮豆炒花枝,地瓜葉,獅子頭),「黑松」茶花綠茶25元(購自C校販賣機)「美蓁小吃店」豬肝20元,餛飩湯30元,中乾意麵60元,空心菜30元,「蜜兔」青草茶35元
  \end{itemize}
\end{enumerate}

\hypertarget{ux6ce8ux91cb-comment-26}{%
\subsection{注釋 Comment}\label{ux6ce8ux91cb-comment-26}}

{[}1{]} 有關病理切片,可見蒼白球日誌0011。

{[}2{]} 新台幣計價。有關新台幣請參見蒼白球日誌0007。

\hypertarget{ux9644ux9304-appendix-25}{%
\subsection{附錄 Appendix}\label{ux9644ux9304-appendix-25}}

\hypertarget{ux84bcux767dux7403ux65e5ux8a8c003420191101}{%
\section{蒼白球日誌0034(20191101)}\label{ux84bcux767dux7403ux65e5ux8a8c003420191101}}

\hypertarget{ux65e5ux671f-date-33}{%
\subsection{日期 Date}\label{ux65e5ux671f-date-33}}

\begin{itemize}
\tightlist
\item
  世界協調時間2019年(中華民國108年,令和1年)11月1日 / Unix 紀元 18201 日
  / 星期五 / 蒼白球紀元第34日
\item
  November 01, 2019 (UTC) / 18201 days since Unix Epoch / Friday /
  Globus Pallidum day 34
\item
  特殊註記:
\end{itemize}

\hypertarget{ux5e74ux9f61-age-33}{%
\subsection{年齡 Age}\label{ux5e74ux9f61-age-33}}

\begin{itemize}
\tightlist
\item
  33 years 6 months 9 days old / 2 years 0 months 20 days after
  acquiring ROC Surgical Pathology Licence
\item
  33 歲 6 個月 9 天 / 成為病理專科醫師 2 年 0 個月 20 天
\end{itemize}

\hypertarget{ux672cux6587-content-33}{%
\subsection{本文 Content}\label{ux672cux6587-content-33}}

\begin{enumerate}
\def\labelenumi{\arabic{enumi}.}
\item
  今天把吃的放在正式條目吧,別放雜記

  因為早上聽的課實在太無聊了,老師講他自己的專業項目,我全無興趣,只能滑手機度過。這個狀況下不如來寫寫中午路過莒光新城{[}1{]}時吃的麵店「大城小吃」。

  莒光新城不是普通眷村,是與一中商圈{[}3{]}鄰接的住宅區,直接面對一中街的挑戰,開在這種地段還能夠常常門庭若市,想必一定好吃。這個想法,再加上對於「眷村」的刻板印象,讓我原本很想嚐嚐他麵食有多高明。無奈的是,因為肚子真的很餓,下午晚上又要趕片子跟團練,吃麵可能會撐不過去,只好在麵店點了爌肉飯來吃。

  麵店的爌肉飯,聽起來就不期不待,沒想到一吃簡直是中到大獎,這個爌肉飯水準之高,可能已經超過了第二市場名店的水準。充滿光澤的三層肉滷成漂亮的紅磚色,咬下去居然酥爛到入口即化,放出美麗的醬油香{[}4{]}以及多種中藥香料的味道,正在納悶他加的香料有哪些的時候,口中咬到了一片月桂葉,喔!原來是這個啊!真是太棒了!

  所附的酸筍跟油豆腐也頗具水準,與爌肉及白飯搭配成雋永的四重奏,是布拉姆斯嗎,或是舒伯特?還是蕭士塔高維契?總之這爌肉四重奏是大師之作。而另外點的粉腸湯更是絕妙得無法形容,來麵店吃飯還真的吃對了。爌肉飯50元{[}5{]},粉腸湯35元。「台灣第一味」甘蔗青茶50元(「先生不好意思,雖然你出示了識別證,但是兩點以後才有打折喔!」)。

  晚餐因為團練,所以C院請客。
\item
  團練

  重頭戲是學弟妹跟母后的鋼琴五重奏,因為母后之前嫌那首曲子的結尾編得太冷場太難聽了,於是我就硬插了一個無比激情的曲子進去收尾,編的時候把我自己操到作息亂掉(見蒼白球日誌0030-0033),然後今天練的時候把他們操到氣喘吁吁。這幾個人應該很久沒有拉這麼難的曲子了。

  不過痛苦是有代價的,這樣合起來效果真的還不錯,再配上一個實力高超的鈴鼓手(醫事室主任的手下,所以被逼著跟我們演出),相信應該是11月11日醫師節的亮點了。
\item
  馬橋詞典閱讀筆記

  馬橋詞典的結語,跟蒼白球日誌想要表達的核心很像,茲節錄了一段如下:
  「從嚴格的意義上來說,所謂``共同的語言'',永遠是人類一個遙遠的目標。如果我們下希望交流成為一種互相抵銷,互相磨滅,我們就必須對交流保持警覺和抗拒,在妥協中守護自己某種頑強的表達------這正是一種良性交流的前提。這就意味著,人們在說話的時候,如果可能的話,每個人都需要一本自己特有的詞典。」
\end{enumerate}

\hypertarget{ux6ce8ux91cb-comment-27}{%
\subsection{注釋 Comment}\label{ux6ce8ux91cb-comment-27}}

{[}1{]}
為一眷村{[}2{]}改建的大型集合住宅,居住者主要以空軍將官為主,因此算是頗為高級的眷村,其飲食與器物也比一般眷村要有格調。

{[}2{]}
眷村是指台灣自1949年起至1960年代,來自中國大陸各省的中華民國國軍及其眷屬,因第二次國共內戰失利而隨中華民國政府遷徙至台灣後,政府機關為其興建或者配置的村落。

{[}3{]} 有關大型商圈「一中街」,請見蒼白球日誌0013。

{[}4{]}
我是濁水溪出身的人,濁水溪產最棒的醬油,所以用好醬油我一吃就知道。

{[}5{]} 新台幣計價。有關新台幣請參見蒼白球日誌0007。

\hypertarget{ux9644ux9304-appendix-26}{%
\subsection{附錄 Appendix}\label{ux9644ux9304-appendix-26}}

\hypertarget{ux84bcux767dux7403ux65e5ux8a8c003520191102}{%
\section{蒼白球日誌0035(20191102)}\label{ux84bcux767dux7403ux65e5ux8a8c003520191102}}

\hypertarget{ux65e5ux671f-date-34}{%
\subsection{日期 Date}\label{ux65e5ux671f-date-34}}

\begin{itemize}
\tightlist
\item
  世界協調時間2019年(中華民國108年,令和1年)11月2日 / Unix 紀元 18202 日
  / 星期六 / 蒼白球紀元第35日
\item
  November 02, 2019 (UTC) / 18202 days since Unix Epoch / Saturday /
  Globus Pallidum day 35
\item
  特殊註記:
\end{itemize}

\hypertarget{ux5e74ux9f61-age-34}{%
\subsection{年齡 Age}\label{ux5e74ux9f61-age-34}}

\begin{itemize}
\tightlist
\item
  33 years 6 months 10 days old / 2 years 0 months 21 days after
  acquiring ROC Surgical Pathology Licence
\item
  33 歲 6 個月 10 天 / 成為病理專科醫師 2 年 0 個月 21 天
\end{itemize}

\hypertarget{ux672cux6587-content-34}{%
\subsection{本文 Content}\label{ux672cux6587-content-34}}

\begin{enumerate}
\def\labelenumi{\arabic{enumi}.}
\item
  很慚愧

  今天遠赴P市講了一場很爛的演講以後,去男友店裡坐坐,發現他忙進忙出招呼客人,忙到沒時間寫日記。看到他忙碌的情形突然覺得很慚愧,我每天還可以空時間出來寫蒼白球日誌,應該是因為很多工作偷懶擺爛的緣故。

  想到這裡就決定要來充實一下,坐車回C市以後馬上到辦公室來進行病理工作,並且拿研究用的影像回家稍微閱讀一下。雖然這個努力工作可能是曇花一現,明天搞不好就繼續擺爛了,但是稍微可以彌補一下我短期的慚愧感吧。
\item
  續第一點,之男友店裡的情形

  這家店真的很會做生意,在週六下午這個時段一直維持坐滿八成的榮景,但最讓我詫異的倒不是男友經營事業如此長袖善舞,是人們居然對帶點刺激酸味的東西趨之若鶩,甘香微苦的美好品項反而乏人問津,北部人真的讓我難以理解。

  我們濁水溪人講究「甘甘香香軟軟」,酸不可以過酸,苦不可以過苦,鹹不可以過鹹,辣不可以過辣,所以我對於甜食的想像都是砂糖與香料,甜滋滋香噴噴軟綿綿的東西,例如說什麼呢,嗯,雖然香蕉飴真的很濁水,但是不要提香蕉飴好了,例如說泡芙,薄荷巧克力冰淇淋,奶油楓糖煎香蕉,或者是用大量砂糖丁香跟肉桂粉煮到酸味全失的蘋果餡。北部人不吃甘甘軟軟這一套,反而一到店裡就吃酸,讓我突然發現濁水溪流域其實並不能代表台灣,我的眼界真的太狹窄了\ldots\ldots{}
\item
  雜記:物價與其他{[}1{]}

  高鐵P市到C市來回共1350元,P市中心一家看起來很破舊的自助餐店午餐100元(爌肉,地瓜葉,花菜,莧菜,不怎麼樣又貴卻生意很好,P市居大不易!),「Fruit
  Drink」青茶30元,「樂檸漢堡」培磨牛肉堡雞塊青茶共179元(沒吃過樂檸的人請務必吃至少一次,身為台灣人沒吃過樂檸是很可恥的)
\end{enumerate}

\hypertarget{ux6ce8ux91cb-comment-28}{%
\subsection{注釋 Comment}\label{ux6ce8ux91cb-comment-28}}

{[}1{]} 新台幣計價。有關新台幣請參見蒼白球日誌0007。

\hypertarget{ux9644ux9304-appendix-27}{%
\subsection{附錄 Appendix}\label{ux9644ux9304-appendix-27}}

\hypertarget{ux84bcux767dux7403ux65e5ux8a8c003620191103}{%
\section{蒼白球日誌0036(20191103)}\label{ux84bcux767dux7403ux65e5ux8a8c003620191103}}

\hypertarget{ux65e5ux671f-date-35}{%
\subsection{日期 Date}\label{ux65e5ux671f-date-35}}

\begin{itemize}
\tightlist
\item
  世界協調時間2019年(中華民國108年,令和1年)11月3日 / Unix 紀元 18203 日
  / 星期日 / 蒼白球紀元第36日
\item
  November 03, 2019 (UTC) / 18203 days since Unix Epoch / Sunday /
  Globus Pallidum day 36
\end{itemize}

\hypertarget{ux5e74ux9f61-age-35}{%
\subsection{年齡 Age}\label{ux5e74ux9f61-age-35}}

\begin{itemize}
\tightlist
\item
  33 years 6 months 11 days old / 2 years 0 months 22 days after
  acquiring ROC Surgical Pathology Licence
\item
  33 歲 6 個月 11 天 / 成為病理專科醫師 2 年 0 個月 22 天
\end{itemize}

\hypertarget{ux672cux6587-content-35}{%
\subsection{本文 Content}\label{ux672cux6587-content-35}}

\begin{enumerate}
\def\labelenumi{\arabic{enumi}.}
\item
  母后大人事蹟之一,感受到文化衝擊的醫師節

  前陣子Y縣醫師節餐會的請帖寄到家中,上面載明可以免費攜眷一人,若繳六百塊就可以攜眷兩人。當時我其實有暗示母后說不要去,無奈的是他聽到大飯店宴席就變得超興高采烈,我實在說不過他,只好被迫攜母后跟父皇參加。

  結果到了現場,母后果然後悔了,大大後悔了,他完全想不到那是一個比應酬還應酬,比官僚還官僚的飯局。因為破事實在太多,我的中文能力不夠把它整理成良好的段落,所以只好浪費空間條列如下。

  \begin{itemize}
  \item
    先花40分鐘頒發服務滿三十年醫師的獎狀,之後才准上菜。母后因此在旁邊的麵包店買了麵包墊肚子。
  \item
    各個醫院的高層輪流到各桌敬酒,所以母后跟父皇也被迫跟我一起每五分鐘起立一次。
  \item
    我的直屬長官剛好坐跟我同一桌,母后跟父皇只好跟我一起陪笑。
  \item
    某政府高層帶了他的一團屬下來一起站在台上一邊搖政府宣傳標語一邊唱卡拉OK。(唯一欣慰的事情是他的屬下裡面有很鮮很鮮的小鮮肉{[}1{]}。
  \item
    然後Y縣醫師公會所有理監事陪著高層一起唱卡拉OK,屬下繼續搖標語。
  \item
    大家很想喝的最後一道菜 -
    雞湯,為了要等立委來致詞因此延遲了半小時還無法上菜,大部分人,包含我們都忍不住離席了。
  \item
    母后回到C市後決定用鹽酥雞填飽肚子。
  \end{itemize}

  母后跟父皇一輩子都待在吃的文化圈,對他們來講無論平常怎麼打官腔,吃的時候至少要吃得開心,所以很難想像有這麼不開心的宴席。好,現在他體會到了,未來所有醫療圈宴席他應該都不會吵著要跟了。
\item
  母后大人事蹟之二,清唱劇

  這個清唱劇是母后大人為了我還有父皇大人苦心安排的曝光機會,原本應該好好做,不過我實在沒有足夠的編曲實力可以撐起這種大場,也太過懶惰,所以最後他只好安排現成的歌曲編一編,然後讓我出一些以前寫過的現成素材,勉強做一個拼裝車清唱劇。

  如果對我的作品量有理解的人,應該會知道我沒很多現成素材可以用(懶惰所以寫很少),不過從電腦裡面苦心搜尋以後,居然還真的有三四首可以用,都是過去做配樂的時候被老闆淘汰的東西。如果來聽這場清唱劇的那些大人物知道這是場廢品再利用的回收劇,不知會做何感想。
\item
  母后大人事蹟之三,評鑑資料

  母后大人去參加某個老人教育的評鑑,為了要討好那些委員,必須把明明是教音樂這件事情包裝成預防失智的偉大教育,所以我就被迫生了一堆腦科學的名詞讓他上去唬爛,今天早上還花了三小時教會他怎麼講得滿口都是似是而非半對半錯的偽腦科學。我其實很擔心,那些程度很差的老人教育科系大學教授,會被他唬爛到真的相信學音樂可以防失智,學音樂可以變聰明。

  雖然說其實我也覺得學音樂可以變聰明就是了\ldots\ldots 但是那到底是我的自我洗腦,還是證據力真的足夠,這可能還要多多釐清。
\end{enumerate}

\hypertarget{ux6ce8ux91cb-comment-29}{%
\subsection{注釋 Comment}\label{ux6ce8ux91cb-comment-29}}

{[}1{]} 此名詞請見蒼白球日誌0030。

\hypertarget{ux9644ux9304-appendix-28}{%
\subsection{附錄 Appendix}\label{ux9644ux9304-appendix-28}}

\hypertarget{ux84bcux767dux7403ux65e5ux8a8c003720191104}{%
\section{蒼白球日誌0037(20191104)}\label{ux84bcux767dux7403ux65e5ux8a8c003720191104}}

\hypertarget{ux65e5ux671f-date-36}{%
\subsection{日期 Date}\label{ux65e5ux671f-date-36}}

\begin{itemize}
\tightlist
\item
  世界協調時間2019年(中華民國108年,令和1年)11月4日 / Unix 紀元 18204 日
  / 星期一 / 蒼白球紀元第37日
\item
  November 04, 2019 (UTC) / 18204 days since Unix Epoch / Monday /
  Globus Pallidum day 37
\item
  特殊註記:
\end{itemize}

\hypertarget{ux5e74ux9f61-age-36}{%
\subsection{年齡 Age}\label{ux5e74ux9f61-age-36}}

\begin{itemize}
\tightlist
\item
  33 years 6 months 12 days old / 2 years 0 months 23 days after
  acquiring ROC Surgical Pathology Licence
\item
  33 歲 6 個月 12 天 / 成為病理專科醫師 2 年 0 個月 23 天
\end{itemize}

\hypertarget{ux672cux6587-content-36}{%
\subsection{本文 Content}\label{ux672cux6587-content-36}}

\begin{enumerate}
\def\labelenumi{\arabic{enumi}.}
\item
  蒼子(2007)(已絕版)

  最近因為常常思考資料存續的問題,因此稍微盤點了一下手上持有的老紙本資料,過程中意外發現了這本被我放在書架一角,從未仔細翻閱的書。

  蒼子是我大學學長,把自己教物理家教、當醫學生跟長期交不到女朋友的經歷寫成了一整本的碎碎念,而且還自費出版,分送給一些親友。而當時我就是其中一個被視為親友的人,交到我手上的時候書的扉頁還蓋了他的實習醫學生職章,超級宇宙無敵中二{[}1{]}的。

  坦白講此人駕馭文法的能力算及格,但因為書中內容實在是太過碎碎念,所以以散文來講實在不太好看,當時會留著這本書,恐怕只是因為我暗戀了蒼子一小段時間{[}2{]}。然而,十二年後對於蒼子的感情早就退去,這本書卻因為其不好看的特質,意外地感覺重要了起來。畢竟,在這個社群網路的時代,醫學生跟家教老師的生活紀錄各種裝模作樣假惺惺,像這本書這種樸拙真實的紀錄,已經變得非常稀有了。而且,還紀錄了一個智慧型手機不普及的年代C市的面貌。而且,還是作者蓋章的絕版書。

  所以我老是鼓勵男友多寫一些甜點星球物語的事情,畢竟在世界協調時間2019年(中華民國108年,令和1年)看起來像碎碎念的東西,也許在2029年就是珍貴實錄了。
\item
  雜記:物價與其他{[}3{]}

  \begin{itemize}
  \tightlist
  \item
    認真練了一整個晚上的小提琴。好痛苦。好討厭拉琴。好後悔醫師節說要上台拉琴。
  \item
    然後為了練琴,臉皮很厚地把所有外院的報告都拖到超過期限,因為外院報告不會罰錢,也不算在考績裡面。真是個不老實工作的病理人。
  \item
    修小提琴弓2000元,買指揮棒與松香1000元,C院午餐72元(紫米飯,芥蘭,洋蔥豬柳,白菜,青花菜),「清心福全」普洱茶5元(使用line
    point20點{[}4{]})
  \end{itemize}
\end{enumerate}

\hypertarget{ux6ce8ux91cb-comment-30}{%
\subsection{注釋 Comment}\label{ux6ce8ux91cb-comment-30}}

{[}1{]}
中二病(日語:中二病、厨二病),簡稱中二,是源自日本的網路流行語,泛指一種自我認知心態,用以形容一些經常自以為是地活在自己世界或做出自我滿足的特別言行的人,是青春期特有的價值觀的總稱。雖然稱「病」,但和醫學上的「疾病」沒有任何關係。中文中接近中二病意象的字詞有年少輕狂、少不更事等,但都未能精準地描述中二病。

{[}2{]}
這個地方應該要註釋什麼叫做異男忘,但這名詞對我來說實在太過酸澀,不想解釋,請看不懂的人自行查詢別的來源。

{[}3{]} 新台幣計價。有關新台幣請參見蒼白球日誌0007。

{[}4{]} 某些信用卡消費後給予回饋的點數,可以在使用line
pay這種行動支付工具消費時視為現金抵用。

\hypertarget{ux9644ux9304-appendix-29}{%
\subsection{附錄 Appendix}\label{ux9644ux9304-appendix-29}}

\hypertarget{ux84bcux767dux7403ux65e5ux8a8c003820191105}{%
\section{蒼白球日誌0038(20191105)}\label{ux84bcux767dux7403ux65e5ux8a8c003820191105}}

\hypertarget{ux65e5ux671f-date-37}{%
\subsection{日期 Date}\label{ux65e5ux671f-date-37}}

\begin{itemize}
\tightlist
\item
  世界協調時間2019年(中華民國108年,令和1年)11月5日 / Unix 紀元 18205 日
  / 星期二 / 蒼白球紀元第38日
\item
  November 05, 2019 (UTC) / 18205 days since Unix Epoch / Tuesday /
  Globus Pallidum day 38
\end{itemize}

\hypertarget{ux5e74ux9f61-age-37}{%
\subsection{年齡 Age}\label{ux5e74ux9f61-age-37}}

\begin{itemize}
\tightlist
\item
  33 years 6 months 13 days old / 2 years 0 months 24 days after
  acquiring ROC Surgical Pathology Licence
\item
  33 歲 6 個月 13 天 / 成為病理專科醫師 2 年 0 個月 24 天
\end{itemize}

\hypertarget{ux672cux6587-content-37}{%
\subsection{本文 Content}\label{ux672cux6587-content-37}}

\begin{enumerate}
\def\labelenumi{\arabic{enumi}.}
\item
  又欠了男友還有diss中一點

  今天因為各種壓力的關係有點發病,對著男友還有diss中講一堆有的沒有的,倒各種情緒垃圾,回頭想想,覺得又欠了一點人情債。

  但反正這債我是永遠還不起的,從他們收留我流浪中的靈魂開始就欠下巨額了吧。
\item
  又把片子放著不寫了

  因為要提早衝回家剪下個禮拜一要表演的影片,然後練琴。說到我之所以今天可以準時練琴,也算是欠了人情債。表演前我決定要把琴弓修一修,樂器行原本無法一天給我,結果拜託了認識的交響樂團耆老,才硬逼著把弓提前修好\ldots\ldots{}
\item
  文法文法文法

  今天狀況實在太不好,且趕著要作影片,顧不得文法了,所以大家就原諒我前兩段寫得亂七八糟吧。
\end{enumerate}

\hypertarget{ux6ce8ux91cb-comment-31}{%
\subsection{注釋 Comment}\label{ux6ce8ux91cb-comment-31}}

狀況不好無法註釋

\hypertarget{ux9644ux9304-appendix-30}{%
\subsection{附錄 Appendix}\label{ux9644ux9304-appendix-30}}

\hypertarget{ux84bcux767dux7403ux65e5ux8a8c003920191106}{%
\section{蒼白球日誌0039(20191106)}\label{ux84bcux767dux7403ux65e5ux8a8c003920191106}}

\hypertarget{ux65e5ux671f-date-38}{%
\subsection{日期 Date}\label{ux65e5ux671f-date-38}}

\begin{itemize}
\tightlist
\item
  世界協調時間2019年(中華民國108年,令和1年)11月6日 / Unix 紀元 18206 日
  / 星期三 / 蒼白球紀元第39日
\item
  November 06, 2019 (UTC) / 18206 days since Unix Epoch / Wednesday /
  Globus Pallidum day 39
\item
  特殊註記:
\end{itemize}

\hypertarget{ux5e74ux9f61-age-38}{%
\subsection{年齡 Age}\label{ux5e74ux9f61-age-38}}

\begin{itemize}
\tightlist
\item
  33 years 6 months 14 days old / 2 years 0 months 25 days after
  acquiring ROC Surgical Pathology Licence
\item
  33 歲 6 個月 14 天 / 成為病理專科醫師 2 年 0 個月 25 天
\end{itemize}

\hypertarget{ux672cux6587-content-38}{%
\subsection{本文 Content}\label{ux672cux6587-content-38}}

\begin{enumerate}
\def\labelenumi{\arabic{enumi}.}
\item
  決定把日誌檔案打包每月一個檔案{[}1{]}

  多個小檔案視覺上非常混亂,而且在儲存媒體上面的讀寫性也不好,因此我決定每月打包一次日誌,把當月所有日期的日誌(已經都是用Markdown{[}2{]}寫的)先用程式合併成一個大的Markdown檔案,再輸出成html、pdf與純文字,這三種我覺得可以存續很久的格式。目前除了pdf輸出還沒完成以外,其他程式都已經寫完了,只要python還穩定存在的一天,每月打包應該都很輕鬆才對。
\item
  疲累

  做了一個晚上的影片剪輯之後,整個人的靈魂完全被榨乾了,不只寫片片的時候不專心,連上健身教練課都打瞌睡。影片剪輯真是太可怕了!以後這種工作我要花錢找人做。
\item
  食記{[}3{]}

  在C院跟健身房之間,有一家裝潢很粗糙,生意看起來很差的冰店,他的外觀看起來糟糕到,十年來我經過這家店數百次,沒有一次想要進去吃。

  可是今天除外。因為整個身心的狀態很類似宿醉,很想隨便補個甜食來刺激腦部,所以就進去吃了。「五種冰」55元,內含芋園、綠豆、紅豆、仙草、薏仁,所有的料都煮得非常的軟爛,而且相當甜膩,果然是家並不會有好口碑的店。但是,但是,以2019年的中華民國,55元的冰來講,他給我的分量可真有誠意啊,所有料都是一坨一坨的加的,尤其是該店在網路上評價最差的紅豆給得最慷慨,感覺好像在說:「紅豆賣不出去就一口氣倒給你吃吧,可憐人」。

  而且一口氣倒給我吃的,還不只紅豆。在老闆把冰端給我的時候,還同時附上了第二碗東西,同時輕柔地說「這個就請你吃了」。

  是一碗湯圓加布丁豆花。我看到這個畫面其實心裡面狂笑不止,想說「會把湯圓連著上面沾著的白色糯米黏液,一起澆在滿滿洋菜質感的布丁豆花上面讓他看起來像餿水的店,實在難怪生意不好」,但是想到老闆應該是因為我長得一副魯蛇臉加可憐小動物表情,才好心餵我吃這麼多東西,就決定要賣力地把滿滿的料還有詭異的湯圓布丁豆花吃到一點都不剩。超級滿足,好久沒有吃甜食吃到肚子脹了。魯蛇臉有時也是很好用的。

  滿肚子糖以後去一中街補一些鹽分稀釋。雖然滿街都是滷味,但不是我想要的鹽分,炸菇才是。小份綜合菇45,菇跟炸粉的味道可以,但鹽加得有點過多。無所謂了,反正我是來稀釋魯蛇湯圓的。

  最後以「花茶大師」桑菊薄荷茶50元作結。平常喝很解膩,但是在大甜大鹹之後喝桑菊居然有點噁心感,可能只有茶葉才敵得過重口味吧。
\end{enumerate}

\hypertarget{ux6ce8ux91cb-comment-32}{%
\subsection{注釋 Comment}\label{ux6ce8ux91cb-comment-32}}

{[}1{]} 指的是電腦檔案。有關這個時代的電腦請見蒼白球日誌0008,
0010及0011。

{[}2{]} 此檔案格式見蒼白球日誌0014。

{[}3{]} 新台幣計價。有關新台幣請參見蒼白球日誌0007。

\hypertarget{ux9644ux9304-appendix-31}{%
\subsection{附錄 Appendix}\label{ux9644ux9304-appendix-31}}

\hypertarget{ux84bcux767dux7403ux65e5ux8a8c004020191107}{%
\section{蒼白球日誌0040(20191107)}\label{ux84bcux767dux7403ux65e5ux8a8c004020191107}}

\hypertarget{ux65e5ux671f-date-39}{%
\subsection{日期 Date}\label{ux65e5ux671f-date-39}}

\begin{itemize}
\tightlist
\item
  世界協調時間2019年(中華民國108年,令和1年)11月7日 / Unix 紀元 18207 日
  / 星期四 / 蒼白球紀元第40日
\item
  November 07, 2019 (UTC) / 18207 days since Unix Epoch / Thursday /
  Globus Pallidum day 40
\item
  特殊註記:
\end{itemize}

\hypertarget{ux5e74ux9f61-age-39}{%
\subsection{年齡 Age}\label{ux5e74ux9f61-age-39}}

\begin{itemize}
\tightlist
\item
  33 years 6 months 15 days old / 2 years 0 months 26 days after
  acquiring ROC Surgical Pathology Licence
\item
  33 歲 6 個月 15 天 / 成為病理專科醫師 2 年 0 個月 26 天
\end{itemize}

\hypertarget{ux672cux6587-content-39}{%
\subsection{本文 Content}\label{ux672cux6587-content-39}}

\begin{enumerate}
\def\labelenumi{\arabic{enumi}.}
\item
  研究重啟的第一天

  兩場演出的籌備稍微告一段落以後,今天終於可以開始做一些研究的事情。雖然說得好像很正經在做什麼偉大的事業,但內容其實很無聊,就是打開電腦,標示一點正常腦部的切片影像,給興大的研究生跑程式而已。

  有關機器學習的研究,在病理醫師這一端的工作就是這麼枯燥乏味,精采程度不如軟體工程方面的演算法,更不如硬體方面的機械設計。雖然這樣說,但病理醫師的資料輸入依然是整個研究不可或缺的一部分,畢竟這不像一般的圖像辨識,可以雇用普通大眾,病理切片只能由病理醫師判讀。

  醫師軟體硬體金三角,三足鼎立,缺一不可,身為三足其中之一,或許我也該有點榮譽感,不該覺得自己都在打雜。
\item
  練小提琴

  換過弓毛,然後讓大師開光過的琴弓拉起來有如神助,覺得自己的琴藝瞬間進步了,下周一可以好好去表演了。這個月該請大師吃頓好料才對。
\item
  增肥

  發現自己減肥成功以後,連續大吃大喝了好幾周,結果今天照鏡子發現臉變圓了,而且痘痘復發了。果然要復胖是很容易的呢,下周開始吃低卡餐跟跑步。
\end{enumerate}

\hypertarget{ux6ce8ux91cb-comment-33}{%
\subsection{注釋 Comment}\label{ux6ce8ux91cb-comment-33}}

懶得注

\hypertarget{ux9644ux9304-appendix-32}{%
\subsection{附錄 Appendix}\label{ux9644ux9304-appendix-32}}

\hypertarget{ux84bcux767dux7403ux65e5ux8a8c004120191108}{%
\section{蒼白球日誌0041(20191108)}\label{ux84bcux767dux7403ux65e5ux8a8c004120191108}}

\hypertarget{ux65e5ux671f-date-40}{%
\subsection{日期 Date}\label{ux65e5ux671f-date-40}}

\begin{itemize}
\tightlist
\item
  世界協調時間2019年(中華民國108年,令和1年)11月8日 / Unix 紀元 18208 日
  / 星期五 / 蒼白球紀元第41日
\item
  November 08, 2019 (UTC) / 18208 days since Unix Epoch / Friday /
  Globus Pallidum day 41
\item
  特殊註記:
\end{itemize}

\hypertarget{ux5e74ux9f61-age-40}{%
\subsection{年齡 Age}\label{ux5e74ux9f61-age-40}}

\begin{itemize}
\tightlist
\item
  33 years 6 months 16 days old / 2 years 0 months 27 days after
  acquiring ROC Surgical Pathology Licence
\item
  33 歲 6 個月 16 天 / 成為病理專科醫師 2 年 0 個月 27 天
\end{itemize}

\hypertarget{ux672cux6587-content-40}{%
\subsection{本文 Content}\label{ux672cux6587-content-40}}

\begin{enumerate}
\def\labelenumi{\arabic{enumi}.}
\item
  食記{[}1{]}

  中午從H大坐公車回家裡拿小提琴的過程中,需要從地方法院轉車。由於下一班車要12分鐘才到,因此在附近找看看有沒有什麼東西可以填肚子當午餐吃。放眼望去,發現這個地方居然沒有便利商店,正在絕望的時候,突然聞到不起眼的小店冒出好香好香的麵包味道,忍不住就走進去了。

  結果居然是大名店「洪瑞珍餅店」的其中一家分店。也開得太低調了,莫非老台中的貴族店都這麼樸實無華嗎?

  全麥火腿三明治30元,坦白講實在不便宜,但名店就是名店,咬下去的口感就是比平常的火腿三明治豐富許多。全中華民國國民都願意從台中訂購這家的三明治,真的是有原因的。配一瓶瑞穗鮮奶32元。

  晚餐:「丼脈」(因為老闆似乎有醫療背景,所以念法應該是「動脈」,取artery之意)薑汁牛丼,「李家紅茶冰」麥香紅茶,不寫價格因為C院請客。
\item
  與學弟妹還有母后一起團練後

  因為有一位學妹需要拿大提琴,所有人都需要拿譜架,所以母后就自告奮勇,11月11日的演出要開車載學弟妹去演出場地。我一方面覺得感謝他願意幫忙這場演出,另一方面又覺得,三十幾歲了這種事情還要老母幫忙,真的在學弟妹面前很沒面子。

  或許只能拿男友曾經講過的話安慰自己:「反正很多人都是靠著別人的善意才得以生存的」。
\item
  賀!

  睡超飽!覺得智商變成了1.5倍!
\end{enumerate}

\hypertarget{ux6ce8ux91cb-comment-34}{%
\subsection{注釋 Comment}\label{ux6ce8ux91cb-comment-34}}

{[}1{]} 新台幣計價。有關新台幣請參見蒼白球日誌0007。

{[}2{]}
引述店家簡介:洪瑞珍餅店自由店成立於民國70年,負責人洪幸雄13歲就投入烘培業,一路從彰化北斗洪瑞珍創始店,
二林店,
台中中山店一直到現在已逾70歲仍然堅守在洪瑞珍烘培師的領域,一步一腳印腳踏實地的做好烘培者的角色。「自由店」的產品都很樸實不走花俏但很堅持產品的內涵,有如負責人的個性一樣,堅持用有品牌有信譽的原料也是自由店的堅持,產品所需的內餡也都依最傳統費時的方式來蒸煮自製這是保護消費者最基本的態度。

\hypertarget{ux9644ux9304-appendix-33}{%
\subsection{附錄 Appendix}\label{ux9644ux9304-appendix-33}}

\hypertarget{ux84bcux767dux7403ux65e5ux8a8c004220191109}{%
\section{蒼白球日誌0042(20191109)}\label{ux84bcux767dux7403ux65e5ux8a8c004220191109}}

\hypertarget{ux65e5ux671f-date-41}{%
\subsection{日期 Date}\label{ux65e5ux671f-date-41}}

\begin{itemize}
\tightlist
\item
  世界協調時間2019年(中華民國108年,令和1年)11月9日 / Unix 紀元 18209 日
  / 星期六 / 蒼白球紀元第42日
\item
  November 09, 2019 (UTC) / 18209 days since Unix Epoch / Saturday /
  Globus Pallidum day 42
\end{itemize}

\hypertarget{ux5e74ux9f61-age-41}{%
\subsection{年齡 Age}\label{ux5e74ux9f61-age-41}}

\begin{itemize}
\tightlist
\item
  33 years 6 months 17 days old / 2 years 0 months 28 days after
  acquiring ROC Surgical Pathology Licence
\item
  33 歲 6 個月 17 天 / 成為病理專科醫師 2 年 0 個月 28 天
\end{itemize}

\hypertarget{ux672cux6587-content-41}{%
\subsection{本文 Content}\label{ux672cux6587-content-41}}

\begin{enumerate}
\def\labelenumi{\arabic{enumi}.}
\item
  數學與對男友情緒勒索

  因為某個契機,突然發現自己並沒有高中畢業應有的基本數學素養。在質疑自己的中華民國公民資格的同時,玻璃心{[}1{]}就碎掉了,於是這兩天強迫男友聽了一堆有的沒有的屁話。

  在情緒稍微平復一點之後,回頭再看一次那些話,驚覺我把自信心問題跟有的沒有的壓力都發洩在別人身上,完全就是一種很糟糕的情緒勒索。不過,做這個動作真是很舒壓,做完心情就平復了一半。另外一半則是開始寫這段文字的時候開始平復的,因為知道寫下來會有人看,舒壓的速度又更快了。我畢竟是一個非常怕孤單的人,所以用這種不太合理的方式尋求陪伴跟傾聽,請大家見諒。

  再次感謝男友跟其他讀者。然後我該來讀微積分跟統計了(棒棒的男友說很樂意教我),生活在這世界上總是要有一點數感比較好。
\item
  跟母后一起去教非洲鼓{[}2{]}

  今天的課程很奇妙,是某個連鎖餐飲店的員工訓練,原因是「老闆希望員工有一些人文素養」。其實人文素養這個東西是什麼,本身就是一個很困難的問題,但是因為這個名詞太炫麗了,於是就成為了中華民國國民做各種自認很有氣質的假掰{[}3{]}事的藉口。至於人文素養裡面有什麼,好像就沒有人在乎了。

  結果這個,嗯,因為老闆認為學音樂增進氣質所以開辦的非洲鼓訓練,最後招到了一些原本就有學音樂的員工來參加,好像做了什麼但是又等於什麼都沒做。現實世界總是比小說還荒謬。

  嗯但是每天寫廢話,這個月底還想要把這些廢話印成書{[}4{]}的我,好像也沒什麼理由批評別人荒謬。
\item
  雜記:物價與其他{[}5{]}

  \begin{itemize}
  \tightlist
  \item
    「麥味登」青醬燻雞義大利麵80元(大地雷!千萬不要去麥味登吃義大利麵!),「李記紅茶冰」25元
  \item
    晚餐回濁水溪老家陪阿嬤跟姑姑們吃祈福的桌菜,所以沒花自己的錢,但是過程很折磨人,因為所有長輩都對現況還有未來大局有非常悲觀的看法,且直接認為選了民進黨政府上台的錯。這會很折磨人的原因,一方面是由於害怕共產黨,所以不管民進黨政府做了多少髒事,我還是偏向支持它,立場差異是第一個不開心的地方。另一方面,我自承對政府運作的了解有限,不可能理解這些複雜的議題,所以沒辦法像某些綠營的狂信者那樣,用膝反射回應膝反射,只能沉默,無話可說是第二個不開心的地方。而在沉默與沉默與壓抑與壓抑之間,唯一的救贖或許只剩下蒼白球日誌。寫下來的事情才能夠釋放,寫下來的事情才能夠遺忘,寫下來的事情才能夠回想。
  \end{itemize}
\end{enumerate}

\hypertarget{ux6ce8ux91cb-comment-35}{%
\subsection{注釋 Comment}\label{ux6ce8ux91cb-comment-35}}

{[}1{]}
玻璃易碎,玻璃心是形容人自尊心極容易受損、心靈異常脆弱、極容易心碎的意思。

{[}2{]} 非洲鼓與我們的非洲鼓課程,請見蒼白球日誌0005。

{[}3{]} 原自閩南語「膣屄」,指一個人思想或行為很做作。

{[}4{]}
暫名「蒼白球日誌2019秋(0001-0063)」,原本想要找C市的小量印刷店印,離工作地點近比較方便,但是看到P市的樺舍有提供一本試印的服務就心動了,到時應該會找樺舍印吧。

{[}5{]} 新台幣計價。有關新台幣請參見蒼白球日誌0007。

\hypertarget{ux9644ux9304-appendix-34}{%
\subsection{附錄 Appendix}\label{ux9644ux9304-appendix-34}}

\hypertarget{ux84bcux767dux7403ux65e5ux8a8c004320191110}{%
\section{蒼白球日誌0043(20191110)}\label{ux84bcux767dux7403ux65e5ux8a8c004320191110}}

\hypertarget{ux65e5ux671f-date-42}{%
\subsection{日期 Date}\label{ux65e5ux671f-date-42}}

\begin{itemize}
\tightlist
\item
  世界協調時間2019年(中華民國108年,令和1年)11月10日 / Unix 紀元 18210
  日 / 星期日 / 蒼白球紀元第43日
\item
  November 10, 2019 (UTC) / 18210 days since Unix Epoch / Sunday /
  Globus Pallidum day 43
\end{itemize}

\hypertarget{ux5e74ux9f61-age-42}{%
\subsection{年齡 Age}\label{ux5e74ux9f61-age-42}}

\begin{itemize}
\tightlist
\item
  33 years 6 months 18 days old / 2 years 0 months 29 days after
  acquiring ROC Surgical Pathology Licence
\item
  33 歲 6 個月 18 天 / 成為病理專科醫師 2 年 0 個月 29 天
\end{itemize}

\hypertarget{ux672cux6587-content-42}{%
\subsection{本文 Content}\label{ux672cux6587-content-42}}

\begin{enumerate}
\def\labelenumi{\arabic{enumi}.}
\item
  音樂欣賞簡報

  母后下周六日要去教一堂七小時的課程,內容有音樂欣賞以及非洲鼓,我今天待在家裡幫他做教材。

  之所以非得我做不可,其中一個原因是母后的課程通常需要很多影片輔助,因此必須去Youtube上面搜尋適合的影片{[}1{]},並且盜拷下來。這個操作對於母后這個年齡的人來說相當吃力,所以都是我在幫忙。另外還有一點就是,有很多有的沒有的創意也只有我可以想得出來。

  簡而言之,母后近年的教學事業其實都是我在背後輔助。其實滿累的。
\item
  文學與政治

  一個時代的文字環境總是與政權息息相關,這無可厚非,文字本來就是一種很政治的工具,但是最近包含朱宥勳激怒駱以軍事件{[}2{]}在內的很多事情,讓我覺得繁體中文文字域的政權味有點太過份了。有點太難忍受了。當然,這個情形當然是本來就很有可能發生的沒錯,畢竟社群媒體能夠無限放大文字的力量,必然造成政黨被迫對這個文字域進行操作,造成人們的文字域裡面逐漸充滿了飽和的,單一的,政黨政治的元素。

  但是仔細想想又很怪。

  我們有人類歷史以來最豐富的文字發表環境,最低的文字發表成本,結果文字域裡面卻填滿了很無聊單一的東西,這實在很弔詭又很討厭。因此我主張人們應該用各式各樣不同的語言文字抵抗政黨的汙染。像韓少功所主張的一樣,每個人都要編一套屬於自己的辭典。
\end{enumerate}

\hypertarget{ux6ce8ux91cb-comment-36}{%
\subsection{注釋 Comment}\label{ux6ce8ux91cb-comment-36}}

{[}1{]}
YouTube是源自美國的影片分享網站,讓使用者上傳、觀看、分享及評論影片。公司於2005年2月14日註冊,網站的口號為「Broadcast
Yourself」,網站的標誌意念來自早期電視顯示器。
目前尚無官方的中文譯名,較為廣泛使用的俗稱有油管、水管、你管等。

{[}2{]}
朱宥勳(1988年1月4日-),臺灣桃園人,小說家、文化評論者、專欄作家,國立清華大學人文社會學系學士(主修社會學、歷史)、國立清華大學台灣文學研究所碩士。駱以軍(1967年3月29日-),台北縣(今新北市)人,籍貫安徽省無為縣,臺灣專職作家。2019年11月5日,朱宥勳在個人臉書公開發表駱以軍新長篇小說《明朝》的書評,因書評的政黨色彩而激怒駱以軍,從而掀起了一場社群媒體論戰。

\hypertarget{ux9644ux9304-appendix-35}{%
\subsection{附錄 Appendix}\label{ux9644ux9304-appendix-35}}

\hypertarget{ux84bcux767dux7403ux65e5ux8a8c004420191111}{%
\section{蒼白球日誌0044(20191111)}\label{ux84bcux767dux7403ux65e5ux8a8c004420191111}}

\hypertarget{ux65e5ux671f-date-43}{%
\subsection{日期 Date}\label{ux65e5ux671f-date-43}}

\begin{itemize}
\tightlist
\item
  世界協調時間2019年(中華民國108年,令和1年)11月11日 / Unix 紀元 18211
  日 / 星期一 / 蒼白球紀元第44日
\item
  November 11, 2019 (UTC) / 18211 days since Unix Epoch / Monday /
  Globus Pallidum day 44
\end{itemize}

\hypertarget{ux5e74ux9f61-age-43}{%
\subsection{年齡 Age}\label{ux5e74ux9f61-age-43}}

\begin{itemize}
\tightlist
\item
  33 years 6 months 19 days old / 2 years 0 months 30 days after
  acquiring ROC Surgical Pathology Licence
\item
  33 歲 6 個月 19 天 / 成為病理專科醫師 2 年 0 個月 30 天
\end{itemize}

\hypertarget{ux672cux6587-content-43}{%
\subsection{本文 Content}\label{ux672cux6587-content-43}}

\begin{enumerate}
\def\labelenumi{\arabic{enumi}.}
\item
  醫師節院內宴會,與醫學生還有母后進行了一場成功的鋼琴-弦樂重奏演出

  之後演出人員坐在同一桌吃這場宴會。如果你以為跟年輕的帥哥美女同桌吃飯是享受,那就大錯特錯了。這是一種最糟糕的應酬,惡夢的宴席。由於主治醫師有權力打醫學生的成績,且未來應徵住院醫師的時候必定會過主治醫師這一關,醫學生在這場飯局中用各種業務般油滑的話術對我各種裝熟,刺探醫院的各種秘辛,還要求合照,甚至是要Instagram{[}1{]}帳號,簡直像在談生意一樣。

  不舒服的除了那些很油很刻意的話術以外,還有那個權力不對等的氣氛。被當成需要加以奉承的長官,對我來講不僅非常怪,而且更嚴重的是,覺得變老了。幸好我下禮拜就擺脫這幾個人了,如果長期要跟醫學生一起合奏的話,我一定會被這些業務嘴煩到瘋掉。

  然後要Instagram的要求被我回絕了。我不想要跟醫學生有太多私交。
\item
  雜記:物價與其他{[}2{]}

  C院午餐72元(紫米飯,洋蔥豬柳,地瓜葉,青花菜,西洋芹),「泰山」氣泡水兩瓶39元(第二瓶十元),醫師節晚宴免費,用C院禮券買的一堆飲料104元(禮券100元,補4元用悠遊卡付)。
\end{enumerate}

\hypertarget{ux6ce8ux91cb-comment-37}{%
\subsection{注釋 Comment}\label{ux6ce8ux91cb-comment-37}}

{[}1{]}
Instagram是Facebook公司旗下一款免費提供線上圖片及視訊分享的社交應用軟體,於2010年10月發布。它可以讓用戶用智慧型手機拍下相片後再將不同的濾鏡效果添加到相片上,然後分享到Facebook、Twitter、Tumblr及Flickr等社群網路服務、或是Instagram的伺服器上。

{[}2{]} 新台幣計價。有關新台幣請參見蒼白球日誌0007。

\hypertarget{ux9644ux9304-appendix-36}{%
\subsection{附錄 Appendix}\label{ux9644ux9304-appendix-36}}

\hypertarget{ux84bcux767dux7403ux65e5ux8a8c004520191112}{%
\section{蒼白球日誌0045(20191112)}\label{ux84bcux767dux7403ux65e5ux8a8c004520191112}}

\hypertarget{ux65e5ux671f-date-44}{%
\subsection{日期 Date}\label{ux65e5ux671f-date-44}}

\begin{itemize}
\tightlist
\item
  世界協調時間2019年(中華民國108年,令和1年)11月12日 / Unix 紀元 18212
  日 / 星期二 / 蒼白球紀元第45日
\item
  November 12, 2019 (UTC) / 18212 days since Unix Epoch / Tuesday /
  Globus Pallidum day 45
\item
  特殊註記:
\end{itemize}

\hypertarget{ux5e74ux9f61-age-44}{%
\subsection{年齡 Age}\label{ux5e74ux9f61-age-44}}

\begin{itemize}
\tightlist
\item
  33 years 6 months 20 days old / 2 years 1 months 0 days after
  acquiring ROC Surgical Pathology Licence
\item
  33 歲 6 個月 20 天 / 成為病理專科醫師 2 年 1 個月 0 天
\end{itemize}

\hypertarget{ux672cux6587-content-44}{%
\subsection{本文 Content}\label{ux672cux6587-content-44}}

\begin{enumerate}
\def\labelenumi{\arabic{enumi}.}
\item
  一罐濃縮時間

  雖然我也知道年過三十要保持青春是不可能的,但是升主治醫師這兩年來,我的外貌跟心智實在是老化得太過頭了。診斷的壓力、研究的壓力、科內的種種變化,讓人好像喝下了一罐濃縮時間一樣,不斷往衰敗的方向前進。

  更可怕的是,這不是那種在歲月的推進中不知不覺變老,而是每一天每一天都覺得被時間追著跑的變老,不只喝下濃縮時間,喝下的過程還每一刻都在舌尖感覺到濃縮時間的苦味。而這就是我對這個日誌如此執著的原因。被灌食這麼苦的濃縮時間,總要結晶出什麼時間的舍利子,不然豈不是太虧了嗎?
\item
  母后送我的高檔防曬乳沒了,於是我跑去添購

  可能是因為我身上散發出的魯味{[}1{]}太濃吧,一到Elisabeth
  Arden的專櫃前面,專櫃店員就很明顯地出現冷淡的表情,完全不想理我。直到我提出要求說要買那管黃色的防曬乳,才吐出一句淡淡的「是幫別人買的嗎?」她心中想到的故事想必是以下這款:

  A君,33歲,幾年前花了家裡的積蓄勉強念到研究所畢業以後,因為體格不好做不了勞力工作,只好為了養活老婆小孩到處打不用體力的工。有一天老婆突然吵說他想要高檔保養品,只好硬擠出了一點錢,走到了從來不敢接近的專櫃前面\ldots\ldots{}

  他絕對想不到我是情形截然不同的33歲B君。算了。

  當我說出是我自己要用的時候,店員才露出了有點詫異的表情,然後有點錯愕地幫我結帳了。我以後可能需要時時霸氣地宣告「林北是主治醫師」才能夠避免被當成魯蛇吧\ldots\ldots 但那又很怪。
\item
  雜記:物價與其他{[}2{]}

  \begin{itemize}
  \tightlist
  \item
    Arden防曬乳,1300元。
  \item
    Tunemaker的一堆保養品,2072元,購於康是美(原價好像2350左右,結果店員因為超過兩千可以免費辦會員卡,就幫我辦了一張,在一路發動許多優惠之後不只折價到2072,還多帶了一罐保養品回家(因為EGF買一送一))
  \item
    C院午餐,72元
    (紫米飯,排骨,白花菜,西洋芹,杏鮑菇),一中街某無名店面的腸粉100元(店長看起來是普通的中年婦女,陪小孩在小小舊舊的店裡面做功課,然後小孩還耍脾氣,媽媽一邊安撫小孩一邊蒸我的腸粉,看起來很魯。但這說不定只是錯覺,這店在一中街精華地帶欸,他搞不好賺很多錢。有很多看起來魯的東西不一定真魯,就像我一樣。),「出櫃
    」蕎麥冰茶20元。
  \end{itemize}
\end{enumerate}

\hypertarget{ux6ce8ux91cb-comment-38}{%
\subsection{注釋 Comment}\label{ux6ce8ux91cb-comment-38}}

{[}1{]}
魯蛇(英語:loser),又稱魯者男、擼蛇、廢青、輸家男等,是大部分東亞地區網絡的一種諷刺語,意即「人生的失敗者」,最早在1993年由韓國匿名網民創設,在網絡作為隱語流通,2012年左右開始有華人網友使用此用法,於是逐漸在華人地區流行。魯味,即魯蛇的味道。

{[}2{]} 新台幣計價。有關新台幣請參見蒼白球日誌0007。

{[}3{]}
康是美藥妝店,簡稱康是美,是台灣一家連鎖藥妝店。截至2018年8月,已開設402家直營門市,在中國大陸亦有店鋪經營。
在台灣的主要競爭對手為屈臣氏。
2018年康是美營業額110.27億新臺幣,盈利2.9億新臺幣,總資產43.98億,負債30.31億,淨資產13.67億新臺幣。

\hypertarget{ux9644ux9304-appendix-37}{%
\subsection{附錄 Appendix}\label{ux9644ux9304-appendix-37}}

\hypertarget{ux84bcux767dux7403ux65e5ux8a8c004620191113}{%
\section{蒼白球日誌0046(20191113)}\label{ux84bcux767dux7403ux65e5ux8a8c004620191113}}

\hypertarget{ux65e5ux671f-date-45}{%
\subsection{日期 Date}\label{ux65e5ux671f-date-45}}

\begin{itemize}
\tightlist
\item
  世界協調時間2019年(中華民國108年,令和1年)11月13日 / Unix 紀元 18213
  日 / 星期三 / 蒼白球紀元第46日
\item
  November 13, 2019 (UTC) / 18213 days since Unix Epoch / Wednesday /
  Globus Pallidum day 46
\item
  特殊註記:
\end{itemize}

\hypertarget{ux5e74ux9f61-age-45}{%
\subsection{年齡 Age}\label{ux5e74ux9f61-age-45}}

\begin{itemize}
\tightlist
\item
  33 years 6 months 21 days old / 2 years 1 months 1 days after
  acquiring ROC Surgical Pathology Licence
\item
  33 歲 6 個月 21 天 / 成為病理專科醫師 2 年 1 個月 1 天
\end{itemize}

\hypertarget{ux672cux6587-content-45}{%
\subsection{本文 Content}\label{ux672cux6587-content-45}}

\begin{enumerate}
\def\labelenumi{\arabic{enumi}.}
\item
  「明朝」{[}1{]}

  在買下紹中的「在流放地」同時,我也下訂了「明朝」這本爭議之作。翻開前幾節的時候其實非常困惑,不是說這是小說嗎?怎麼會結構如此鬆散,幾乎沒有在正經說故事呢?而且,不時就跳到作者對古今文化的評論,非常出戲。

  直到我開始利用零碎時間把這本書跳著看以後,驚覺,啊,這本本來就不是小說。他其實是藉由一個科幻小說設定的軀殼,寫各種文化評論跟台灣社會的隨筆。所以我試著直接把那些看起來很不自然的情節拋在腦後,讀剩下的散文部分在腦中浮現,意外地覺得讀起來還算有意思。難怪某警犬會說「我覺得這本某個程度很適合你的調性」「讀不完但有可讀性」,某醫學生會說「我覺得你可以讀讀看」。

  這本坦白講絕對不會成為名著,但應該是有它的價值存在的。
\item
  與男友炫耀論文被引用

  講的時候放不下矜持,一直說什麼哎呀其實很意外人家寫大作會引用到這麼爛的期刊,什麼其實這個也沒啥創見只是剛好印證了某大型機構的資料,等等謙抑的說詞。炫耀完以後才驚覺,其實心裡根本就是爽翻天,幹嘛還要憋著,講一堆那麼收斂的說詞呢?

  應該要放聲大笑才對。哇哈哈哈哈哈,哇哈哈哈哈哈,哇哈哈哈哈哈,我的論文被Cell{[}2{]}上面的文章引用了!超爽的!超爽的!超爽的!
\item
  雜記:物價與其他{[}3{]}

  \begin{itemize}
  \tightlist
  \item
    C院午餐72元(紫米飯,排骨、青白混合花菜、菠菜、然後忘記一種菜),「全家便利商店」腿排便當44元(原價62元,友善食光7折{[}4{]}),「雀巢」黑糖奶茶30元(上健身教練克前補一點糖),十顆抗組織胺75元,「清心福全」普洱25元
  \end{itemize}
\end{enumerate}

\hypertarget{ux6ce8ux91cb-comment-39}{%
\subsection{注釋 Comment}\label{ux6ce8ux91cb-comment-39}}

{[}1{]} 指駱以軍的書。有關此書與其掀起的爭議,請見蒼白球日誌0043。

{[}2{]}
《細胞》為一份同行評審科學期刊,主要發表生命科學領域中的最新研究發現。《細胞》刊登過許多重大的生命科學研究進展,與《自然》和《科學》並列,是全世界最權威的學術雜誌之一。

{[}3{]} 新台幣計價。有關新台幣請參見蒼白球日誌0007。

{[}4{]} 快要過保存期限的熟食打折販賣

\hypertarget{ux9644ux9304-appendix-38}{%
\subsection{附錄 Appendix}\label{ux9644ux9304-appendix-38}}

\begin{verbatim}
圖:2019/11/13台中中山堂旁的滿月 https://imgur.com/gallery/WPl9ww3
註:難得台中的空氣乾淨到能夠看到這麼亮的滿月,不禁覺得充滿希望,讓人想起高橋禮子在經典歌曲METHOD_REPLEKIA/.中寫下的詩句:
xA rre wArAmA maen a.u.k. zess titia/. (黑暗中皎潔光亮的滿月升起,顯現了聖潔與公義)
xE rre hAkAtt nafan ouwua siann arhou/. (月光放出千萬個希望,照耀人間)
xA rre sEnEkk mirie, ag hEmmrA eje/. (而心之歌就如月光般溫暖光亮)
\end{verbatim}

\hypertarget{ux84bcux767dux7403ux65e5ux8a8c004720191114}{%
\section{蒼白球日誌0047(20191114)}\label{ux84bcux767dux7403ux65e5ux8a8c004720191114}}

\hypertarget{ux65e5ux671f-date-46}{%
\subsection{日期 Date}\label{ux65e5ux671f-date-46}}

\begin{itemize}
\tightlist
\item
  世界協調時間2019年(中華民國108年,令和1年)11月14日 / Unix 紀元 18214
  日 / 星期四 / 蒼白球紀元第47日
\item
  November 14, 2019 (UTC) / 18214 days since Unix Epoch / Thursday /
  Globus Pallidum day 47
\end{itemize}

\hypertarget{ux5e74ux9f61-age-46}{%
\subsection{年齡 Age}\label{ux5e74ux9f61-age-46}}

\begin{itemize}
\tightlist
\item
  33 years 6 months 22 days old / 2 years 1 months 2 days after
  acquiring ROC Surgical Pathology Licence
\item
  33 歲 6 個月 22 天 / 成為病理專科醫師 2 年 1 個月 2 天
\end{itemize}

\hypertarget{ux672cux6587-content-46}{%
\subsection{本文 Content}\label{ux672cux6587-content-46}}

\begin{enumerate}
\def\labelenumi{\arabic{enumi}.}
\item
  路上遇到曾經跟我們科關係密切的院內資深學者J博士

  她驚訝於我在一兩個月內變得非常憔悴的面容,不禁問:「發生什麼事了?」

  於是我開始訴說起再研究計畫核銷跟臨床工作的壓力下,又加上兩場院慶演出對我造成的重大負擔。在聽到這些以後,J博士馬上傳授我研究計畫核銷的各種秘訣,最後丟下兩句話:「自己身體要顧好。要懂得閃一些事情。」

  我會謹記在心的。
\item
  一直在想蒼白球日誌要直印還是橫印的問題

  於是翻閱了我手上有的書籍,基本上散文或小說絕大多數都是直印,閱讀起來比較舒服,唯有一本例外:蒼子的《潮間帶》{[}1{]}。

  所以我稍微又把這本翻了一下,其實也沒有因為橫印而難閱讀,那麼我還是橫印好了,對LaTeX{[}2{]}來說比較方便。
\item
  雜記:物價與其他{[}3{]}

  C院便當72元(因為餐廳要慶生會所以只賣預配好的便當不賣自助餐,菜色很糟糕,有我最討厭的糖醋豆皮,任何東西加了糖醋或美乃滋都會瞬間變得難以下嚥,不知道發明這兩種東西的人類在想什麼),「泰山」氣泡水兩瓶39元,「韓石館」石頭牛雜鍋145元(好吃且好飽)
\end{enumerate}

\hypertarget{ux6ce8ux91cb-comment-40}{%
\subsection{注釋 Comment}\label{ux6ce8ux91cb-comment-40}}

{[}1{]} 我與此書的因緣請見蒼白球日誌0037。

{[}2{]} 有關LaTeX請見蒼白球日誌0028。

{[}3{]} 新台幣計價。有關新台幣請參見蒼白球日誌0007。

\hypertarget{ux9644ux9304-appendix-39}{%
\subsection{附錄 Appendix}\label{ux9644ux9304-appendix-39}}

\hypertarget{ux84bcux767dux7403ux65e5ux8a8c004820191115}{%
\section{蒼白球日誌0048(20191115)}\label{ux84bcux767dux7403ux65e5ux8a8c004820191115}}

\hypertarget{ux65e5ux671f-date-47}{%
\subsection{日期 Date}\label{ux65e5ux671f-date-47}}

\begin{itemize}
\tightlist
\item
  世界協調時間2019年(中華民國108年,令和1年)11月15日 / Unix 紀元 18215
  日 / 星期五 / 蒼白球紀元第48日
\item
  November 15, 2019 (UTC) / 18215 days since Unix Epoch / Friday /
  Globus Pallidum day 48
\end{itemize}

\hypertarget{ux5e74ux9f61-age-47}{%
\subsection{年齡 Age}\label{ux5e74ux9f61-age-47}}

\begin{itemize}
\tightlist
\item
  33 years 6 months 23 days old / 2 years 1 months 3 days after
  acquiring ROC Surgical Pathology Licence
\item
  33 歲 6 個月 23 天 / 成為病理專科醫師 2 年 1 個月 3 天
\end{itemize}

\hypertarget{ux672cux6587-content-47}{%
\subsection{本文 Content}\label{ux672cux6587-content-47}}

\begin{enumerate}
\def\labelenumi{\arabic{enumi}.}
\item
  聽蛋白質結構的課

  老師提到最近四十年已經沒有人在手算蛋白質立體結構了,也不再製作實體模型,都是用電腦建模,然後儲存在線上資料庫。這讓我突然發現一件事情,當人類失去電力的時候,蛋白質的知識會馬上化為泡影,實在是很悲傷。

  然而人類其實沒有什麼方式可以處理這個問題,因為蛋白質立體結構複雜到紙本難以描述。或許只能過一天算一天,在電力允許的範圍內盡量維護現有的蛋白質資料庫。
\item
  我的粉專{[}2{]}在醫學生之間有點太過有名

  而造成了一些困擾。在病理部這梯醫學生結訓座談上面,當我跟醫學生因為無話可談而陷入沉默的時候,他們居然用「老師要多貼粉專喔!我們都有在看!」做結語結束這個座談,被這樣套交情實在太尷尬了。這種會模糊師生界線的工具實在是必須慎用。
\item
  雜記:物價與其他{[}3{]}

  幫母后線上買譜,兩份共170元,「麥當勞」大麥克餐122元,印譜(約70頁,訂騎馬釘)68元。
\end{enumerate}

\hypertarget{ux6ce8ux91cb-comment-41}{%
\subsection{注釋 Comment}\label{ux6ce8ux91cb-comment-41}}

{[}1{]} Facebook{[}2{]}
企業粉絲專頁是公開觸及桌上型電腦和手機用戶粉絲群的免費管道。Facebook
企業粉絲專頁是為企業、品牌、名人、公益理念和團體組織打造的專屬空間。Google
可能會將您的粉絲專頁編入索引,讓其他人更更容易找到您的企業。請使用桌上型電腦、手機以及專頁小助手應用程式來管理粉絲專頁。

{[}2{]}
Facebook(簡稱FB)是源於美國的社群網路服務及社會化媒體網站,總部位於美國加州聖馬刁郡門洛公園市。成立初期原名為「thefacebook」,名稱的靈感來自美國高中提供給學生包含相片和聯絡資料的通訊錄(或稱花名冊)之暱稱「face
book」。目前尚無官方的中文譯名,較為廣泛使用則為臉書。

{[}3{]} 新台幣計價。有關新台幣請參見蒼白球日誌0007。

\hypertarget{ux9644ux9304-appendix-40}{%
\subsection{附錄 Appendix}\label{ux9644ux9304-appendix-40}}

\hypertarget{ux84bcux767dux7403ux65e5ux8a8c004920191116}{%
\section{蒼白球日誌0049(20191116)}\label{ux84bcux767dux7403ux65e5ux8a8c004920191116}}

\hypertarget{ux65e5ux671f-date-48}{%
\subsection{日期 Date}\label{ux65e5ux671f-date-48}}

\begin{itemize}
\tightlist
\item
  世界協調時間2019年(中華民國108年,令和1年)11月16日 / Unix 紀元 18216
  日 / 星期六 / 蒼白球紀元第49日
\item
  November 16, 2019 (UTC) / 18216 days since Unix Epoch / Saturday /
  Globus Pallidum day 49
\item
  特殊註記:
\end{itemize}

\hypertarget{ux5e74ux9f61-age-48}{%
\subsection{年齡 Age}\label{ux5e74ux9f61-age-48}}

\begin{itemize}
\tightlist
\item
  33 years 6 months 24 days old / 2 years 1 months 4 days after
  acquiring ROC Surgical Pathology Licence
\item
  33 歲 6 個月 24 天 / 成為病理專科醫師 2 年 1 個月 4 天
\end{itemize}

\hypertarget{ux672cux6587-content-48}{%
\subsection{本文 Content}\label{ux672cux6587-content-48}}

早上去院慶演出,接著去跟母后一起教非洲鼓,然後回家幫母后準備明天的教案。因為這樣真的累所以今天不想寫什麼。

\hypertarget{ux6ce8ux91cb-comment-42}{%
\subsection{注釋 Comment}\label{ux6ce8ux91cb-comment-42}}

\hypertarget{ux9644ux9304-appendix-41}{%
\subsection{附錄 Appendix}\label{ux9644ux9304-appendix-41}}

\hypertarget{ux84bcux767dux7403ux65e5ux8a8c005020191117}{%
\section{蒼白球日誌0050(20191117)}\label{ux84bcux767dux7403ux65e5ux8a8c005020191117}}

\hypertarget{ux65e5ux671f-date-49}{%
\subsection{日期 Date}\label{ux65e5ux671f-date-49}}

\begin{itemize}
\tightlist
\item
  世界協調時間2019年(中華民國108年,令和1年)11月17日 / Unix 紀元 18217
  日 / 星期日 / 蒼白球紀元第50日
\item
  November 17, 2019 (UTC) / 18217 days since Unix Epoch / Sunday /
  Globus Pallidum day 50
\item
  特殊註記:
\end{itemize}

\hypertarget{ux5e74ux9f61-age-49}{%
\subsection{年齡 Age}\label{ux5e74ux9f61-age-49}}

\begin{itemize}
\tightlist
\item
  33 years 6 months 25 days old / 2 years 1 months 5 days after
  acquiring ROC Surgical Pathology Licence
\item
  33 歲 6 個月 25 天 / 成為病理專科醫師 2 年 1 個月 5 天
\end{itemize}

\hypertarget{ux672cux6587-content-49}{%
\subsection{本文 Content}\label{ux672cux6587-content-49}}

\begin{enumerate}
\def\labelenumi{\arabic{enumi}.}
\item
  人跟人之間的共振(本來對自己說好不寫的,但是不寫實在太悶了,只好寫出來)

  昨天又去了一場全桌一齊數落民進黨政府的應酬,真的非常悶。悶的原因是,一方面我在明年大選傾向支持民進黨,另一方面是嫉妒這群人能夠彼此達成如此高度的共鳴,突然覺得寂寞。跟某神人講了這件事之後,他提醒我,相同製程製造的人類,就會有相近的頻率,因此發生共振是非常自然的。

  於是我突然想起了和聲學。

  大三和弦之所以會聽起來廣大而和諧,原因是這三個音的泛音列達成高度重疊,而出現大幅度的共振。被調成同一個大三和絃的人們,遇到政治事件的時候就會聚在一起共振,產生廣大且單調的聲音。目前台灣的民粹政治就是調音的藝術,兩大陣營互相比賽如何將自己的支持者調成自己的調性,消滅異端調性,直到聽起來像是一個單調的和絃為止。因此,我們會看到各種無益於實際政治運作的,莫名其妙的共振。

  而在這些互相競賽的交響樂團裡面,都找不到共振的我,到底是什麼呢?我屬於哪一個調性?莫非我在十幾年來的各種個人矛盾裡面,已經走音,跑到十二平均律{[}1{]}以外,無法歸類為任何調性了嗎?真的好落寞喔,這走音的感覺。
\item
  食記 {[}2{]}

  「萬代福排骨飯」{[}3{]}
  店面裝潢看起來非常舊,帶著1988年左右的泡沫經濟時代風格,且散發著濃濃的酸筍臭味,如果不是因為心情差想嘗鮮,真的完全不會走進這家店。

  本來以為自己會後悔,結果料理意外地相當不錯。排骨飯70元,竟然是裹著厚厚麵包粉跟一層味精炸的,拿到的時候覺得痘痘隱隱作痛了一下。對,一層味精,我甚至可以在表面看到那個針狀的結晶。在這種崇尚天然調味的世界,我得講一句政治不正確的話:那層味精的正是排骨美味的秘密!當滿滿肉汁的雪白豬肉、回鍋油、被炸成金色的麵包粉、還有滿滿味精,所有會被醫師譴責的邪惡風味物質一起帶著濃濃的健康詛咒入口的時候,只有「爽」一個字可以形容。一切最古老的、不健康的食慾好像都在這一口裡面爆發,對營養師齊聲喊了一句「幹你娘」。

  然後旁邊的白菜滷跟酸筍,怎麼說呢,簡直就是把排骨的爽字增幅到了好幾倍。白菜滷充滿了會讓營養師加倍尖叫的濃濃醬油味,然後酸筍就跟臭豆腐一樣,越臭就越香醇。店面聞起來很噁心,正是因為他們的酸筍真香。豬血湯20元。我愛豬血湯成癡,所以對豬血湯的評論可能不公允,姑且略過。

  中華路「40年老店青草茶」

  大杯50元,一句話爆苦。我不會再喝這家了。
\item
  雜記

  \begin{itemize}
  \tightlist
  \item
    滑輪下拉10下三組,機械胸推10下三組,二頭肌10下三組,三頭肌10下三組,跑步8分鐘。
  \item
    跟母后去教鼓很開心。只有在做音樂的時候我們之間的矛盾才會稍微緩解一點。
  \item
    母后看到我的臉很糟糕,送我幾條昂貴品牌的試用品,晚上用了以後隔天就馬上感覺到改善。原來對於沒有天生麗質的人而言,外貌也是需要花大錢才能擁有的東西嗎?世界真是不公平啊。
  \end{itemize}
\end{enumerate}

\hypertarget{ux6ce8ux91cb-comment-43}{%
\subsection{注釋 Comment}\label{ux6ce8ux91cb-comment-43}}

{[}1{]}
十二平均律,又稱十二等程律,是一種音樂的定律方法,將一個八度平均分成十二等份,每等分稱為半音,是最主要的調音法。音高八度音指的是頻率加倍(即二倍頻率)。八度音的頻率分為十二等分,即是分為十二項的等比數列,也就是每個音的頻率為前一個音的2的12次方根。19世紀以後的十二平均律,一般而言主音定為A1(440Hz)。

{[}2{]} 新台幣計價。有關新台幣請參見蒼白球日誌0007。

{[}3{]} 在C市中心著名的萬代福電影院旁邊,於是很乾脆的取成這個名字。

\hypertarget{ux9644ux9304-appendix-42}{%
\subsection{附錄 Appendix}\label{ux9644ux9304-appendix-42}}

\hypertarget{ux84bcux767dux7403ux65e5ux8a8c005120191118}{%
\section{蒼白球日誌0051(20191118)}\label{ux84bcux767dux7403ux65e5ux8a8c005120191118}}

\hypertarget{ux65e5ux671f-date-50}{%
\subsection{日期 Date}\label{ux65e5ux671f-date-50}}

\begin{itemize}
\tightlist
\item
  世界協調時間2019年(中華民國108年,令和1年)11月18日 / Unix 紀元 18218
  日 / 星期一 / 蒼白球紀元第51日
\item
  November 18, 2019 (UTC) / 18218 days since Unix Epoch / Monday /
  Globus Pallidum day 51
\item
  特殊註記:
\end{itemize}

\hypertarget{ux5e74ux9f61-age-50}{%
\subsection{年齡 Age}\label{ux5e74ux9f61-age-50}}

\begin{itemize}
\tightlist
\item
  33 years 6 months 26 days old / 2 years 1 months 6 days after
  acquiring ROC Surgical Pathology Licence
\item
  33 歲 6 個月 26 天 / 成為病理專科醫師 2 年 1 個月 6 天
\end{itemize}

\hypertarget{ux672cux6587-content-50}{%
\subsection{本文 Content}\label{ux672cux6587-content-50}}

\begin{enumerate}
\def\labelenumi{\arabic{enumi}.}
\item
  11月18日凌晨,香港警方圍困香港理工大學。

  原本只是將重要歷史事件記錄一下,以作為時間的定錨,不過因為突然想到村上春樹,所以順便補述一段,顯示自己很有學問。

  在小說「1973年的彈珠玩具」(村上春樹,1992)裡,有許多對於日本大型示威運動「全共鬥」的描寫。在這些片段中,最精彩的是其中警方攻陷東京大學的場景,回頭看起來,這段與香港理工大學的圍城衝突意外地相似。香港與東京,1969與2019,完全不同性質的政權,完全不同性質的人民,完全不同性質的時代,相互衝突的場景卻驚人地類似。歷史或許真的會重複吧。

  僅摘錄此書片段如下:
  「他屬於一個政治性的社團,那個社團佔據了大學的九號館。」
  「晴朗得令人愉快的十一月某個下午,當第三機動隊衝進九號館時,據說韋瓦第的``調和的靈感''正以全音量播出{[}1{]}。」
\item
  雜記:物價與其他{[}2{]}

  \begin{itemize}
  \tightlist
  \item
    五月我有一份報告打得有瑕疵,結果十一月病人復發,開刀檢體又落到我手上,回頭看那份報告覺得事情做壞了,心情不好。
  \item
    C院午餐72元(紫米飯,洋蔥豬柳,小白菜,杏鮑菇,菜豆,黃豆芽),「台灣第一味」甘蔗青茶50元,用C院院慶的兩百塊禮券去H棟吃了貴又難吃又鹹的麵店,因為太糟所以我不想紀錄吃了什麼。
  \end{itemize}
\end{enumerate}

\hypertarget{ux6ce8ux91cb-comment-44}{%
\subsection{注釋 Comment}\label{ux6ce8ux91cb-comment-44}}

{[}1{]} 調和的靈感(L'estro
armonico,定年1711年),義大利作曲家韋瓦第(Antonio
Vivaldi,1678-1741)為絃樂器所做的十二首協奏曲。村上春樹並未提及此處播放的是十二首中的哪一首、哪一段,但咸信是充滿氣魄,戰鬥般的第三號協奏曲第三樂章。對了,村上春樹介紹的音樂沒有不好聽的,基本上村上春樹的小說不只是小說,還是史書跟可靠的樂評。

{[}2{]} 新台幣計價。有關新台幣請參見蒼白球日誌0007。

\hypertarget{ux9644ux9304-appendix-43}{%
\subsection{附錄 Appendix}\label{ux9644ux9304-appendix-43}}

\hypertarget{ux84bcux767dux7403ux65e5ux8a8c005220191119}{%
\section{蒼白球日誌0052(20191119)}\label{ux84bcux767dux7403ux65e5ux8a8c005220191119}}

\hypertarget{ux65e5ux671f-date-51}{%
\subsection{日期 Date}\label{ux65e5ux671f-date-51}}

\begin{itemize}
\tightlist
\item
  世界協調時間2019年(中華民國108年,令和1年)11月19日 / Unix 紀元 18219
  日 / 星期二 / 蒼白球紀元第52日
\item
  November 19, 2019 (UTC) / 18219 days since Unix Epoch / Tuesday /
  Globus Pallidum day 52
\item
  特殊註記:
\end{itemize}

\hypertarget{ux5e74ux9f61-age-51}{%
\subsection{年齡 Age}\label{ux5e74ux9f61-age-51}}

\begin{itemize}
\tightlist
\item
  33 years 6 months 27 days old / 2 years 1 months 7 days after
  acquiring ROC Surgical Pathology Licence
\item
  33 歲 6 個月 27 天 / 成為病理專科醫師 2 年 1 個月 7 天
\end{itemize}

\hypertarget{ux672cux6587-content-51}{%
\subsection{本文 Content}\label{ux672cux6587-content-51}}

\begin{enumerate}
\def\labelenumi{\arabic{enumi}.}
\item
  罕見地失眠了

  因此整個早上腦袋都在亂放電,做的事情包含恍神,亂逛網路,對男友胡言亂語,看「明朝」{[}1{]},總之無法做正事。如果被不知道內情的人看到這個景象,一定會以為病理科醫師{[}2{]}都是薪水小偷。算了我也懶得澄清了。

  睡了個午覺以後整個狀態就好多了,可以完成一些日常雜事。但睡了午覺以後反而開始擔心,晚上又失眠怎麼辦?
\item
  不知道該不該寫的事情

  但還是寫好了。某個為了自己的喜好而安排院慶管弦樂團演出的院內高官,今天送了一條蜂蜜蛋糕給我,還很興高采烈地詢問我「如果院內這個管弦樂團,要辦全場音樂會的話可以嗎?」這當然意謂著一場報酬微薄,對病理工作毫無助益的大型編曲,但我還是答應了。一方面我自己喜歡出風頭,一方面人在屋簷下不得不低頭,不得不跪舔。

  好討厭沒骨氣又容易被虛名引誘的自己。
\item
  雜記:物價與其他{[}3{]}

  \begin{itemize}
  \item
    C院午餐72元(洋蔥豬柳,紫米飯,菜豆,苦瓜,白花菜),C院晚餐72元(洋蔥豬肉片,白米飯,芥蘭,高麗菜,白花菜),「光泉」頂級鮮奶優酪一杯40元(購於全家便利商店,很貴但值得,因為他真的徹底無糖且菌種很豐富,我就是吃這個跟低卡餐瘦下來的),「朝日」十六茶29元,「多喝水」香檳氣泡水26元(他之前出的櫻花跟玫瑰風味都超難喝,但這次終於成功了,那個淡淡的葡萄香非常宜人)
  \item
    機械輔助引體向上10次三組,機械胸推10次三組,二頭肌10次三組,三頭肌10次3組
  \end{itemize}
\end{enumerate}

\hypertarget{ux6ce8ux91cb-comment-45}{%
\subsection{注釋 Comment}\label{ux6ce8ux91cb-comment-45}}

{[}1{]} 指駱以軍的書。有關此書與其掀起的爭議,請見蒼白球日誌0043。

{[}2{]} 有關這個職業請見蒼白球日誌0011。

{[}3{]} 新台幣計價。有關新台幣請參見蒼白球日誌0007。

\hypertarget{ux9644ux9304-appendix-44}{%
\subsection{附錄 Appendix}\label{ux9644ux9304-appendix-44}}

\hypertarget{ux84bcux767dux7403ux65e5ux8a8c005320191120}{%
\section{蒼白球日誌0053(20191120)}\label{ux84bcux767dux7403ux65e5ux8a8c005320191120}}

\hypertarget{ux65e5ux671f-date-52}{%
\subsection{日期 Date}\label{ux65e5ux671f-date-52}}

\begin{itemize}
\tightlist
\item
  世界協調時間2019年(中華民國108年,令和1年)11月20日 / Unix 紀元 18220
  日 / 星期三 / 蒼白球紀元第53日
\item
  November 20, 2019 (UTC) / 18220 days since Unix Epoch / Wednesday /
  Globus Pallidum day 53
\item
  特殊註記:
\end{itemize}

\hypertarget{ux5e74ux9f61-age-52}{%
\subsection{年齡 Age}\label{ux5e74ux9f61-age-52}}

\begin{itemize}
\tightlist
\item
  33 years 6 months 28 days old / 2 years 1 months 8 days after
  acquiring ROC Surgical Pathology Licence
\item
  33 歲 6 個月 28 天 / 成為病理專科醫師 2 年 1 個月 8 天
\end{itemize}

\hypertarget{ux672cux6587-content-52}{%
\subsection{本文 Content}\label{ux672cux6587-content-52}}

\begin{enumerate}
\def\labelenumi{\arabic{enumi}.}
\item
\item
  雜記:物價與其他{[}2{]}
\end{enumerate}

\hypertarget{ux6ce8ux91cb-comment-46}{%
\subsection{注釋 Comment}\label{ux6ce8ux91cb-comment-46}}

{[}1{]}

{[}2{]} 新台幣計價。有關新台幣請參見蒼白球日誌0007。

\hypertarget{ux9644ux9304-appendix-45}{%
\subsection{附錄 Appendix}\label{ux9644ux9304-appendix-45}}

\hypertarget{ux84bcux767dux7403ux65e5ux8a8c005420191121}{%
\section{蒼白球日誌0054(20191121)}\label{ux84bcux767dux7403ux65e5ux8a8c005420191121}}

\hypertarget{ux65e5ux671f-date-53}{%
\subsection{日期 Date}\label{ux65e5ux671f-date-53}}

\begin{itemize}
\tightlist
\item
  世界協調時間2019年(中華民國108年,令和1年)11月21日 / Unix 紀元 18221
  日 / 星期四 / 蒼白球紀元第54日
\item
  November 21, 2019 (UTC) / 18221 days since Unix Epoch / Thursday /
  Globus Pallidum day 54
\item
  特殊註記:
\end{itemize}

\hypertarget{ux5e74ux9f61-age-53}{%
\subsection{年齡 Age}\label{ux5e74ux9f61-age-53}}

\begin{itemize}
\tightlist
\item
  33 years 6 months 29 days old / 2 years 1 months 9 days after
  acquiring ROC Surgical Pathology Licence
\item
  33 歲 6 個月 29 天 / 成為病理專科醫師 2 年 1 個月 9 天
\end{itemize}

\hypertarget{ux672cux6587-content-53}{%
\subsection{本文 Content}\label{ux672cux6587-content-53}}

\begin{enumerate}
\def\labelenumi{\arabic{enumi}.}
\item
\item
  雜記:物價與其他{[}2{]}
\end{enumerate}

\hypertarget{ux6ce8ux91cb-comment-47}{%
\subsection{注釋 Comment}\label{ux6ce8ux91cb-comment-47}}

{[}1{]}

{[}2{]} 新台幣計價。有關新台幣請參見蒼白球日誌0007。

\hypertarget{ux9644ux9304-appendix-46}{%
\subsection{附錄 Appendix}\label{ux9644ux9304-appendix-46}}

\hypertarget{ux84bcux767dux7403ux65e5ux8a8c005520191122}{%
\section{蒼白球日誌0055(20191122)}\label{ux84bcux767dux7403ux65e5ux8a8c005520191122}}

\hypertarget{ux65e5ux671f-date-54}{%
\subsection{日期 Date}\label{ux65e5ux671f-date-54}}

\begin{itemize}
\tightlist
\item
  世界協調時間2019年(中華民國108年,令和1年)11月22日 / Unix 紀元 18222
  日 / 星期五 / 蒼白球紀元第55日
\item
  November 22, 2019 (UTC) / 18222 days since Unix Epoch / Friday /
  Globus Pallidum day 55
\item
  特殊註記:
\end{itemize}

\hypertarget{ux5e74ux9f61-age-54}{%
\subsection{年齡 Age}\label{ux5e74ux9f61-age-54}}

\begin{itemize}
\tightlist
\item
  33 years 6 months 30 days old / 2 years 1 months 10 days after
  acquiring ROC Surgical Pathology Licence
\item
  33 歲 6 個月 30 天 / 成為病理專科醫師 2 年 1 個月 10 天
\end{itemize}

\hypertarget{ux672cux6587-content-54}{%
\subsection{本文 Content}\label{ux672cux6587-content-54}}

\begin{enumerate}
\def\labelenumi{\arabic{enumi}.}
\item
\item
  雜記:物價與其他{[}2{]}
\end{enumerate}

\hypertarget{ux6ce8ux91cb-comment-48}{%
\subsection{注釋 Comment}\label{ux6ce8ux91cb-comment-48}}

{[}1{]}

{[}2{]} 新台幣計價。有關新台幣請參見蒼白球日誌0007。

\hypertarget{ux9644ux9304-appendix-47}{%
\subsection{附錄 Appendix}\label{ux9644ux9304-appendix-47}}

\hypertarget{ux84bcux767dux7403ux65e5ux8a8c005620191123}{%
\section{蒼白球日誌0056(20191123)}\label{ux84bcux767dux7403ux65e5ux8a8c005620191123}}

\hypertarget{ux65e5ux671f-date-55}{%
\subsection{日期 Date}\label{ux65e5ux671f-date-55}}

\begin{itemize}
\tightlist
\item
  世界協調時間2019年(中華民國108年,令和1年)11月23日 / Unix 紀元 18223
  日 / 星期六 / 蒼白球紀元第56日
\item
  November 23, 2019 (UTC) / 18223 days since Unix Epoch / Saturday /
  Globus Pallidum day 56
\item
  特殊註記:
\end{itemize}

\hypertarget{ux5e74ux9f61-age-55}{%
\subsection{年齡 Age}\label{ux5e74ux9f61-age-55}}

\begin{itemize}
\tightlist
\item
  33 years 7 months 0 days old / 2 years 1 months 11 days after
  acquiring ROC Surgical Pathology Licence
\item
  33 歲 7 個月 0 天 / 成為病理專科醫師 2 年 1 個月 11 天
\end{itemize}

\hypertarget{ux672cux6587-content-55}{%
\subsection{本文 Content}\label{ux672cux6587-content-55}}

\begin{enumerate}
\def\labelenumi{\arabic{enumi}.}
\item
\item
  雜記:物價與其他{[}2{]}
\end{enumerate}

\hypertarget{ux6ce8ux91cb-comment-49}{%
\subsection{注釋 Comment}\label{ux6ce8ux91cb-comment-49}}

{[}1{]}

{[}2{]} 新台幣計價。有關新台幣請參見蒼白球日誌0007。

\hypertarget{ux9644ux9304-appendix-48}{%
\subsection{附錄 Appendix}\label{ux9644ux9304-appendix-48}}

\hypertarget{ux84bcux767dux7403ux65e5ux8a8c005720191124}{%
\section{蒼白球日誌0057(20191124)}\label{ux84bcux767dux7403ux65e5ux8a8c005720191124}}

\hypertarget{ux65e5ux671f-date-56}{%
\subsection{日期 Date}\label{ux65e5ux671f-date-56}}

\begin{itemize}
\tightlist
\item
  世界協調時間2019年(中華民國108年,令和1年)11月24日 / Unix 紀元 18224
  日 / 星期日 / 蒼白球紀元第57日
\item
  November 24, 2019 (UTC) / 18224 days since Unix Epoch / Sunday /
  Globus Pallidum day 57
\item
  特殊註記:
\end{itemize}

\hypertarget{ux5e74ux9f61-age-56}{%
\subsection{年齡 Age}\label{ux5e74ux9f61-age-56}}

\begin{itemize}
\tightlist
\item
  33 years 7 months 1 days old / 2 years 1 months 12 days after
  acquiring ROC Surgical Pathology Licence
\item
  33 歲 7 個月 1 天 / 成為病理專科醫師 2 年 1 個月 12 天
\end{itemize}

\hypertarget{ux672cux6587-content-56}{%
\subsection{本文 Content}\label{ux672cux6587-content-56}}

\begin{enumerate}
\def\labelenumi{\arabic{enumi}.}
\item
\item
  雜記:物價與其他{[}2{]}
\end{enumerate}

\hypertarget{ux6ce8ux91cb-comment-50}{%
\subsection{注釋 Comment}\label{ux6ce8ux91cb-comment-50}}

{[}1{]}

{[}2{]} 新台幣計價。有關新台幣請參見蒼白球日誌0007。

\hypertarget{ux9644ux9304-appendix-49}{%
\subsection{附錄 Appendix}\label{ux9644ux9304-appendix-49}}

\hypertarget{ux84bcux767dux7403ux65e5ux8a8c005820191125}{%
\section{蒼白球日誌0058(20191125)}\label{ux84bcux767dux7403ux65e5ux8a8c005820191125}}

\hypertarget{ux65e5ux671f-date-57}{%
\subsection{日期 Date}\label{ux65e5ux671f-date-57}}

\begin{itemize}
\tightlist
\item
  世界協調時間2019年(中華民國108年,令和1年)11月25日 / Unix 紀元 18225
  日 / 星期一 / 蒼白球紀元第58日
\item
  November 25, 2019 (UTC) / 18225 days since Unix Epoch / Monday /
  Globus Pallidum day 58
\item
  特殊註記:
\end{itemize}

\hypertarget{ux5e74ux9f61-age-57}{%
\subsection{年齡 Age}\label{ux5e74ux9f61-age-57}}

\begin{itemize}
\tightlist
\item
  33 years 7 months 2 days old / 2 years 1 months 13 days after
  acquiring ROC Surgical Pathology Licence
\item
  33 歲 7 個月 2 天 / 成為病理專科醫師 2 年 1 個月 13 天
\end{itemize}

\hypertarget{ux672cux6587-content-57}{%
\subsection{本文 Content}\label{ux672cux6587-content-57}}

\begin{enumerate}
\def\labelenumi{\arabic{enumi}.}
\item
\item
  雜記:物價與其他{[}2{]}
\end{enumerate}

\hypertarget{ux6ce8ux91cb-comment-51}{%
\subsection{注釋 Comment}\label{ux6ce8ux91cb-comment-51}}

{[}1{]}

{[}2{]} 新台幣計價。有關新台幣請參見蒼白球日誌0007。

\hypertarget{ux9644ux9304-appendix-50}{%
\subsection{附錄 Appendix}\label{ux9644ux9304-appendix-50}}

\hypertarget{ux84bcux767dux7403ux65e5ux8a8c005920191126}{%
\section{蒼白球日誌0059(20191126)}\label{ux84bcux767dux7403ux65e5ux8a8c005920191126}}

\hypertarget{ux65e5ux671f-date-58}{%
\subsection{日期 Date}\label{ux65e5ux671f-date-58}}

\begin{itemize}
\tightlist
\item
  世界協調時間2019年(中華民國108年,令和1年)11月26日 / Unix 紀元 18226
  日 / 星期二 / 蒼白球紀元第59日
\item
  November 26, 2019 (UTC) / 18226 days since Unix Epoch / Tuesday /
  Globus Pallidum day 59
\item
  特殊註記:
\end{itemize}

\hypertarget{ux5e74ux9f61-age-58}{%
\subsection{年齡 Age}\label{ux5e74ux9f61-age-58}}

\begin{itemize}
\tightlist
\item
  33 years 7 months 3 days old / 2 years 1 months 14 days after
  acquiring ROC Surgical Pathology Licence
\item
  33 歲 7 個月 3 天 / 成為病理專科醫師 2 年 1 個月 14 天
\end{itemize}

\hypertarget{ux672cux6587-content-58}{%
\subsection{本文 Content}\label{ux672cux6587-content-58}}

\begin{enumerate}
\def\labelenumi{\arabic{enumi}.}
\item
\item
  雜記:物價與其他{[}2{]}
\end{enumerate}

\hypertarget{ux6ce8ux91cb-comment-52}{%
\subsection{注釋 Comment}\label{ux6ce8ux91cb-comment-52}}

{[}1{]}

{[}2{]} 新台幣計價。有關新台幣請參見蒼白球日誌0007。

\hypertarget{ux9644ux9304-appendix-51}{%
\subsection{附錄 Appendix}\label{ux9644ux9304-appendix-51}}

\hypertarget{ux84bcux767dux7403ux65e5ux8a8c006020191127}{%
\section{蒼白球日誌0060(20191127)}\label{ux84bcux767dux7403ux65e5ux8a8c006020191127}}

\hypertarget{ux65e5ux671f-date-59}{%
\subsection{日期 Date}\label{ux65e5ux671f-date-59}}

\begin{itemize}
\tightlist
\item
  世界協調時間2019年(中華民國108年,令和1年)11月27日 / Unix 紀元 18227
  日 / 星期三 / 蒼白球紀元第60日
\item
  November 27, 2019 (UTC) / 18227 days since Unix Epoch / Wednesday /
  Globus Pallidum day 60
\item
  特殊註記:
\end{itemize}

\hypertarget{ux5e74ux9f61-age-59}{%
\subsection{年齡 Age}\label{ux5e74ux9f61-age-59}}

\begin{itemize}
\tightlist
\item
  33 years 7 months 4 days old / 2 years 1 months 15 days after
  acquiring ROC Surgical Pathology Licence
\item
  33 歲 7 個月 4 天 / 成為病理專科醫師 2 年 1 個月 15 天
\end{itemize}

\hypertarget{ux672cux6587-content-59}{%
\subsection{本文 Content}\label{ux672cux6587-content-59}}

\begin{enumerate}
\def\labelenumi{\arabic{enumi}.}
\item
\item
  雜記:物價與其他{[}2{]}
\end{enumerate}

\hypertarget{ux6ce8ux91cb-comment-53}{%
\subsection{注釋 Comment}\label{ux6ce8ux91cb-comment-53}}

{[}1{]}

{[}2{]} 新台幣計價。有關新台幣請參見蒼白球日誌0007。

\hypertarget{ux9644ux9304-appendix-52}{%
\subsection{附錄 Appendix}\label{ux9644ux9304-appendix-52}}

\hypertarget{ux84bcux767dux7403ux65e5ux8a8c006120191128}{%
\section{蒼白球日誌0061(20191128)}\label{ux84bcux767dux7403ux65e5ux8a8c006120191128}}

\hypertarget{ux65e5ux671f-date-60}{%
\subsection{日期 Date}\label{ux65e5ux671f-date-60}}

\begin{itemize}
\tightlist
\item
  世界協調時間2019年(中華民國108年,令和1年)11月28日 / Unix 紀元 18228
  日 / 星期四 / 蒼白球紀元第61日
\item
  November 28, 2019 (UTC) / 18228 days since Unix Epoch / Thursday /
  Globus Pallidum day 61
\item
  特殊註記:
\end{itemize}

\hypertarget{ux5e74ux9f61-age-60}{%
\subsection{年齡 Age}\label{ux5e74ux9f61-age-60}}

\begin{itemize}
\tightlist
\item
  33 years 7 months 5 days old / 2 years 1 months 16 days after
  acquiring ROC Surgical Pathology Licence
\item
  33 歲 7 個月 5 天 / 成為病理專科醫師 2 年 1 個月 16 天
\end{itemize}

\hypertarget{ux672cux6587-content-60}{%
\subsection{本文 Content}\label{ux672cux6587-content-60}}

\begin{enumerate}
\def\labelenumi{\arabic{enumi}.}
\item
\item
  雜記:物價與其他{[}2{]}
\end{enumerate}

\hypertarget{ux6ce8ux91cb-comment-54}{%
\subsection{注釋 Comment}\label{ux6ce8ux91cb-comment-54}}

{[}1{]}

{[}2{]} 新台幣計價。有關新台幣請參見蒼白球日誌0007。

\hypertarget{ux9644ux9304-appendix-53}{%
\subsection{附錄 Appendix}\label{ux9644ux9304-appendix-53}}

\hypertarget{ux84bcux767dux7403ux65e5ux8a8c006220191129}{%
\section{蒼白球日誌0062(20191129)}\label{ux84bcux767dux7403ux65e5ux8a8c006220191129}}

\hypertarget{ux65e5ux671f-date-61}{%
\subsection{日期 Date}\label{ux65e5ux671f-date-61}}

\begin{itemize}
\tightlist
\item
  世界協調時間2019年(中華民國108年,令和1年)11月29日 / Unix 紀元 18229
  日 / 星期五 / 蒼白球紀元第62日
\item
  November 29, 2019 (UTC) / 18229 days since Unix Epoch / Friday /
  Globus Pallidum day 62
\item
  特殊註記:
\end{itemize}

\hypertarget{ux5e74ux9f61-age-61}{%
\subsection{年齡 Age}\label{ux5e74ux9f61-age-61}}

\begin{itemize}
\tightlist
\item
  33 years 7 months 6 days old / 2 years 1 months 17 days after
  acquiring ROC Surgical Pathology Licence
\item
  33 歲 7 個月 6 天 / 成為病理專科醫師 2 年 1 個月 17 天
\end{itemize}

\hypertarget{ux672cux6587-content-61}{%
\subsection{本文 Content}\label{ux672cux6587-content-61}}

\begin{enumerate}
\def\labelenumi{\arabic{enumi}.}
\item
\item
  雜記:物價與其他{[}2{]}
\end{enumerate}

\hypertarget{ux6ce8ux91cb-comment-55}{%
\subsection{注釋 Comment}\label{ux6ce8ux91cb-comment-55}}

{[}1{]}

{[}2{]} 新台幣計價。有關新台幣請參見蒼白球日誌0007。

\hypertarget{ux9644ux9304-appendix-54}{%
\subsection{附錄 Appendix}\label{ux9644ux9304-appendix-54}}

\hypertarget{ux84bcux767dux7403ux65e5ux8a8c006320191130}{%
\section{蒼白球日誌0063(20191130)}\label{ux84bcux767dux7403ux65e5ux8a8c006320191130}}

\hypertarget{ux65e5ux671f-date-62}{%
\subsection{日期 Date}\label{ux65e5ux671f-date-62}}

\begin{itemize}
\tightlist
\item
  世界協調時間2019年(中華民國108年,令和1年)11月30日 / Unix 紀元 18230
  日 / 星期六 / 蒼白球紀元第63日
\item
  November 30, 2019 (UTC) / 18230 days since Unix Epoch / Saturday /
  Globus Pallidum day 63
\item
  特殊註記:
\end{itemize}

\hypertarget{ux5e74ux9f61-age-62}{%
\subsection{年齡 Age}\label{ux5e74ux9f61-age-62}}

\begin{itemize}
\tightlist
\item
  33 years 7 months 7 days old / 2 years 1 months 18 days after
  acquiring ROC Surgical Pathology Licence
\item
  33 歲 7 個月 7 天 / 成為病理專科醫師 2 年 1 個月 18 天
\end{itemize}

\hypertarget{ux672cux6587-content-62}{%
\subsection{本文 Content}\label{ux672cux6587-content-62}}

\begin{enumerate}
\def\labelenumi{\arabic{enumi}.}
\item
\item
  雜記:物價與其他{[}2{]}
\end{enumerate}

\hypertarget{ux6ce8ux91cb-comment-56}{%
\subsection{注釋 Comment}\label{ux6ce8ux91cb-comment-56}}

{[}1{]}

{[}2{]} 新台幣計價。有關新台幣請參見蒼白球日誌0007。

\hypertarget{ux9644ux9304-appendix-55}{%
\subsection{附錄 Appendix}\label{ux9644ux9304-appendix-55}}

\end{document}
